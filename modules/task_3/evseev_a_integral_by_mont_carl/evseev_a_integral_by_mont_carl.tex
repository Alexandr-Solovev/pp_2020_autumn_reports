\documentclass{report}

\usepackage[T2A]{fontenc}
\usepackage[utf8]{luainputenc}
\usepackage[english, russian]{babel}
\usepackage[pdftex]{hyperref}
\usepackage[14pt]{extsizes}
\usepackage{listings}
\usepackage{color}
\usepackage{geometry}
\usepackage{enumitem}
\usepackage{multirow}
\usepackage{graphicx}
\usepackage{indentfirst}
\usepackage{amsmath}

\geometry{a4paper,top=2cm,bottom=2cm,left=2.5cm,right=1.5cm}
\setlength{\parskip}{0.5cm}
\setlist{nolistsep, itemsep=0.3cm,parsep=0pt}

\usepackage{xcolor}
\definecolor{weborange}{RGB}{255,165,0}
\definecolor{codepurple}{RGB}{220,20,60}
\definecolor{codecyan}{RGB}{135,0,198}
\definecolor{codeblue}{RGB}{69,97,189}
\definecolor{backcolour}{RGB}{230,230,220}

\usepackage{listings}
\lstset{language=C++,
	basicstyle=\ttfamily\footnotesize,
	backgroundcolor=\color{backcolour},
    	emph={int, double, time_t, char, new},
    	emphstyle={\color{blue}},
    	commentstyle=\color{green!60!black},
 	keywordstyle={\color{codepurple}},
 	keywords=[2]{std},
    	keywordstyle=[2]{\color{weborange}},
    	otherkeywords = {:},
	morekeywords = [2]{:},
   	stringstyle=\color{codepurple},
	morecomment=[l][\color{codecyan}]{\#}, 
	tabsize=4,
	breaklines=true,                               
  	breakatwhitespace=true,
  	title=\lstname,  
}

\makeatletter
\renewcommand\@biblabel[1]{#1.\hfil}
\makeatother

\begin{document}

\begin{titlepage}

\begin{center}
Министерство науки и высшего образования Российской Федерации
\end{center}

\begin{center}
Федеральное государственное автономное образовательное учреждение высшего образования \\
Национальный исследовательский Нижегородский государственный университет им. Н.И. Лобачевского
\end{center}

\begin{center}
Институт информационных технологий, математики и механики
\end{center}

\vspace{4em}

\begin{center}
\textbf{\LargeОтчет по лабораторной работе} \\
\end{center}
\begin{center}
\textbf{\Large«Вычисление многомерных интегралов методом Монте-Карло»} \\
\end{center}

\vspace{4em}

\newbox{\lbox}
\savebox{\lbox}{\hbox{text}}
\newlength{\maxl}
\setlength{\maxl}{\wd\lbox}
\hfill\parbox{7cm}{
\hspace*{5cm}\hspace*{-5cm}\textbf{Выполнил:} \\ студент группы 381806-2 \\ Евсеев А. Д.\\
\\
\hspace*{5cm}\hspace*{-5cm}\textbf{Проверил:}\\ доцент кафедры МОСТ, \\ кандидат технических наук \\ Сысоев А. В.\\
}
\vspace{\fill}

\begin{center} Нижний Новгород \\ 2020 \end{center}

\end{titlepage}

\setcounter{page}{2}

% Содержание
\tableofcontents
\newpage

% Введение
\section*{Введение}
\addcontentsline{toc}{section}{Введение}
\par Датой рождения метода Монте - Карло признано считать 1949 год, когда американские ученые Н. Метрополис и С. Услам опубликовали статью под названием «Метод Монте - Карло», в которой были изложены принципы этого метода. Название метода происходит от названия города Монте – Карло, славившегося своими игорными заведениями, непременным атрибутом которых являлась рулетка – одно из простейших средств получения случайных чисел с хорошим равномерным распределением, на использовании которых основан этот метод.

\par Метод Монте – Карло это статистический метод. Его используют при вычислении сложных интегралов, решении систем алгебраических уравнений высокого порядка, моделировании поведения элементарных частиц, в теориях передачи информации, при исследовании сложных экономических систем
\newpage

% Постановка задачи
\section*{Постановка задачи}
\addcontentsline{toc}{section}{Постановка задачи}
\par В данном лабораторной работе требуется реализовать последовательную и параллельную схему метода Монте-Карло для вычисления многомерных интегралов и  провести анализ по времени работы. А также подтвердить корректность, объяснить полученные результаты и сделать выводы.
\par Параллельная версия будет реализована при помощи технологии MPI. Проверка корректности будет осуществлена при помощи библиотеки для модульного тестирования Google C++ Testing Framework.
\newpage

% Описание алгоритма
\section*{Описание алгоритма}
\addcontentsline{toc}{section}{Описание алгоритма}
\par Задан тройной интеграл вида:
$$\iiint\limits_D f(x,y,z) \,dx\,dy\,dz, $$
где подинтегральная функция задана в трехмерном промежутке (параллелепипеде) $D=[a_1,b_1]\times\*[a_2,b_2]\times[a_3,b_3]$.
\par Введем трехмерную случайную величину $X$, имеющую в замкнутной области равномерное распределение вероятностей с дифференциальной функцией плотности
$$p(x)=\frac{1}{\mu(D)}.$$
\par Функция плотности равномерного распределения есть величина постоянная, поэтому введем ее под знак интеграла следующим образом:
$$I=\iiint\limits_D f(x,y,z)\frac{\mu(D)}{\mu(D)} \,dx\,dy\,dz.$$
\par Вынесем числитель дроби за знак интеграла, т. е.
$$I=\mu(D)\iiint\limits_D f(x,y,z)\frac{1}{\mu(D)} \,dx\,dy\,dz.$$
\par Получается, что тройной интеграл - математическое ожидание от функции $f(x)$ случайной величины в предположении, что случайная величина $X$ распределена равномерно с рассмотренной выше плотностью. Значит, мы можем записать
$$I=\mu(D)M[f(\xi)],$$
где $\xi=(\xi_1,\xi_2,\xi_3)$.
\par В свою очередь, математическое ожидание может быть оценено с помощью арифметического среднего. Тогда приближенное значение тройного интеграла будет определяться приближенной формулой
$$I\approx\frac{\mu(D)}{N}\sum\limits_{k=1}^N f(\xi^{(k)}),$$
где $\xi^{(k)} \in D$ - значение случайной величины $X$ в $k$-м испытании.
\par Чтобы смоделировать выборку трехмерной случайной величины $Х$, равномерно распределенной в трехмерном промежутке $D$, используются псевдослучайные числа. Для этого в каждом испытании с номером $k$ выбирают 3 псевдослучайных числа $r_i^{(k)}$, $i=\overline{1,3}$, и по ним определяют координаты случайных величин $\xi_i^{(k)}=a_i+(b_i-a_i)\*r_i^{(k)}$, $i=\overline{1,3}$, псевдослучайной точки $\xi^{(k)}$.
\par Таким образом, техника применения метода Монте-Карло будет заключаться в генерировании в области $D$ псевдослучайных чисел и применении приближенной формулы.
\newpage

% Схема распараллеливания
\section*{Схема распараллеливания}
\addcontentsline{toc}{section}{Схема распараллеливания}
\par Для распараллеливания подсчета тройного интеграла методом Монте-Карло, разделим параллелепипед $D$ на равные части: отрезки $[a_1,b_1]$ и $[a_2,b_2]$ оставляем прежними, а $[a_3,b_3]$ делим на части одинаковой длины, количество этих частей равно запускаемым процессам. Получаем несколько параллелепипедов. Распределим изначальное количество испытаний между ними (последнему параллелепипеду может достаться меньше, в случае если невозможно разделить испытания на процессы без остатка). После этого, параллельно запускаем последовательный алгоритм на всех процессах.
\par В итоге, каждый процесс считает свою часть интеграла и отправляет ее корневому процессу, после чего тот суммирует все присланные ему части и возвращает уже итоговый интеграл.
\newpage

% Описание программной реализации
\section*{Описание программной реализации}
\addcontentsline{toc}{section}{Описание программной реализации}
\par Для начала опишем последовательную схему алгоритма.
\begin{lstlisting}
double MonteCarloSequential (std::vector<double> start_point,
                             double edge_size,
                             double (*pfunction) (std::vector<double>),
                             bool (*parea) (std::vector<double>),
                             unsigned int dimension,
                             int number_of_points);
\end{lstlisting}
Первым аргументом является угловая точка - \lstinline$std::vector<double> start_point$, далее \lstinline$double edge_size$ - размер грани квадрата(куба и т.д), следующим аргументом передается указатель на используемую функцию, которая была выбрана для вычисления интеграла, далее идут границы интегрирования.
Внутри функции создаем генератор случаных чисел с равномерным распределением в интервале от 0 до 1:
\begin{lstlisting}
std::mt19937 gen;
  gen.seed (static_cast<unsigned int> (time (0)));
  std::uniform_real_distribution<> urd (0, 1);
\end{lstlisting}
По методу Монте-Карло начинаем генерировать точки в ограничивающим квадрате(куба и тд, зависит от размера пространства, в котором считаем)
\begin{lstlisting}
std::vector<double> points (dimension * number_of_points);
\end{lstlisting}
Берём координату угловой точки в измерении и отходим на значение в интервале от 0 до edge size. В результате получим набор точек в пространстве размером dimension \begin{lstlisting}
for (unsigned int i = 0; i < points.size (); i++) {
    points[i] = start_point[i % dimension] + urd (gen) * edge_size;
  }\end{lstlisting}
Далее проходим по всем точкам, ниже выбираем отделбную точку, проверяем, что данная точка принадлежит интегрируюмой области (площадь/ объём, который нужно найти), Ищем среднее арифметическое для значений точек, которые попали в область, Зная количество попаших точек, грубо рассчитываем, какую часть от ограничивающего квадрата(куба и тд) занимает целевая область. В конце умножаем найденную часть на среднее значение для попавших точек.
\begin{lstlisting}  
for (int i = 0; i < number_of_points; ++i) {
    std::vector<double> cpoint;
    // Ниже выбираем отделбную точку
    for (unsigned int j = 0; j < dimension; ++j) {
      cpoint.push_back (p[dimension * i + j]);
    }
    if (parea (cpoint)) {
      n++;
      val += pfunction (cpoint);
    }
  }

  val = val / n;
  double area = std::pow (edge_size, dimension) * n / number_of_points;

  return area * val;
  \end{lstlisting}
\par Перейдем к параллельной схеме
\par Она получает такие же параметры, что и ее последовательная версия
\begin{lstlisting}
double MonteCarloParallel (std::vector<double> start_point,
                           double edge_size,
                           double (*pfunction) (std::vector<double>),
                           bool (*parea) (std::vector<double>),
                           unsigned int dimension,
                           int number_of_points);
\end{lstlisting}
Узнаём в каждом процессе количество всех процессов и номер текущего процесса
 \begin{lstlisting}
 MPI_Comm_size (MPI_COMM_WORLD, &size);
 MPI_Comm_rank (MPI_COMM_WORLD, &rank);
 \end{lstlisting} Используется распределение вычисление среди процессов с помощью MPI. Здесь у нас есть root процесс с rank == 0, в котором генерируются наши точки. Далее их количество делится на равные части и каждому процессу достаётся своя часть Каждый процесс проходит по своим точкам, ищет те, которые входят в нужную нам область и считает их количество и сумму их значений.
Затем эти результаты передаются обратно в root процесс, где суммируются.
Дальше всё аналогично последовательному алгоритму\begin{lstlisting}
  int delta = number_of_points / size;
  std::vector<double> p (dimension * number_of_points);

  if (rank == 0) {
    std::mt19937 gen;
    gen.seed (static_cast<unsigned int> (time (NULL)));
    std::uniform_real_distribution<> urd (0, 1);
    for (unsigned int i = 0; i < p.size (); i++) {
      p[i] = start_point[i % dimension] + urd (gen) * edge_size;
    }
  }

  std::vector<double> loc_p (delta * dimension);

  MPI_Scatter (&p[0], delta * dimension, MPI_DOUBLE, &loc_p[0], delta, dimension, MPI_DOUBLE, 0, MPI_COMM_WORLD);     

  int loc_n = 0;
  int n = 0;
  double loc_val = 0;
  double val = 0;

  for (int i = 0; i < delta; ++i) {
    std::vector<double> cpoint;
    for (unsigned int j = 0; j < dimension; ++j) {
      cpoint.push_back (loc_p[dimension * i + j]);
    }
    if (parea (cpoint)) {
      loc_n++;
      loc_val += pfunction (cpoint);
    }
  }

  MPI_Reduce (&loc_n, &n, 1, MPI_INT, MPI_SUM, 0, MPI_COMM_WORLD);  
 
  MPI_Reduce (&loc_val, &val, 1, MPI_DOUBLE, MPI_SUM, 0, MPI_COMM_WORLD);
 \end{lstlisting}
\newpage

% Подтверждение корректности
\section*{Подтверждение корректности}
\addcontentsline{toc}{section}{Подтверждение корректности}
Для подтверждения корректности в программе представлен набор тестов, разработанных с помощью использования Google C++ Testing Framework.
\par Набор представляет из себя тесты, которые проверяют:
\begin{itemize}
\item Корректность входных данных (количество испытаний не может быть отрицательным);
\item Корректность вычислений (проверка результатов работы последовательной версии, идет сравнение с результатами, полученными аналитическим интегрированием);
\item Эффективность по времени работы последовательной и параллельной функции.
\end{itemize}
\par Прохождение всего набора тестов подтверждает правильность реализации программы.
\newpage

% Результаты экспериментов
\section*{Результаты экспериментов}
\addcontentsline{toc}{section}{Результаты экспериментов}
Вычислительные эксперименты для оценки временной эффективности параллельной реализации вычисления тройных интегралов методом Монте-Карло проводились на оборудовании со следующей конфигурацией:
\begin{itemize}
\item Процессор: Intel Core i5-8300H, 4000 MHz, физических ядер: 4, потоков: 8;
\item Оперативная память: 8192 МБ (DDR4), 2666 MHz;
\item ОС: Microsoft Windows 10 Pro, версия 1909 сборка 18363.1198.
\end{itemize}

\par Результаты экспериментов представлены в Таблице 1.

\begin{table}[!h]
\caption{Результаты вычислительных экспериментов}
\centering
\begin{tabular}{| r | r | r | r |}
\hline
Количество процессов & Последовательно & Параллельно & Ускорение  \\[5pt]
\hline
1        & 7.78654        & 7.68045     & 0.106       \\
2        & 7.84059        & 4.54447     & 3.29       \\
3        & 7.60638        & 3.52266     & 4.08       \\
4        & 7.77826        & 3.21183     & 4.56       \\
5        & 7.81269        & 3.02723     & 4.78       \\
6        & 8.13739        & 2.83937     & 5.29       \\
7        & 7.85683        & 2.74129     & 5.39	    \\
8        & 8.24313        & 2.75751     & 5.48	  \\
\hline
\end{tabular}
\end{table}

\par Из данных, представленных в Таблице 1 видно, что параллельная версия быстрее последовательной, а ее ускорение в общем случае пропорционально количеству запускаемых процессов. Такой хороший результат объясняется тем, что мы использовали относительно большое количество испытаний, из-за чего последовательная версия выполнялась гораздо дольше.
\newpage

% Заключение
\section*{Заключение}
\addcontentsline{toc}{section}{Заключение}
В результате лабораторной работы были разработаны последовательная и параллельная реализация подсчета тройных интегралов методом Монте-Карло.
\par Главной задачей данной лабораторной работы была реализация параллельной версии, которая должна была быть эффективнее последовательной. Эта задача была успешно достигнута, о чем говорят результаты экспериментов, проведенных в ходе работы.
\par Кроме того, были разработаны и доведены до успешного выполнения тесты, созданные для данного программного проекта с использованием Google C++ Testing Framework и необходимые для подтверждения корректности работы программы.
\newpage

% Список литературы
\begin{thebibliography}{1}
\addcontentsline{toc}{section}{Список литературы}
\bibitem{Voytishek} Войтишек А.В. Основы метода Монте-Карло: Учеб. пособие / Новосиб. гос. ун-т. Новосибирск, 2010, 108 с. 
\bibitem{Zorinl} Зорин А.В, Федоткин М.А. Методы Монте-Карло для параллельных вычислений. — М.: Изд-во Московского университета, 2013.
\end{thebibliography}
\newpage

% Приложение
\section*{Приложение}
\addcontentsline{toc}{section}{Приложение}


\begin{lstlisting}
//integral_by_mont_carl.h
#ifndef MODULES_TASK_3_EVSEEV_A_INTERGRAL_BY_MONT_CARL_INTEGRAL_BY_MONT_CARL_H_
#define MODULES_TASK_3_EVSEEV_A_INTERGRAL_BY_MONT_CARL_INTEGRAL_BY_MONT_CARL_H_

#include <vector>

double MonteCarloParallel(std::vector<double> a, double sd,
                                   double (*pfunction)(std::vector<double>),
                                   bool (*parea)(std::vector<double>),
                                   unsigned int dm, int p_count);
double MonteCarloSequential(
    std::vector<double> a, double sd,
    double (*pfunction)(std::vector<double>), bool (*parea)(std::vector<double>),
    unsigned int dm, int p_count);

#endif  // MODULES_TASK_3_EVSEEV_A_INTERGRAL_BY_MONT_CARL_INTEGRAL_BY_MONT_CARL_H_
\end{lstlisting}

\begin{lstlisting}
// integral_by_mont_carl.cpp
// Copyright 2020 Evseev Alexander
#include <mpi.h>
#include <algorithm>
#include <ctime>
#include <iostream>
#include <random>
#include <vector>

#include "../../../modules/task_3/evseev_a_intergral_by_mont_carl/integral_by_mont_carl.h"

double MonteCarloSequential(
    std::vector<double> a, double sd,
    double (*pfunction)(std::vector<double>), bool (*parea)(std::vector<double>),
    unsigned int dim, int p_count) {
  std::mt19937 gen;
  gen.seed(static_cast<unsigned int>(time(0)));
  std::uniform_real_distribution<> urd(0, 1);

  std::vector<double> p(dim * p_count);
  for (unsigned int i = 0; i < p.size(); i++) {
    p[i] = a[i % dim] + urd(gen) * sd;
  }

  int n = 0;
  double val = 0;

  for (int i = 0; i < p_count; ++i) {
    std::vector<double> cpoint;
    for (unsigned int j = 0; j < dim; ++j) {
      cpoint.push_back(p[dim * i + j]);
    }
    if (parea(cpoint)) {
      n++;
      val += pfunction(cpoint);
    }
  }

  val = val / n;
  double area = std::pow(sd, dim) * n / p_count;

  return area * val;
}

double MonteCarloParallel(std::vector<double> a, double sd,
                                   double (*pfunction)(std::vector<double>),
                                   bool (*parea)(std::vector<double>),
                                   unsigned int dim, int p_count) {
  if (a.size() != dim) throw(1);

  int size, rank;
  MPI_Comm_size(MPI_COMM_WORLD, &size);
  MPI_Comm_rank(MPI_COMM_WORLD, &rank);

  int delta = p_count / size;
  std::vector<double> p(dim * p_count);

  if (rank == 0) {
    std::mt19937 gen;
    gen.seed(static_cast<unsigned int>(time(NULL)));
    std::uniform_real_distribution<> urd(0, 1);
    for (unsigned int i = 0; i < p.size(); i++) {
      p[i] = a[i % dim] + urd(gen) * sd;
    }
  }

  std::vector<double> loc_p(delta * dim);

  MPI_Scatter(&p[0], delta * dim, MPI_DOUBLE, &loc_p[0],
              delta * dim, MPI_DOUBLE, 0, MPI_COMM_WORLD);

  int loc_n = 0;
  int n = 0;
  double loc_val = 0;
  double val = 0;

  for (int i = 0; i < delta; ++i) {
    std::vector<double> cpoint;
    for (unsigned int j = 0; j < dim; ++j) {
      cpoint.push_back(loc_p[dim * i + j]);
    }
    if (parea(cpoint)) {
      loc_n++;
      loc_val += pfunction(cpoint);
    }
  }
  MPI_Reduce(&loc_n, &n, 1, MPI_INT, MPI_SUM, 0,
             MPI_COMM_WORLD);
  MPI_Reduce(&loc_val, &val, 1, MPI_DOUBLE, MPI_SUM, 0, MPI_COMM_WORLD);

  val = val / n;
  double area = std::pow(sd, dim) * n / p_count;

  return area * val;
}
\end{lstlisting}

\begin{lstlisting}
// main.cpp
// Copyright 2020 Evseev Alexander
#include <gtest-mpi-listener.hpp>
#include <gtest/gtest.h>
#include <vector>
#include "./integral_by_mont_carl.h"

double function_const(std::vector<double> args) { return 1; }

double func_quadratic(std::vector<double> args) { return args[0] * args[0]; }

double func_linear(std::vector<double> args) { return args[0] + args[1]; }

bool area_2_parabols(std::vector<double> args) {
  return (args[1] > (args[0] * args[0] - 1)) &&
         (args[1] < -args[0] * args[0] + 1);
}

bool intcircle(std::vector<double> args) {
  return (args[0] * args[0] + args[1] * args[1]) < 4;
}

bool intsphere1(std::vector<double> args) {
  return (args[0] * args[0] + args[1] * args[1] + args[2] * args[2]) < 4;
}

bool intsphere2(std::vector<double> args) {
  return (args[0] * args[0] + args[1] * args[1] + args[2] * args[2] +
          args[3] * args[3]) < 4;
}

bool intsphere3(std::vector<double> args) {
  return (args[0] > 0 && args[2] > 0 &&
          (args[0] * args[0] + args[1] * args[1] + args[2] * args[2]) < 4);
}

TEST(Monte_Karlo_MPI, TEST_const_func_int_three_sphere) {
  int rank;
  MPI_Comm_rank(MPI_COMM_WORLD, &rank);

  const double threr = 10.0;

  std::vector<double> s_point = {-2, -2, -2, -2};

  double integration_result = MonteCarloParallel(
      s_point, 4, function_const, intsphere2, 4, 10000);
  if (rank == 0) {
    ASSERT_NEAR(3.14 * 3.14 * 16 / 2, integration_result, threr);
  }
}

TEST(Monte_Karlo_MPI, TEST_quadr_func_int_sphere) {
  int rank;
  MPI_Comm_rank(MPI_COMM_WORLD, &rank);

  const double threr = 5.0;

  std::vector<double> s_point = {-2, -2, -2};

  double integration_result = MonteCarloParallel(
      s_point, 4, func_quadratic, intsphere3, 3, 10000);
  if (rank == 0) {
    ASSERT_NEAR(3.14 * 32 / 15, integration_result, threr);
  }
}

TEST(Monte_Karlo_MPI, TEST_const_func_intsphere) {
  int rank;
  MPI_Comm_rank(MPI_COMM_WORLD, &rank);

  const double threr = 5.0;

  std::vector<double> s_point = {-2, -2, -2};

  double integration_result = MonteCarloParallel(
      s_point, 4, function_const, intsphere1, 3, 10000);
  if (rank == 0) {
    ASSERT_NEAR(3.14 * 4 * 8 / 3, integration_result, threr);
  }
}

TEST(Monte_Karlo_MPI, TEST_linear_func_int_2_parabols) {
  int rank;
  MPI_Comm_rank(MPI_COMM_WORLD, &rank);

  const double threr = 1.0;

  std::vector<double> s_point = {-1, -1};

  double integration_result = MonteCarloParallel(
      s_point, 2, func_linear, area_2_parabols, 2, 10000);

  if (rank == 0) {
    ASSERT_NEAR(0, integration_result, threr);
  }
}

TEST(Monte_Karlo_MPI, TEST_const_func_intcircle) {
  int rank;
  MPI_Comm_rank(MPI_COMM_WORLD, &rank);

  const double threr = 1.0;

  std::vector<double> s_point = {-2, -2};

  double integration_result = MonteCarloParallel(
      s_point, 4, function_const, intcircle, 2, 10000);
  if (rank == 0) {
    ASSERT_NEAR(3.14 * 4, integration_result, threr);
  }
}
TEST(Multidimensional_Monte_Karlo_MPI, Time_Test) {
  int rank;
  MPI_Comm_rank(MPI_COMM_WORLD, &rank);

  const double threr = 0.5;
  const int point_count = 1000000;

  std::vector<double> s_point = {-2, -2, -2, -2};
  double a1, b1, a2, b2;
  double resSequaintal;

  if (rank == 0) {
    a2 = MPI_Wtime();
    resSequaintal = MonteCarloSequential(
        s_point, 4, function_const, intsphere2, 4, point_count);
    b2 = MPI_Wtime();
  }

  if (rank == 0) a1 = MPI_Wtime();
  double integration_result = MonteCarloParallel(
      s_point, 4, function_const, intsphere2, 4, point_count);
  if (rank == 0) b1 = MPI_Wtime();

  if (rank == 0) {
    std::cout << "Sequential time " << b2 - a2 << std::endl;
    std::cout << "Parralel time " << b1 - a1;
    
  }
  if (rank == 0) {
    ASSERT_NEAR(3.14 * 3.14 * 16 / 2, integration_result, threr);
  }
}


int main(int argc, char** argv) {
    ::testing::InitGoogleTest(&argc, argv);
    MPI_Init(&argc, &argv);

    ::testing::AddGlobalTestEnvironment(new GTestMPIListener::MPIEnvironment);
    ::testing::TestEventListeners& listeners =
        ::testing::UnitTest::GetInstance()->listeners();

    listeners.Release(listeners.default_result_printer());
    listeners.Release(listeners.default_xml_generator());

    listeners.Append(new GTestMPIListener::MPIMinimalistPrinter);
    return RUN_ALL_TESTS();
}
\end{lstlisting}

\end{document}
