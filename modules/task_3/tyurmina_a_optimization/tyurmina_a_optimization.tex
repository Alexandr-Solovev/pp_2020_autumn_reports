\documentclass[a4paper]{report}

\usepackage[T2A]{fontenc}
\usepackage[utf8]{inputenc}
\usepackage[english, russian]{babel}
\usepackage[pdftex]{hyperref}
\usepackage[14pt]{extsizes}
\usepackage{listings}
\usepackage{color}
\usepackage{geometry}
\usepackage{multirow}
\usepackage{graphicx}
\usepackage{indentfirst}

\geometry{top=2cm,bottom=3cm,left=2cm,right=1.5cm}
\setlength{\parskip}{0.5cm}
\usepackage{enumitem}
\setlist{nolistsep,itemsep=0.3cm,parsep=0pt}

\lstset{language=C++,
		basicstyle=\footnotesize,
		keywordstyle=\color{blue}\ttfamily,
		stringstyle=\color{red}\ttfamily,
		commentstyle=\color{green}\ttfamily,
		morecomment=[l][\color{magenta}]{\#}, 
		tabsize=4,
		breaklines=true,
  		breakatwhitespace=true,
  		title=\lstname,       
}

\addto\captionsrussian{\renewcommand \contentsname {\LARGE Содержание}}

\begin{document}
\begin{titlepage}
\begin{center}
МИНИСТЕРСТВО НАУКИ И ВЫСШЕГО ОБРАЗОВАНИЯ РОССИЙСКОЙ ФЕДЕРАЦИИ\\
Федеральное государственное автономное образовательное учреждение высшего образования \\
\textbf{"Национальный исследовательский Нижегородский государственный университет им. Н.И. Лобачевского" (ННГУ)}
\end{center}

\begin{center}
\textbf{Институт информационных технологий, математики и механики}
\end{center}
\vspace{4em}

\begin{center}
Отчет по лабораторной работе \\
\vspace{1em}
\textbf{\Large Многошаговая схема решения двумерных задач глобальной оптимизации. Распараллеливание путем разделения области поиска.} \\
\end{center}

\vspace{4em}

\newbox{\lbox}
\savebox{\lbox}{\hbox{text}}
\newlength{\maxl}
\setlength{\maxl}{\wd\lbox}
\hfill\parbox{7cm}{
\textbf{Выполнил:} \\ 
студент группы 381806-3 \\
Тюрмина А.Н. \\
\\
\textbf{Проверил:} \\ 
доцент кафедры МОСТ, \\ 
кандидат технических наук \\ 
Сысоев А. В.
}
\vspace{\fill}

\begin{center} Нижний Новгород \\ 2020 \end{center}
\end{titlepage}

\setcounter{page}{2}
% содержание
\tableofcontents{}
\clearpage

\begin{center}
\section*{Введение}
\end{center}
\addcontentsline{toc}{section}{Введение}
\par Оптимизация -- в математике, информатике и иследовании операций задача нахождения экстремума (минимума или максимума) целевой функции в некоторой области конечномерного векторного пространства, ограниченной набором линейных или нелинейных равенств (неравенств). В анализе данных глобальная оптимизация - это раздел прикладной математики и численного анализа, в большинстве случаев решается задача минимизации, поскольку максимизация эквивалентна поиску обратного функционала для задачи минимизации.
\par Глобальная оптимизация намного сложнее, чем локальная, поскольку аналитические методы неприменимы, а численные в большинстве случаев приводят к очень сложным решениям.
\par Целью лабораторной работы является реализация параллельной многошаговой схемы решения двумерных задач глобальной оптимизации с распараллеливанием путем разделения области поиска. 
\newpage

\begin{center}
\section*{Постановка задач}
\end{center}
\addcontentsline{toc}{section}{Постановка задач}
В данной лабораторной работе необходимо:
\begin{enumerate}
\item Реализовать последовательный алгоритм метода решения задачи глобальной оптимизации.
\item Реализовать параллельный алгоритм метода решения задачи глобальной оптимизации.
\item Вспомогательные функции для эффективности.
\item Сделать вывод на основе полученных результатов.
\end{enumerate}
\newpage

\begin{center}
\section*{Описание алгоритма}
\end{center}
\addcontentsline{toc}{section}{Описание алгоритма}
Воспользуемся методом Стронгина. Пусть на отрезке $[a,b]$ определена функция f(y).
Алгоритм:
\begin{enumerate}
\item y(0)=a, y(1)=b, n=1
\item Вычисляем m, оценку константы Липшица 
$$M = max\left|\frac{z_s - z_{s-1}}{y_s - y_{s-1}}\right|$$
Тогда m принимает два значения 
$$m = 1, M = 0$$
$$m = r*M, m > 0$$, \\
где параметр r > 1
\item Для всех отрезков $y_s - y_{s-1}$, где $1 \leq s \leq n$
\par вычисляем характеристику:
\begin{equation}
    R(s) = m*(y_s - y_{s-1}) + \frac{(z_s - z_{s-1})^2}{m*(y_s - y_{s-1})} -2*(z_s + z_{s-1})
\end{equation}
и находим максимум R(s), $$1 \leq s \leq n$$
\item Находим новую точку разбиения на интервале максимальной характеристики $[y_{s-1}, y_s]$
$\gamma = s$, $n = n + 1$,
\begin{equation}
    y_n^* = \frac{y_\gamma - y_{\gamma - 1}}{2} + \frac{z_\gamma - z_{\gamma - 1}}{2*m}
\end{equation}
\item Проверяем критерий останова (останов по точности) $y_\gamma - y_{\gamma - 1} < \epsilon$, если выполняется, то алгоритм завершается. Результат работы алгоритма $min = f(y_\gamma)$, иначе упорядочиваем массив $y(i)$ по возрастанию и переходим на шаг 2.
\end{enumerate}
\newpage

\begin{center}
\section*{Схема распараллеливания}
\end{center}
\addcontentsline{toc}{section}{Схема распараллеливания}
\par В данной лабораторной работе применяется распараллеливание по области поиска. Пусть
количество процессов равно n. В начале каждый процесс получает определенную область
размера step = (b – a)/n, принимает блок данных от a1 = a + step * procrank до b1 = a1 + step.
\par В каждом из интервалов параллельно вычисляется оптимальное значение (в этот процесс входит
и вычисление характеристики инервала).
\par Затем 0-ой процесс сравнивает полученные процессами результаты и получает наиболее
оптимальный.
\newpage

\begin{center}
\section*{Описание программной реализации}
\end{center}
\addcontentsline{toc}{section}{Описание программной реализации}
\par Программа состоит из следующих функций:\\
Функция, которая получает значение функции в точке x:
\begin{lstlisting}
double GlobalOpt::get_func_val(double x)
\end{lstlisting}
Функция, которая вычисляет параметр для константы Липшица:
\begin{lstlisting}
double GlobalOpt::get_M(int i, const std::vector<double>& x)
\end{lstlisting}
Функция, которая вычисляет константу Липшица:
\begin{lstlisting}
double GlobalOpt::get_m(double r, double M)
\end{lstlisting}
Функция, которая присваивает значения:
\begin{lstlisting}
GlobalOpt::GlobalOpt(double _a1, double _b1, std::function<double(double*)> _test_func,
    double _eps)
\end{lstlisting}
Функция, которая вычисляет характеристику R(s):
\begin{lstlisting}
double GlobalOpt::get_R(int i, double m, const std::vector<double>& x)
\end{lstlisting}
Функция, которая получает новое значение x из заданного интервала:
\begin{lstlisting}
double GlobalOpt::get_new_x(int t, double m, const std::vector<double>& x)
\end{lstlisting}
Функция, которая решает задачу глобальной оптимизации последовательно:
\begin{lstlisting}
double GlobalOpt::GlobalSearchSeq(int iterationsTops)
\end{lstlisting}
Функция, которая решает задачу глобальной оптимизации параллельно:
\begin{lstlisting}
double GlobalOpt::GlobalSearchPar(int iterationsTops)
\end{lstlisting}
\newpage

\begin{center}
\section*{Подтверждение корректности}
\end{center}
\addcontentsline{toc}{section}{Подтверждение корректности}
\par Для подтверждения корректности в программе реализована система тестов, разработанная с помощью использования Google C++ Testing Framework.
\par Тесты проверяют корректную работу последовательного и параллельного алгоритмов решения задач глобальной оптимизации заданных функции и сравниваются с известными точками минимума этих функций.
\begin{itemize}
\item\textit{TEST0\_seq\_min} – проверка работы последовательной реализации;
\item\textit{TEST1\_par\_min} – проверка работы параллельной реализации;
\item\textit{TEST2\_parallel\_min\_for\_f} – проверка работы параллельной реализации;
\item\textit{TEST3\_a\_greater\_then\_b\_par} – проверка работы алгоритма параллельной реализации в случае неправильного интервала;
\item\textit{TEST4\_a\_greater\_then\_b\_seq} - проверка работы последовательной реализации в случае неправильного интервала;
\item\textit{TEST5\_timetest} - проверка работы времени;
\end{itemize}
\par Успешное прохождение тестов доказывает корректность выполнения алгоритма.
\newpage

\begin{center}
\section*{Результаты эксперементов}
\end{center}
\addcontentsline{toc}{section}{Результаты эксперементов}
\par Эксперименты проводились на оборудовании со следующей аппаратной конфигурацией:
\begin{enumerate}
\item Процессор: Intel(R) Core(TM) i5-7200U CPU @ 2.50GHz 2701МГц, ядер: 2, логических процессоров: 4
\item Оперативная память: 8 Gb,
\item Операционная система: Windows 10;
\item Среда разработки: Visual Studio 2017.
\end{enumerate}
\par Функции, с которыми проводим эксперименты 
\par $f_1 = \cos(x - 1) * \cos(x * 8 + 1) + 31$, 
\par $f_2 = x + 19 * x$
\par Результаты экспериментов представлены в Таблице 1.
\begin{table}[!h]
\caption{Результаты экспериментов}
\centering
\begin{tabular}{|p{3cm}|p{4cm}|p{4cm}|p{3cm}|}
\hline
Количество процессов & Время работы послед. алгоритма,сек & Время работы паралл. алгоритма,сек & Ускорение  \\\hline
1  & 0.0023 & 0.0022 & 1.04  \\\hline
2  & 0.0032 & 0.0023 & 1.391  \\\hline
3  & 0.0033 & 0.0023 & 1.435  \\\hline
4  & 0.0034 & 0.002 & 1.7  \\\hline
5  & 0.0039 & 0.0021 & 1.858  \\
\hline
\end{tabular}
\end{table}
\par На основе полученных данных, можно сделать вывод, что параллельный случай работает быстрее, чем последовательный. Ускорение возрастает с количеством процессов.
\newpage

\begin{center}
\section*{Заключение}
\end{center}
\addcontentsline{toc}{section}{Заключение}
\par В результате лабораторной работы была многошаговая схема решения двумерных задач глобальной оптимизации распараллеленная путем разделения области поиска.
\par Реализация параллельной версии алгоритма была успешно достигнута, о чем говорят результаты экспериментов, проведенных в ходе работы. Они показывают, что параллельный случай работает быстрее, чем последовательный. 
\par Для подтверждения корректности работы программы были разработаны и доведены до успешного выполнения тесты, созданные с использованием Google C++ Testing Framework. 
\newpage

\addcontentsline{toc}{section}{Литература}
\begin{thebibliography}{}
\bibitem{1} Городецкий С.Ю. Лекции по нелинейному математическому программированию – 2020.
\bibitem{2} Метод Стронгина и метод Пиявского [Электронный ресурс]:Центр суперкомпьютерных технологий, НГУ им. Лобачевского // URL: \url{http://www.hpcc.unn.ru/?dir=893}.
\bibitem{3} Оптимизация [Электронный ресурс]: Википедия // URL: \url{https://ru.wikipedia.org/wiki} 
\bibitem{4}  В. П. Гергель, В. А. Гришин, А. В. Гергель Многомерная многоэкстремальная оптимизация на основе адаптивной многошаговой редукции размерности - 2010
\end{thebibliography}{}
\newpage

\begin{center}
\section*{Приложение}
\end{center}
\addcontentsline{toc}{section}{Приложение}
\begin{lstlisting}
//  Copyright 2020 Tyurmina Alexandra
#ifndef MODULES_TASK_3_TYURMINA_A_OPTIMIZATION_MINIMIZE_H_
#define MODULES_TASK_3_TYURMINA_A_OPTIMIZATION_MINIMIZE_H_

#include <functional>
#include <vector>

class GlobalOpt {
    double a1; double b1;
    double epsilon;
    std::function<double(double*)> test_func;

 public:
    // XY search
    double GlobalSearchSeq(int iterationsTops);
    double GlobalSearchPar(int iterationsTops);

    GlobalOpt(double a1, double b1, std::function<double(double*)> test_func, double eps);


 protected:
    double get_func_val(double x);
    double get_m(double r, double M);
    double get_M(int i, const std::vector<double>& x);
    double get_R(int i, double m, const std::vector<double>& x);
    double get_new_x(int t, double m, const std::vector<double>& x);
};

#endif  // MODULES_TASK_3_TYURMINA_A_OPTIMIZATION_MINIMIZE_H_
\end{lstlisting}
\begin{lstlisting}
  // Copyright 2020 Tyurmina Alexandra

#include <mpi.h>
#include <utility>
#include <list>
#include <algorithm>
#include <iostream>
#include <vector>
#include <cmath>
#include "../../../modules/task_3/tyurmina_a_optimization/minimize.h"

double GlobalOpt::get_func_val(double x) {
    return test_func(&x);
}

double GlobalOpt::get_M(int i, const std::vector<double>& x) {
    double diff1 = get_func_val(x[i]) - get_func_val(x[i - 1]);
    double diff2 = x[i] - x[i - 1];

    return std::abs(diff1) / diff2;
}

double GlobalOpt::get_m(double r, double M) {
    if (M == 0) {
        return 1;
    } else {
        return  M * r;
    }
}

GlobalOpt::GlobalOpt(double _a1, double _b1, std::function<double(double*)> _test_func,
    double _eps) {
    a1 = _a1; b1 = _b1;
    test_func = _test_func;
    epsilon = _eps;
}

double GlobalOpt::get_R(int i, double m, const std::vector<double>& x) {
    double diff = get_func_val(x[i]) - get_func_val(x[i - 1]);
    double summ = get_func_val(x[i]) + get_func_val(x[i - 1]);
    double koef = m * (x[i] - x[i - 1]);
    double val = std::pow(diff, 2) / koef - 2 * summ + koef;

    return val;
}

double GlobalOpt::get_new_x(int t, double m, const std::vector<double>& x) {
    double diff = get_func_val(x[t]) - get_func_val(x[t - 1]);
    double summ = x[t] + x[t - 1];
    double val = (summ - diff / m) / 2;

    return val;
}

double GlobalOpt::GlobalSearchSeq(int iterationsTops) {
    std::vector<double> x_val(iterationsTops + 1, 0);
    int iterationsNum = 1;
    int t = 0;

    double r = 2;
    double R = 0;
    double temp = 0;
    double M = get_M(1, x_val);
    double m = get_m(r, M);

    x_val[0] = a1;
    x_val[1] = b1;
    x_val[2] = get_new_x(1, m, x_val);

    ++iterationsNum;

    if (a1 > b1) {
        throw "b is a right border (must be b > a)";
    }
    while (iterationsNum < iterationsTops) {
        sort(x_val.begin(), x_val.begin() + iterationsNum + 1);
        M = get_M(1, x_val);

        for (int i = 2; i <= iterationsNum; ++i) {
            M = std::max(M, get_M(i, x_val));
        }

        m = get_m(r, M);
        R = get_R(1, m, x_val);
        t = 1;

        for (int i = 2; i <= iterationsNum; ++i) {
            temp = get_R(i, m, x_val);
            if (R < temp) {
                R = temp;
                t = i;
            }
        }

        x_val[iterationsNum + 1] = get_new_x(t, m, x_val);
        ++iterationsNum;

        if (x_val[t] - x_val[t - 1] <= epsilon) {
            break;
        }
    }
    return x_val[t];
}

double GlobalOpt::GlobalSearchPar(int iterationsTops) {
    int prNum, prRank;

    MPI_Comm_size(MPI_COMM_WORLD, &prNum);
    MPI_Comm_rank(MPI_COMM_WORLD, &prRank);

    std::vector<double> result(prNum);

    if (a1 > b1) {
        throw "b is a right border (must be b > a)";
    }
    if (prNum > 1) {
        double step = (b1 - a1) / prNum;
        double a = a1 + step * prRank;
        double b = a + step;

        if (a != b) {
            double temp = 0;
            GlobalOpt opt(a, b, test_func, epsilon);
            double* local = new double[prNum];
            *local = opt.GlobalSearchSeq(iterationsTops);
            MPI_Gather(&local[0], 1, MPI_DOUBLE, &result[0], 1, MPI_DOUBLE, 0, MPI_COMM_WORLD);

            double root = get_func_val(result[0]);
            if (prRank == 0) {
                for (int i = 1; i < prNum; ++i) {
                    temp = get_func_val(result[i]);
                    if (temp < root) {
                        root = temp;
                        std::swap(result[i], result[0]);
                    }
                }
            }
        }
    } else {
        return GlobalSearchSeq(iterationsTops);
    }
    return result[0];
}

\end{lstlisting}
\begin{lstlisting}
//  Copyright 2020 Tyurmina Alexandra
#include <gtest-mpi-listener.hpp>
#include <gtest/gtest.h>
#include <iostream>
#include <cmath>
#include "./minimize.h"

double f_test1(double* x) {
    return cos((*x) - 1) * cos((*x) * 8 + 1) + 31;
}

double f_test2(double* x) {
    return (*x) + 19 * (*x);
}

TEST(global_GlobalOpt, TEST0_seq_min) {
    GlobalOpt opt(0, 2.2, f_test1, 1e-5);

    double result = opt.GlobalSearchSeq(800);
    double correct_result = 1.0529;
    ASSERT_NEAR(result, correct_result, 1e-3);
}

TEST(global_GlobalOpt, TEST1_par_min) {
    int rank;
    MPI_Comm_rank(MPI_COMM_WORLD, &rank);
    GlobalOpt opt(0.3, 2.2, f_test2, 1e-5);

    double result = opt.GlobalSearchPar(800);
    double correct_result = 0.3;
    if (rank == 0) {
        ASSERT_NEAR(result, correct_result, 1e-4);
    }
}

TEST(global_GlobalOpt, TEST2_parallel_min_for_f) {
    int rank;
    MPI_Comm_rank(MPI_COMM_WORLD, &rank);
    GlobalOpt opt(0, 1, f_test2, 1e-5);

    double result = opt.GlobalSearchPar(800);
    double correct_result = 0;
    if (rank == 0) {
        ASSERT_NEAR(result, correct_result, 1e-5);
    }
}

TEST(global_GlobalOpt, TEST3_a_greater_then_b_par) {
    GlobalOpt opt(2.2, 0, f_test1, 1e-5);
    EXPECT_ANY_THROW(opt.GlobalSearchPar(800));
}

TEST(global_GlobalOpt, TEST4_a_greater_then_b_seq) {
    GlobalOpt opt(2.2, 0, f_test1, 1e-5);
    EXPECT_ANY_THROW(opt.GlobalSearchSeq(800));
}

TEST(global_GlobalOpt, TEST5_timetest) {
    int rank;
    MPI_Comm_rank(MPI_COMM_WORLD, &rank);
    GlobalOpt opt(0, 1, f_test1, 1e-5);

    double start1 = MPI_Wtime();
    double resp = opt.GlobalSearchPar(800);
    double end1 = MPI_Wtime();
    if (rank == 0) {
        std::cout << "Parall: " << end1 - start1 << std::endl;
        double start2 = MPI_Wtime();
        double ress = opt.GlobalSearchSeq(800);
        double end2 = MPI_Wtime();
        std::cout << "Seq: " << end2 - start2 << std::endl;
        ASSERT_NEAR(resp, ress, 1e-5);
    }
}

int main(int argc, char** argv) {
    ::testing::InitGoogleTest(&argc, argv);
    MPI_Init(&argc, &argv);

    ::testing::AddGlobalTestEnvironment(new GTestMPIListener::MPIEnvironment);
    ::testing::TestEventListeners& listeners =
        ::testing::UnitTest::GetInstance()->listeners();

    listeners.Release(listeners.default_result_printer());
    listeners.Release(listeners.default_xml_generator());

    listeners.Append(new GTestMPIListener::MPIMinimalistPrinter);
    return RUN_ALL_TESTS();
}
\end{lstlisting}
\end{document}