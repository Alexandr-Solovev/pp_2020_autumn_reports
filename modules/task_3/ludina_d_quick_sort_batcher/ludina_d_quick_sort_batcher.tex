\documentclass[a4paper]{report}

\usepackage[T2A]{fontenc}
\usepackage[utf8]{inputenc}
\usepackage[english, russian]{babel}
\usepackage[pdftex]{hyperref}
\usepackage[14pt]{extsizes}
\usepackage{listings}
\usepackage{color}
\usepackage{geometry}
\usepackage{multirow}
\usepackage{graphicx}
\usepackage{indentfirst}

\geometry{top=2cm,bottom=3cm,left=2cm,right=1.5cm}
\setlength{\parskip}{0.5cm}
\usepackage{enumitem}
\setlist{nolistsep,itemsep=0.3cm,parsep=0pt}

\lstset{language=C++,
		basicstyle=\footnotesize,
		keywordstyle=\color{blue}\ttfamily,
		stringstyle=\color{red}\ttfamily,
		commentstyle=\color{green}\ttfamily,
		morecomment=[l][\color{magenta}]{\#}, 
		tabsize=4,
		breaklines=true,
  		breakatwhitespace=true,
  		title=\lstname,       
}

\addto\captionsrussian{\renewcommand \contentsname {\LARGE Содержание}}

\begin{document}
\begin{titlepage}
\begin{center}
МИНИСТЕРСТВО НАУКИ И ВЫСШЕГО ОБРАЗОВАНИЯ РОССИЙСКОЙ ФЕДЕРАЦИИ\\
Федеральное государственное автономное образовательное учреждение высшего образования \\
\textbf{"Национальный исследовательский Нижегородский государственный университет им. Н.И. Лобачевского" (ННГУ)}
\end{center}

\begin{center}
\textbf{Институт информационных технологий, математики и механики}
\end{center}
\vspace{4em}

\begin{center}
Отчет по лабораторной работе \\
\vspace{1em}
\textbf{\Large Быстрая сортировка с четно-нечетным слиянием Бэтчера} \\
\end{center}

\vspace{4em}

\newbox{\lbox}
\savebox{\lbox}{\hbox{text}}
\newlength{\maxl}
\setlength{\maxl}{\wd\lbox}
\hfill\parbox{7cm}{
\textbf{Выполнил:} \\ 
студент группы 381806-3 \\
Лудина Д. А. \\
\\
\textbf{Проверил:} \\ 
доцент кафедры МОСТ, \\ 
кандидат технических наук \\ 
Сысоев А. В.
}
\vspace{\fill}

\begin{center} Нижний Новгород \\ 2020 \end{center}
\end{titlepage}

\setcounter{page}{2}
% содержание
\tableofcontents{}
\clearpage

\begin{center}
\section*{Введение}
\end{center}
\addcontentsline{toc}{section}{Введение}
\par Под сортировкой понимают процесс размещения элементов заданного множества объектов в некотором определенном порядке. Она является одной из базовых операций обработки данных.
Когда элементы отсортированы, их проще найти, производить с ними различные операции. \\
Один из самых известных и широко используемых алгоритмов сортировки - быстрая сортировка.
\par При больших объемах данных даже наиболее оптимальные методы
сортировки работают долго. Поэтому исходные данные следует разбивать
на группы меньшего размера и производить упорядочивание параллельно. Это значительно
ускоряет сортировку, но могут возникнуть проблемы корректного слияния полученных
фрагментов. Использование слияния Бэтчера с построением сети сортировки может решить эту
проблему. 
\par Поэтому целью данной работы является реализация параллельной быстрой сортировки с четно - нечетным слиянием Бэтчера и сравнение ее с последовательной быстрой сортировкой.
\newpage

\begin{center}
\section*{Постановка задач}
\end{center}
\addcontentsline{toc}{section}{Постановка задач}
В данной лабораторной работе необходимо:
\begin{enumerate}
\item Реализовать последовательный алгоритм быстрой сортировки.
\item Реализовать параллельный алгоритм быстрой сортировки с четно - нечетным слиянием Бэтчера с использованием MPI.
\item Провести вычислительные эксперементы и сравнить время работы двух алгоритмов.
\item Сделать вывод на основе полученных результатов.
\end{enumerate}
\newpage

\begin{center}
\section*{Описание алгоритма}
\end{center}
\addcontentsline{toc}{section}{Описание алгоритма}
Алгоритм быстрой сортировки:
\begin{enumerate}
\item Выбираем ведущий элемент (середина входного массива).
\item Переставляем элементы массива следующим образом: меньше ведущего в левую чать, а те, которые больше в правую. 
\item Рекурсивно выполняем фунцию для левой и правой частей. 
\end{enumerate}
Алгоритм построения сети сортировки для слияния Батчера:
\begin{enumerate}
\item Если длинна входного массива меньше или равна 1, то мы завершаем работу.
\item Разбиваем входной массив на два массива up и down.
\item Рекурсивно выполняем функцию для обоих массивов.
\item Для полученных массивов выполняем следующее:
\begin{itemize}
\item Если сумма размеров массивов равна 1, то мы завершаем работу.
\item Если сумма размеров массивов равна 2, то добавляем пару элементов в сеть
сортировки.
\item Разбиваем каждый массив на два: в одном элементы с четными индексами, а в другом с нечетными.
\item Рекурсивно выполняем процедуру для пар полученных массивов, отдельно с четными и нечетными индексами.
\item Добавляем полученные массивы в один.
\item Добавляем в сеть сортировки пары элементов с индексами (1, 2), (3, 4) и т. д.
\end{itemize}
\end{enumerate}
\newpage

\begin{center}
\section*{Схема распараллеливания}
\end{center}
\addcontentsline{toc}{section}{Схема распараллеливания}
\par Для удобства будем считать, что сеть сортировки уже построена. Для распараллеливания алгоритма быстрой сортировки необходимо разбить исходный массив на подмассивы, количесво которых равно количеству параллельно работающих процессов.\\
Для корректной работы алгорита число элементов массива должно быть кратно числу процессов, если это не так, то в массив добавляются фиктивные элементы, заведомо больше всех остальных элементов. После разбиения надо сделать рассылку подмассивов процессам, затем происходит быстрая сортировка каждого подмассива в своем процессе.\\
\par Следующим этапом является само слияние Бэтчера. В цикле по массиву, хранящему сеть сортировки, каждый элемент которой представляет собой пару номеров процессов или компаратор, процессы с номерами, входящими в один компаратор, отправляют друг другу свои данные. Потом каждый процесс выполняет упорядоченное слияние полученных данных со своими, причем первый процесс в компораторе сохраняет в своем подмассиве первую половину получившегося массива, а второй процесс – вторую.\\
Затем осуществляется сбор подмассивов от всех процессов в корневой массив и удаление фиктивных переменных. В итоге получется отсортированный массив.
\newpage

\begin{center}
\section*{Описание программной реализации}
\end{center}
\addcontentsline{toc}{section}{Описание программной реализации}
\par Последовательный алгоритм быстрой сортировки представлен двумя функциями:\\
Основной функцией 
\begin{lstlisting}
void quickSort(std::vector<int>* vec, int first, int last);
\end{lstlisting}
и вспомогательной
\begin{lstlisting}
int part(std::vector<int>* vec, int first, int last);
\end{lstlisting}
В которые в качетве входных параметров передаются ссылка на вектор и индексы первого и последнего элементов сортируемого фрагмента.
\par Построение сети сортировка для четно - нечетного слияния Бэтчера состоит из двух функций:
\begin{lstlisting}
void network(std::vector<int> procs);
\end{lstlisting}
На вход передается вектор, хранящий в себе номера процессов.
\begin{lstlisting}
void merge(std::vector<int> vec_up, std::vector<int> vec_down);
\end{lstlisting}
В качестве входных параметров передаются вектора, в которых хранятся номера процессов, выбранные четным и нечетным правилом.
\par Параллельный алгоритм быстрой сортировки представлен функцией:
\begin{lstlisting}
void quickSortBatcher(std::vector<int>* vec);
\end{lstlisting}
В качетве входного параметра передается ссылка на вектор, который необходимо отсортировать.
\par Также в программе реализована вспомогательная функция, заполняющая вектор случайными целыми значениями от 1 до 99. На вход принимает размер вектора
\begin{lstlisting}
std::vector<int> createRandomVector(int size_v);
\end{lstlisting}
\newpage

\begin{center}
\section*{Подтверждение корректности}
\end{center}
\addcontentsline{toc}{section}{Подтверждение корректности}
\par Для подтверждения корректности в программе реализована система тестов, разработанная с помощью использования Google C++ Testing Framework.
\begin{itemize}
\item\textit{Test\_Quick\_Sort\_1} – проверка работы последовательной быстрой сортировки на массиве с большим размером;
\item\textit{Test\_Quick\_Sort\_2} – проверка работы последовательной быстрой сортировки на массиве с маленьким размером;
\item\textit{Test\_Batcher\_Sort\_1} – проверка работы алгоритма параллельной быстрой сортировки с четно - нечетным слиянием Бэтчера на малом массиве;
\item\textit{Test\_Batcher\_Sort\_2} – проверка работы алгоритма параллельной быстрой сортировки с четно - нечетным слиянием Бэтчера на большом массиве;
\item\textit{TimeTest} - проверка эффективности параллельной схемы. Сравнение времени последовательного и параллельного алгоритмов;
\end{itemize}
\par Успешное прохождение данных тестов доказывает корректность работы всей программы.
\newpage

\begin{center}
\section*{Результаты эксперементов}
\end{center}
\addcontentsline{toc}{section}{Результаты эксперементов}
\par Вычислительные эксперименты для оценки эффективности параллельной быстрой сортировки с четно-нечетным слиянием Бэтчера проводились на ПК со следующей аппаратной конфигурацией:
\begin{enumerate}
\item Процессор: Intel(R) Core(TM) i5-8250U CPU @ 1.60GHz 1.80GHz, ядер: 4, логических процессоров: 8;
\item Оперативная память: 6 Gb, 2133 MHz;
\item Операционная система: Windows 10 Домашняя;
\item Среда разработки: Visual Studio 2017.
\end{enumerate}
\par Для проведения тестов производилась сортировка векторов размером 10000000 на числе процессов от 1 до 10.
В рамках эксперимента было вычислено время работы параллельного и последовательного алгоритмов, усреднение по трем экспериментам.\\
Результаты экспериментов представлены в Таблице 1.
\begin{table}[!h]
\caption{Результаты вычислительных экспериментов}
\centering
\begin{tabular}{|p{3cm}|p{4cm}|p{4cm}|p{3cm}|}
\hline
Количество процессов & Последовательный алгоритм,сек & Параллельный алгоритм,сек & Ускорение  \\\hline
1  & 0.474 & 0.332 & 1.428  \\\hline
2  & 0.469 & 0.240 & 1.954  \\\hline
3  & 0.476 & 0.281 & 1.694  \\\hline
4  & 0.475 & 0.262 & 1.813  \\\hline
5  & 0.490 & 0.330 & 1.485  \\\hline
6  & 0.476 & 0.356 & 1.337  \\\hline
7  & 0.477 & 0.381 & 1.252  \\\hline
8  & 0.482 & 0.374 & 1.289  \\\hline
9  & 0.477 & 0.459 & 1.039  \\\hline
10 & 0.474 & 0.480 & 0.987  \\
\hline
\end{tabular}
\end{table}
\par На основе полученных данных, можно сделать вывод, что параллельный случай работает быстрее, чем последовательный. Однако, разница по времени между алгоритмами не сильно высока, а при 10 процессах последовательный случай по времени меньше параллельного.\\
Это объясняется следующим:\\
Во - первых, переключениями между процессами, которые предполагает алгоритма параллельной быстрой сортировки с четно - нечетным слиянием Бэтчера.\\
Во - вторых, использованием коллективных операций для обмена данными между ними.
\par Кроме того, можно сделать вывод, что при данной аппаратной конфигурации наиболее эффективным будет использование 2 - 4 процессов для параллельной быстрой сортировки с четно - нечетным слиянием Бэтчера.
\newpage

\begin{center}
\section*{Заключение}
\end{center}
\addcontentsline{toc}{section}{Заключение}
\par В ходе данной лабораторной работы мною были реализованы два алгоритма сортировки: классическая быстрая сортировка и параллельная быстрая сортировка с четно - нечетным слиянием Бэтчера. Кроме того, были разработаны и доведены до успешного выполнения тесты, созданные для данного программного проекта с использованием Google C++ Testing Framework и необходимые для подтверждения корректности работы программы.
\par Основной целью работы была реализация параллельной версии сортировки, которая должна работать быстрее по времени, чем последовательная.
Вычислительные эксперименты нам это показали.
\par Быстрая сортировка в параллельном алгоритме работает на меньшем объеме данных,
что сказывается на времени работы. Этот факт делает ее более предпочтительной для использования в реальных вычислительных задачах.
\newpage

\addcontentsline{toc}{section}{Литература}
\begin{thebibliography}{}
\bibitem{1} Гергель В. П. Теория и практика параллельных вычислений. – 2007.
\bibitem{2} Гергель В. П., Стронгин Р. Г. Основы параллельных вычислений для многопроцессорных вычислительных систем. – 2003.
\bibitem{3} Якобовский М. В. Параллельные алгоритмы сортировки больших объемов
данных // Фундаментальные физико-математические проблемы и
моделирование технико-технологических систем: Сб. науч. тр.: Янус-К, 2004,
с. 235 – 249.
\bibitem{4} Сеть обменной сортировки со слиянием Бэтчера [Электронный ресурс]: Хабр,
сообщество IT-специалистов // URL: \url{https://habr.com/ru/post/275889/}
\end{thebibliography}{}
\newpage

\begin{center}
\section*{Приложение}
\end{center}
\addcontentsline{toc}{section}{Приложение}
\begin{lstlisting}
// quick_sort_batcher.h
// Copyright 2020 Ludina Daria
#ifndef MODULES_TASK_3_LUDINA_D_QUICK_SORT_BATCHER_QUICK_SORT_BATCHER_H_
#define MODULES_TASK_3_LUDINA_D_QUICK_SORT_BATCHER_QUICK_SORT_BATCHER_H_

#include <mpi.h>
#include <random>
#include <ctime>
#include <vector>
#include <utility>
#include <algorithm>

std::vector<int> createRandomVector(int size_v);
int part(std::vector<int>* vec, int first, int last);
void quickSort(std::vector<int>* vec, int first, int last);
void merge(std::vector<int> vec_up, std::vector<int> vec_down);
void network(std::vector<int> procs);
void quickSortBatcher(std::vector<int>* vec);

#endif  // MODULES_TASK_3_LUDINA_D_QUICK_SORT_BATCHER_QUICK_SORT_BATCHER_H_
\end{lstlisting}

\begin{lstlisting}
// quick_sort_batcher.cpp
// Copyright 2020 Ludina Daria
#include "../../../modules/task_3/ludina_d_quick_sort_batcher/quick_sort_batcher.h"
#include <vector>
#include <utility>
#include <algorithm>

std::vector<std::pair<int, int>> comparators;

std::vector<int> createRandomVector(int size_v) {
  std::mt19937 random;
  random.seed(static_cast<unsigned int>(time(0)));
  std::vector<int> vec(size_v);
  for (int i = 0; i < size_v; i++) {
    vec[i] = random() % 100;
  }
  return vec;
}

int part(std::vector<int>* vec, int first, int last) {
  int pivot = (*vec)[(first + last) / 2];
  int f = first;
  int  l = last;
  int tmp;
  while (f <= l) {
    while ((*vec)[f] < pivot) f++;
    while ((*vec)[l] > pivot) l--;
    if (f >= l)
      break;
    tmp = (*vec)[f];
    (*vec)[f] = (*vec)[l];
    (*vec)[l] = tmp;
    f++;
    l--;
  }
  return l;
}

void quickSort(std::vector<int>* vec, int first, int last) {
  if (first < last) {
    int pivot = part(vec, first, last);
    quickSort(vec, first, pivot);
    quickSort(vec, pivot + 1, last);
  }
}

void merge(std::vector<int> vec_up, std::vector<int> vec_down) {
  int vec_size = vec_up.size() + vec_down.size();
  if (vec_size == 1) {
    return;
  } else if (vec_size == 2) {
    comparators.push_back(std::pair<int, int>(vec_up[0], vec_down[0]));
    return;
  }

  std::vector<int> vec_up_odd, vec_up_even;
  std::vector<int> vec_down_odd, vec_down_even;
  std::vector<int> vec_result(vec_size);
  int vec_up_size = vec_up.size();
  for (int i = 0; i < vec_up_size; i++) {
    i % 2 ? vec_up_even.push_back(vec_up[i]) : vec_up_odd.push_back(vec_up[i]);
  }
  int vec_down_size = vec_down.size();
  for (int i = 0; i < vec_down_size; i++) {
    i % 2 ? vec_down_even.push_back(vec_down[i]) : vec_down_odd.push_back(vec_down[i]);
  }

  merge(vec_up_odd, vec_down_odd);
  merge(vec_up_even, vec_down_even);

  std::copy(vec_up.begin(), vec_up.end(), vec_result.begin());
  std::copy(vec_down.begin(), vec_down.end(), vec_result.begin() + vec_up.size());
  int res_size = vec_result.size();
  for (int i = 1; i < res_size - 1; i += 2) {
    comparators.push_back(std::pair<int, int>(vec_result[i], vec_result[i + 1]));
  }
}

void network(std::vector<int> procs) {
  int p_size = procs.size();
  if (p_size <= 1) {
    return;
  }
  std::vector<int> vec_up(p_size / 2);
  std::vector<int> vec_down(p_size / 2 + p_size % 2);

  std::copy(procs.begin(), procs.begin() + vec_up.size(), vec_up.begin());
  std::copy(procs.begin() + vec_up.size(), procs.end(), vec_down.begin());
  network(vec_up);
  network(vec_down);
  merge(vec_up, vec_down);
}

void quickSortBatcher(std::vector<int>* vec) {
  int rank, size;
  MPI_Comm_rank(MPI_COMM_WORLD, &rank);
  MPI_Comm_size(MPI_COMM_WORLD, &size);
  MPI_Status status;

  int vec_size = vec->size();
  int fict = 0;
  while (vec_size % size) {
    vec->push_back(1000);
    vec_size++;
    fict++;
  }
  int delta = vec_size / size;
  std::vector<int> result(delta);
  std::vector<int> curr_vec(delta);
  std::vector<int> tmp_vec(delta);

  MPI_Scatter(&(*vec)[0], delta, MPI_INT, &result[0], delta, MPI_INT, 0, MPI_COMM_WORLD);
  std::sort(result.begin(), result.end());

  std::vector<int> procs(size);
  int s_procs = procs.size();
  for (int i = 0; i < s_procs; i++)
    procs[i] = i;
  network(procs);

  int comp_size = comparators.size();
  for (int i = 0; i < comp_size; i++) {
    std::pair<int, int> comparator = comparators[i];
    if (rank == comparator.first) {
      MPI_Send(result.data(), delta, MPI_INT, comparator.second, 0, MPI_COMM_WORLD);
      MPI_Recv(curr_vec.data(), delta, MPI_INT, comparator.second, 0, MPI_COMM_WORLD, &status);
      int resi = 0, curi = 0;
      for (int tmpi = 0; tmpi < delta; tmpi++) {
        int res = result[resi];
        int cur = curr_vec[curi];
        if (res < cur) {
          tmp_vec[tmpi] = res;
          resi++;
        } else {
          tmp_vec[tmpi] = cur;
          curi++;
        }
      }
      result = tmp_vec;
    } else if (rank == comparator.second) {
      MPI_Recv(curr_vec.data(), delta, MPI_INT, comparator.first, 0, MPI_COMM_WORLD, &status);
      MPI_Send(result.data(), delta, MPI_INT, comparator.first, 0, MPI_COMM_WORLD);
      int start = delta - 1;
      int resi = start, curi = start;
      for (int tmpi = start; tmpi >= 0; tmpi--) {
        int res = result[resi];
        int cur = curr_vec[curi];
        if (res > cur) {
          tmp_vec[tmpi] = res;
          resi--;
        } else {
          tmp_vec[tmpi] = cur;
          curi--;
        }
      }
      result = tmp_vec;
    }
  }
  MPI_Gather(&result[0], delta, MPI_INT, &(*vec)[0], delta, MPI_INT, 0, MPI_COMM_WORLD);
  if (rank == 0 && fict != 0) {
    vec->erase(vec->begin() + vec_size - fict, vec->end());
  }
  return;
}
\end{lstlisting}

\begin{lstlisting}
//main.cpp
// Copyright 2020 Ludina Daria
#include <gtest-mpi-listener.hpp>
#include <gtest/gtest.h>
#include <vector>
#include <algorithm>
#include "./quick_sort_batcher.h"

TEST(Quick_Sort_Batcher, Test_Quick_Sort_1) {
  int rank;
  MPI_Comm_rank(MPI_COMM_WORLD, &rank);
  int vec_size = 100;
  std::vector<int> vec_1(vec_size);
  std::vector<int> vec_2(vec_size);

  if (rank == 0) {
    vec_1 = createRandomVector(vec_size);
    vec_2 = vec_1;
    quickSort(&vec_1, 0, 99);
    std::sort(vec_2.begin(), vec_2.end());
    ASSERT_EQ(vec_1, vec_2);
  }
}

TEST(Quick_Sort_Batcher, Test_Quick_Sort_2) {
  int rank;
  MPI_Comm_rank(MPI_COMM_WORLD, &rank);
  int vec_size = 10;
  std::vector<int> vec_1(vec_size);
  std::vector<int> vec_2(vec_size);

  if (rank == 0) {
    vec_1 = { 8, 4, 17, 3, 11, 2, 1, 9, 7, 20 };
    vec_2 = vec_1;
    quickSort(&vec_1, 0, 9);
    std::sort(vec_2.begin(), vec_2.end());
    ASSERT_EQ(vec_1, vec_2);
  }
}

TEST(Quick_Sort_Batcher, Test_Batcher_Sort_1) {
  int rank;
  MPI_Comm_rank(MPI_COMM_WORLD, &rank);
  int vec_size = 100;
  std::vector<int> vec_1(vec_size);
  std::vector<int> vec_2(vec_size);

  if (rank == 0) {
    vec_1 = createRandomVector(vec_size);
    vec_2 = vec_1;
  }

  quickSortBatcher(&vec_1);
  if (rank == 0) {
    std::sort(vec_2.begin(), vec_2.end());
    ASSERT_EQ(vec_1, vec_2);
  }
}

TEST(Quick_Sort_Batcher, Test_Batcher_Sort_2) {
  int rank;
  MPI_Comm_rank(MPI_COMM_WORLD, &rank);
  int vec_size = 1000;
  std::vector<int> vec_1(vec_size);
  std::vector<int> vec_2(vec_size);

  if (rank == 0) {
    vec_1 = createRandomVector(vec_size);
    vec_2 = vec_1;
  }

  quickSortBatcher(&vec_1);
  if (rank == 0) {
    std::sort(vec_2.begin(), vec_2.end());
    ASSERT_EQ(vec_1, vec_2);
  }
}

TEST(Quick_Sort_Batcher, TimeTest) {
  int rank;
  MPI_Comm_rank(MPI_COMM_WORLD, &rank);
  int vecsize = 100000;
  std::vector<int> vec(vecsize);
  std::vector<int> vec2(vecsize);

  if (rank == 0) {
    vec = createRandomVector(vecsize);
    vec2 = vec;
  }
  double startTime = MPI_Wtime();
  quickSortBatcher(&vec);
  double finishTime = MPI_Wtime();

  if (rank == 0) {
    std::cout << "Parallel sort:" <<  finishTime - startTime << std::endl;
    startTime = MPI_Wtime();
    std::sort(vec2.begin(), vec2.end());
    // quickSort(&vec2, 0, 99999);
    double finishTime = MPI_Wtime();
    std::cout << "Quick sort:" << finishTime - startTime << std::endl;
    ASSERT_EQ(vec2, vec);
  }
}

int main(int argc, char** argv) {
  ::testing::InitGoogleTest(&argc, argv);
  MPI_Init(&argc, &argv);

  ::testing::AddGlobalTestEnvironment(new GTestMPIListener::MPIEnvironment);
  ::testing::TestEventListeners& listeners =
    ::testing::UnitTest::GetInstance()->listeners();

  listeners.Release(listeners.default_result_printer());
  listeners.Release(listeners.default_xml_generator());

  listeners.Append(new GTestMPIListener::MPIMinimalistPrinter);
  return RUN_ALL_TESTS();
}
\end{lstlisting}

\end{document}