\documentclass{report}

\usepackage[T2A]{fontenc}
\usepackage[utf8]{luainputenc}
\usepackage[english, russian]{babel}
\usepackage[pdftex]{hyperref}
\usepackage[14pt]{extsizes}
\usepackage{listings}
\usepackage{color}
\usepackage{geometry}
\usepackage{enumitem}
\usepackage{multirow}
\usepackage{graphicx}
\usepackage{indentfirst}
\usepackage{amsmath}

\geometry{a4paper,top=2cm,bottom=3cm,left=2.5cm,right=1.5cm}
\setlength{\parskip}{0.5cm}
\setlist{nolistsep, itemsep=0.3cm,parsep=0pt}

\lstset{language=C++,
		basicstyle=\footnotesize,
		keywordstyle=\color{blue}\ttfamily,
		stringstyle=\color{green}\ttfamily,
		commentstyle=\color{green}\ttfamily,
		morecomment=[l][\color{red}]{\#}, 
		tabsize=2,
		breaklines=true,
  		breakatwhitespace=true,
  		title=\lstname,
}

\makeatletter
\renewcommand\@biblabel[1]{#1.\hfil}
\makeatother

\begin{document}

\begin{titlepage}
\begin{center}
Министерство образования и науки Российской Федерации
\end{center}
\begin{center}
Федеральное государственное автономное образовательное учреждение высшего образования \\
Национальный исследовательский Нижегородский государственный университет им. Н.И. Лобачевского(ННГУ)
\end{center}
\begin{center}
Институт информационных технологий, математики и механики
\end{center}
\begin{center}
Направление подготовки: «Фундаментальная информатика и информационные технологии»
\end{center}

\vspace{2em}

\begin{center}
\textbf{\LargeОтчет по лабораторной работе} 
\end{center}
\begin{center}
\textbf{\Large«Вычисление многомерных интегралов \\
с использованием многошаговой схемы \\
(метод прямоугольников)»} \\
\end{center}

\vspace{4em}

\newbox{\lbox}
\savebox{\lbox}{\hbox{text}}
\newlength{\maxl}
\setlength{\maxl}{\wd\lbox}
\hfill\parbox{7cm}{
\textbf{Выполнил:} \\
студент группы 381806-3 \\
Булычев В. Д.\\
\\
\textbf{Проверил:} \\
доцент кафедры МОСТ, \\
кандидат технических наук \\
Сысоев А. В.
}
\vspace{\fill}
\begin{center} Нижний Новгород \\ 2020 \end{center}
\end{titlepage}

\setcounter{page}{2}

\tableofcontents
\newpage

\section*{Введение}
\addcontentsline{toc}{section}{Введение}
Численное интегрирование — приближенное вычисление значения определенного интеграла. Допустим, нам дана функция f(x), определенная на отрезке или многомерном кубе, и возможность получать ее численное значение в любой из точек области определения за фиксированное время. Задача — вычислить определенный интеграл данной функции по заданной области и дать оценку погрешности вычисления.
\par Большинство методов численного интегрирования, объединенны общим принципом: область интегрирования разбивается по каждой из осей на равное количество частей. В каждой из полученных маленьких областей интеграл приближается простой функцией (например, линейной), значение которой вычисляется явно (путем вычисления подынтегрального выражения в одной или нескольких точках). Ввиду линейности оператора интегрирования по областям полученные значения суммируются и представляют собой результат интегрирования.
\par К данному типу относятся одномерные метод прямоугольников, метод трапеций, метод парабол (метод Симпсона). Перечисленные методы обобщается и для многомерного случая. Обобщенно все эти методы называются квадратурными. Также говорится, что перечисленные методы используют квадратурные формулы.
\newpage

\section*{Постановка задачи}
\addcontentsline{toc}{section}{Постановка задачи}
Цель данной работы – реализовать параллельный алгоритм метода прямоугольников и сравнить его с последовательным алгоритмом.
\par Необходимо выполнить работу для вычисления интегралов:
$$\int\limits_a^b 
\int\limits_c^d 
\int\limits_e^f \,f(x,y,z)\,dx\,dy\,dz
$$
\par Данная цель предполагает решение следующих основных задач:
\begin{itemize}
\item Изучение и реализация общей схемы численного интегрирования для многомерных случаев.
\item Написание последовательной и параллельной версии этих алгоритмов.
\item Тестирование написанных алгоритмов посредствам тестов.
\item Провести численные эксперименты, и сравнить результаты тестирования.
\end{itemize}
\newpage

\section*{Метод решения}
\addcontentsline{toc}{section}{Метод решения}
Пусть функция $y = f(x)$ непрерывна на отрезке $[a; b]$. Нам требуется вычислить определенный
интеграл $\int\limits_a^b\,f(x)\,dx$.
\par Обратимся к понятию определенного интеграла. Разобьем отрезок $[a; b]$ на $n$ частей $[x_{i-1}; x_{i}]$, $i = 1, 2, ..., n$ точками $a = x0 < x1 < x2 < ... < x_{n-1} < x_{n} = b$. Внутри каждого отрезка $[x_{i-1}; x_{i}]$, $i =1, 2,...,n$ выберем точку $\xi{i}$. Так как по определению определенный интеграл есть предел интегральных сумм при бесконечном уменьшении длины элементарного отрезка разбиения $\lambda = \max_{i = 1,2,...,n} x_{i} - x_{i-1} \rightarrow 0$, то любая из интегральных сумм является приближенным значением интеграла $\int\limits_a^b\,f(x)\,dx\approx\sum_{i=1}^n f(\xi_{i})(x_{i}-x_{i-1})$.
\par Если разбить интегрируемый отрезок $[a; b]$ на одинаковые части точкой $h$, то получим $a = x0$, $x1=x0+h$, $x2=x0+2h$, $...$, $x_{n-1}=x0+(n-1)h$, $x_{n}=x0+nh$, то есть $h = x_{i}- x_{i-1} = (b-a)/n$, $i=1, 2, ..., n$. Серединами точек $\xi{i}$ выбираются элементарные отрезки $[x_{i-1}; x_{i}]$, $i = 1, 2, ..., n$ , значит $\xi{i}=xi-1 + h/2$, $i=1, 2, ..., n$.
\par Тогда приближенное значение $\int\limits_a^b\,f(x)\,dx\approx\sum_{i=1}^n f(\xi_{i}) (x_{i}-x_{i-1})$ записывается таким образом  $\int\limits_a^b\,f(x)\,dx\approx h \times \sum_{i=1}^n f(\xi_{i}) (x_{i-1} - \frac{h}{2})$. Данная формула называется формулой метода средних прямоугольников.
\par Такое название метод получает из-за характера выбора точек $\xi{i}$, где $h=\frac{b-a}{n}$ называют шагом разбиения отрезка $[a; b]$.
\par Основная cуть метода прямоугольников заключается в том, что в качестве приближенного значения определенного интеграла берут интегральную сумму.
\newpage

\section*{Схема распараллеливания}
\addcontentsline{toc}{section}{Схема распараллеливания}
В работе осуществляется распараллеливание интеграла на заданное число процессов. В начале имеется область интегрирование, заданная отрезками $[a_{i}; b_{i}]$, $i = 1, 2, ..., n$, количество которых совпадает с количеством интегралов n в многомерном интеграле. Также изначально задаётся шаг разбиения данного отрезка, так как для каждого примера имеется своё индивидуальное значение.
\par Один из наиболее простых подходов распараллеливания метода прямоугольников является геометрическая декомпозиция данных. Данный подход предполагает разделение данных на части и применение к ним одного и того же алгоритма. В численном интегрировании мы имеем дело с прямоугольной сеткой, в каждом узле которой вычисляется функция и умножается на квадрат шага. Вычисление функции в одном узле сетки не зависит от соседних узлов, таким образом, поделив сетку между процессами, можно получить параллельную версию.
\newpage

\section*{Описание программной реализации}
\addcontentsline{toc}{section}{Описание программной реализации}
Рассмотрим основные моменты программной реализации.
\par Программа состоит из заголовочного файла calculation.h в котором объявлены функции:
\begin{enumerate}
\item SequentialCalculation - вычисление многомерного интеграла методом прямоугольников последовательная версия.
\item ParallelCalculation - вычисление многомерного интеграла методом прямоугольников параллельная версия.
\end{enumerate}
\par В файле calculation.cpp содержится реализация данных функций.
\par В файле main.cpp содержаться тесты для проверки корректности программы.
\newpage

\section*{Подтверждение корректности}
\addcontentsline{toc}{section}{Подтверждение корректности}
Для подтверждения корректности в программе реализован набор тестов с использованием библиотеки для модульного тестирования: 
\par Правильность получаемых результатов проверяется на различных функциях разной сложности.
\par Примеры подынтегральных функций некоторых из них:
\begin{itemize}
\item $x + 3 + y * 2 * y + z * z * z - 8$
\item $x + y * y$
\item $x + y + z$
\end{itemize}
\par Все тесты проходят проверку, что является доказательством корректной работы программы
\newpage

\section*{Результаты экспериментов}
\addcontentsline{toc}{section}{Результаты экспериментов}
Эксперименты проводились на разном числе процессов для вычисления трехмерного интеграла для функции $F = x + y + z$ по области $x\in[0,1$], $y\in[1,57]$, $z\in[50,152]$, c числом разбиений равным $100$ для каждой переменной.
\par Вычисления проводились на ПК со следующими характеристиками:
\begin{itemize}
\item Процессор: Intel(R) Core(TM) i5-4690K 3.50GHz, ядер: 4, логических процессоров: 4;
\item Оперативная память: 8 ГБ DDR3;
\item ОС: Microsoft Windows 10;
\end{itemize}
\begin{table}[!h]
\centering
\begin{tabular}{lllll}
Кол-во процессов & Последовательно (сек) & Параллельно (сек) & Ускорение  \\
1  & 1.101 & 1.174 & 0.937  \\
4  & 1.127 & 0.299 & 3.769  \\
7  & 1.066 & 0.309 & 3.449  \\
11 & 1.101 & 0.342 & 3.219  \\
\end{tabular}
\caption{Результаты вычислительных экспериментов}
\end{table}
\par Полученные данные демонстрируют разность во времени работы при последовательном и параллельном вычислениях. По результатам можно сделать вывод, что параллельное выполнение программы выигрывает во времени у последовательного во всех случаях, кроме случая с одним процессом, что логично, ведь в параллельном алгоритме, в этой ситуации довольно большие накладные расходы. Можно сделать вывод: чем больше процессов, тем быстрее работает программа.
\newpage

\section*{Заключение}
\addcontentsline{toc}{section}{Заключение}
В результате выполнения лабораторной работы была разработана библиотека, реализующая параллельный метод интегрирования прямоугольниками, используя технологию MPI.
\par Для подтверждения корректности работы программы разработан и доведен до успешного выполнения набор тестов с использованием библиотеки модульного тестирования Google C++ Testing Framework.
\par По данным экспериментов удалось сравнить время работы параллельного и последовательного алгоритмов. Выявлено, что параллельный алгоритм интегрирования показывает более высокую эффективность на большем числе процессов.
\newpage

\begin{thebibliography}{1}
\addcontentsline{toc}{section}{Список литературы}
\bibitem{Strongin} Гергель В.П., Стронгин Р.Г. Основы параллельных вычислений для многопроцессорных вычислительных систем. Учебное пособие – Нижний Новгород: Изд-во ННГУ им. Н.И. Лобачевского, 2003. 184 с. ISBN 5-85746-602-4.
\bibitem{Grishagin} Гришагин В.А., Свистунов А.Н. Параллельное программирование на основе MPI. Учебное пособие – Нижний Новгород: Изд-во ННГУ им.Н.И. Лобачевского, 2005. - 93 с
\bibitem{Gergel} Гергель В.П., Национальный Открытый Университет «ИНТУИТ». Академия: ИнтернетУниверситет Суперкомпьютерных Технологий. Курс: Теория и практика параллельных вычислений. ISBN: 978-5-9556-0096-3.
\bibitem{Samarski} Самарский А.А., Гулин А.В. Численные методы. М.: Наука,1989. 432 с.
\end{thebibliography}
\newpage

\section*{Приложение}
\addcontentsline{toc}{section}{Приложение}
В данном разделе находится листинг всего кода, написанного в рамках лабораторной работы.
\begin{lstlisting}
// calculation.h

// Copyright 2020 Bulychev Vladislav
#ifndef MODULES_TASK_3_BULYCHEV_V_CALCULATION_OF_INTEGRALS_CALCULATION_H_
#define MODULES_TASK_3_BULYCHEV_V_CALCULATION_OF_INTEGRALS_CALCULATION_H_

#include <vector>

double SequentialCalculation(std::vector<double> a, std::vector<double> b, int n, double(*f)(std::vector<double>));

double ParallelCalculation(std::vector<double> a, std::vector<double> b, int n, double (*func)(std::vector<double>));

#endif  // MODULES_TASK_3_BULYCHEV_V_CALCULATION_OF_INTEGRALS_CALCULATION_H_
\end{lstlisting}

\begin{lstlisting}
// calculation.cpp

// Copyright 2020 Bulychev Vladislav
#include <mpi.h>
#include <vector>
#include <cmath>
#include "/modules/task_3/bulychev_v_calculation_of_integrals/calculation.h"

double SequentialCalculation(std::vector<double> a, std::vector<double> b, int n, double(*f)(std::vector<double>)) {
    int size_a = a.size();
    std::vector<double> h;
    double result = 0.0;
    std::vector <double> p(size_a);
    int num = pow(n, size_a);

    for (int i = 0; i < size_a; i++) {
        double t1 = b[i] - a[i];
        double t2 = t1 / n;
        h.push_back(t2);
    }

    for (int i = 0; i < num; i++) {
        for (int j = 0; j < size_a; j++) {
            double t3 = h[j] * 0.5;
            p[j] = (i % n) * h[j] + a[j] + t3;
        }
        result += f(p);
    }

    int t4 = size_a;
    double t5 = 1;
    for (int i = 0; i < t4; i++) {
        t5 = t5 * h[i];
    }

    result = result * t5;

    return result;
}

double ParallelCalculation(std::vector<double> a, std::vector<double> b, int n, double (*f)(std::vector<double>)) {
    int size, rank;
    MPI_Comm_size(MPI_COMM_WORLD, &size);
    MPI_Comm_rank(MPI_COMM_WORLD, &rank);

    int s = a.size();
    std::vector <double> p;
    double result = 0.0;
    double l_result = 0.0;

    int t3 = s - 1;
    int num = pow(n, t3);
    int score = num / size;
    int step = 0;
    if (rank < num % size) {
        score += 1;
        step = rank * score;
    } else {
        step = num % size + rank * score;
    }

    std::vector<double> h(s);
    if (rank == 0) {
        for (int i = 0; i < s; ++i) {
            double t1 = b[i] - a[i];
            double t2 = t1 / n;
            h[i] = t2;
        }
    }

    MPI_Bcast(&a[0], s, MPI_DOUBLE, 0, MPI_COMM_WORLD);
    MPI_Bcast(&h[0], s, MPI_DOUBLE, 0, MPI_COMM_WORLD);

    int t7 = s - 1;
    std::vector<double> tmp;
    for (int i = 0; i < score; ++i) {
        int number = step + i;
        p.resize(t7);
        for (int j = 0; j < s - 1; ++j) {
            double t6 = h[j] * 0.5;
            p[j] = (number % n) * h[j] + a[j] + t6;
        }
        for (int j = 0; j < n; ++j) {
            tmp = p;
            double t8 = h[t7] * 0.5;
            double t9 = j * h[t7] + t8 + a[t7];
            tmp.push_back(t9);
            l_result += f(tmp);
            tmp.clear();
        }
        p.clear();
    }

    int t4 = a.size();
    double t5 = 1;
    for (int i = 0; i < t4; i++) {
        t5 = t5 * h[i];
    }
    l_result = l_result * t5;

    MPI_Reduce(&l_result, &result, 1, MPI_DOUBLE, MPI_SUM, 0, MPI_COMM_WORLD);

    return result;
}
\end{lstlisting}

\begin{lstlisting}
// main.cpp

// Copyright 2020 Bulychev Vladislav
#include <mpi.h>
#include <gtest-mpi-listener.hpp>
#include <gtest/gtest.h>
#include <vector>
#include "/modules/task_3/bulychev_v_calculation_of_integrals/calculation.h"

double f1(std::vector<double> t) {
    double x = t[0];
    double y = t[1];
    return (x * 2 * x + 5 * y);
}

double f2(std::vector<double> t) {
    double x = t[0];
    double y = t[1];
    double z = t[2];
    return (x + y + 6 + z * z);
}

double f3(std::vector<double> t) {
    double x = t[0];
    double y = t[1];
    return (x + y + 1);
}

double f4(std::vector<double> t) {
    double x = t[0];
    double y = t[1];
    double z = t[2];
    double d = t[3];
    return (x + y + 6 + z * z * 2 - + d + 3 );
}

TEST(Calculation_Integraion, Return_correct_answer_sequential_method) {
    int rank;
    MPI_Comm_rank(MPI_COMM_WORLD, &rank);

    int s = 2;
    std::vector<double> a(s);
    std::vector<double> b(s);

    if (rank == 0) {
        a[0] = 5;
        b[0] = 14;

        a[1] = 3;
        b[1] = 21;

        ASSERT_NO_THROW(SequentialCalculation(a, b, 5, f3));
    }
}

TEST(Calculation_Integraion, Return_correct_answer_parallel_method) {
    int rank;
    MPI_Comm_rank(MPI_COMM_WORLD, &rank);

    int s = 2;
    std::vector<double> a(s);
    std::vector<double> b(s);

    if (rank == 0) {
        a[0] = 5;
        b[0] = 14;

        a[1] = 3;
        b[1] = 21;
    }

    ASSERT_NO_THROW(ParallelCalculation(a, b, 5, f3));
}

TEST(Calculation_Integraion, Test_1_for_two_dimensional) {
    int rank;
    MPI_Comm_rank(MPI_COMM_WORLD, &rank);

    int s = 2;
    std::vector<double> a(s);
    std::vector<double> b(s);

    if (rank == 0) {
        a[0] = 0;
        b[0] = 100;

        a[1] = 0;
        b[1] = 10;
    }

    double result_par = ParallelCalculation(a, b, 10, f3);

    if (rank == 0) {
        double result_seq = SequentialCalculation(a, b, 10, f3);

        ASSERT_NEAR(result_par, result_seq, 0.01);
    }
}

TEST(Calculation_Integraion, Test_2_for_two_dimensional) {
    int rank;
    MPI_Comm_rank(MPI_COMM_WORLD, &rank);

    int s = 2;
    std::vector<double> a(s);
    std::vector<double> b(s);

    if (rank == 0) {
        a[0] = 0;
        b[0] = 1;

        a[1] = 0;
        b[1] = 10;
    }

    double result_par = ParallelCalculation(a, b, 10, f1);

    if (rank == 0) {
        double result_seq = SequentialCalculation(a, b, 10, f1);

        ASSERT_NEAR(result_par, result_seq, 0.01);
    }
}

TEST(Calculation_Integraion, Test_for_three_dimensional) {
    int rank;
    MPI_Comm_rank(MPI_COMM_WORLD, &rank);

    int s = 3;
    std::vector<double> a(s);
    std::vector<double> b(s);
    if (rank == 0) {
        a[0] = 0;
        b[0] = 1;

        a[1] = 0;
        b[1] = 10;

        a[2] = 0;
        b[2] = 100;
    }

    double result_par = ParallelCalculation(a, b, 10, f2);

    if (rank == 0) {
        double result_seq = SequentialCalculation(a, b, 10, f2);
        ASSERT_NEAR(result_par, result_seq, 0.001);
    }
}

TEST(Calculation_Integraion, Test_for_four_dimensional) {
    int rank;
    MPI_Comm_rank(MPI_COMM_WORLD, &rank);

    int s = 4;
    std::vector<double> a(s);
    std::vector<double> b(s);
    if (rank == 0) {
        a[0] = 0;
        b[0] = 1;

        a[1] = 0;
        b[1] = 10;

        a[2] = 0;
        b[2] = 100;

        a[3] = 4;
        b[3] = 15;
    }

    double result_par = ParallelCalculation(a, b, 5, f4);

    if (rank == 0) {
        double result_seq = SequentialCalculation(a, b, 5, f4);
        ASSERT_NEAR(result_par, result_seq, 0.001);
    }
}

int main(int argc, char** argv) {
    ::testing::InitGoogleTest(&argc, argv);
    MPI_Init(&argc, &argv);

    ::testing::AddGlobalTestEnvironment(new GTestMPIListener::MPIEnvironment);
    ::testing::TestEventListeners& listeners =
        ::testing::UnitTest::GetInstance()->listeners();

    listeners.Release(listeners.default_result_printer());
    listeners.Release(listeners.default_xml_generator());

    listeners.Append(new GTestMPIListener::MPIMinimalistPrinter);
    return RUN_ALL_TESTS();
}
\end{lstlisting}

\end{document}