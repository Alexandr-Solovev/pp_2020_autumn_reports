\documentclass{report}

\usepackage[T2A]{fontenc}
\usepackage[utf8]{luainputenc}
\usepackage[english, russian]{babel}
\usepackage[pdftex]{hyperref}
\usepackage[14pt]{extsizes}
\usepackage{listings}
\usepackage{color}
\usepackage{geometry}
\usepackage{enumitem}
\usepackage{multirow}
\usepackage{graphicx}
\usepackage{indentfirst}

\geometry{a4paper,top=2cm,bottom=3cm,left=2cm,right=1.5cm}
\setlength{\parskip}{0.5cm}
\setlist{nolistsep, itemsep=0.3cm,parsep=0pt}

\lstset{language=C++,
		basicstyle=\footnotesize,
		keywordstyle=\color{blue}\ttfamily,
		stringstyle=\color{red}\ttfamily,
		commentstyle=\color{green}\ttfamily,
		morecomment=[l][\color{magenta}]{\#}, 
		tabsize=4,
		breaklines=true,
  		breakatwhitespace=true,
  		title=\lstname,       
}

\makeatletter
\renewcommand\@biblabel[1]{#1.\hfil}
\makeatother

\begin{document}

\begin{titlepage}

\begin{center}
Министерство науки и высшего образования Российской Федерации
\end{center}

\begin{center}
Федеральное государственное автономное образовательное учреждение высшего образования \\
Национальный исследовательский Нижегородский государственный университет им. Н.И. Лобачевского
\end{center}

\begin{center}
Институт информационных технологий, математики и механики
\end{center}

\vspace{4em}

\begin{center}
\textbf{\LargeОтчет по лабораторной работе} \\
\end{center}
\begin{center}
\textbf{\Large«Построение выпуклой оболочки для компонент бинарного изображения»} \\
\end{center}

\vspace{4em}

\newbox{\lbox}
\savebox{\lbox}{\hbox{text}}
\newlength{\maxl}
\setlength{\maxl}{\wd\lbox}
\hfill\parbox{7cm}{
\hspace*{5cm}\hspace*{-5cm}\textbf{Выполнил:} \\ студент группы 381806-1 \\ Соловьев А.Ю.\\
\\
\hspace*{5cm}\hspace*{-5cm}\textbf{Проверил:}\\ доцент кафедры МОСТ, \\ кандидат технических наук \\ Сысоев А. В.\\
}
\vspace{\fill}

\begin{center} Нижний Новгород \\ 2020 \end{center}

\end{titlepage}

\setcounter{page}{2}

% Содержание
\tableofcontents
\newpage

% Введение
\section*{Введение}
\addcontentsline{toc}{section}{Введение}
Задача построения выпуклой оболочки для компонент бинарного изображения важна, и используется во многих сферах человеческой деятельности. Она позволяет разрешить целый ряд других, иногда с первого взгляда не связанных с ней вопросов: построение диаграмм Вороного, построение триангуляций и т.д. Построение выпуклой оболочки конечного множества точек на плоскости довольно широко исследовано и имеет множество приложений. Очень широко алгоритмы построения выпуклой оболочки используются в геоинформатике и геоинформационных системах. Я использую язык C++, с возможносятми технологий MPI, для реализации данной задачи.
\newpage

% Постановка задачи
\section*{Постановка задачи}
\addcontentsline{toc}{section}{Постановка задачи}
Данная задача разбивается на две подзадачи. Первая - перевести бинарное изображение в набор точек вектора, а вторая -  реализовать построение выпуклой оболочки используя один из алгоритмов. Я решил выполнить эту задачу, используя проход Джарвиса для точек $p_1, p_2, ..., p_n,$ где $n \ge 1$. Для этого необходимо написать две реализации: последовательную и параллельную.
\par
Также нужно провести сравнение времени работы для параллельной и последовательной реализации прохода Джарвиса для разных наборов данных и на разном количестве процессов. Также нужно проанализировать полученные результаты.
\par
Кроме того, мне необходимо проверить корректность работы алгоритма программы для разного количества процессов, и графически продемонстрировать построение выпуклой оболочки, используя технологии OpenCV. 
\newpage

% Метод решения
\section*{Описание алгоритма}
\addcontentsline{toc}{section}{Описание алгоритма}
Изначально мы переводим бинарное изображение, заданное в форме двумерного массива, в вектор точек. Для этого мы считаем количество "1" в исходном двумерном массиве, выделяем соответствующий вектор, и добавляем в него координаты точек. Затем реализовываем алгоритм Джарвиса. Пусть мы имеем множество точек $P=\{\,p_1, p_2, ..., p_n\}$. В качестве начальной берётся самая левая нижняя точка $p_1$ (её можно найти за $O(n)$ обычным проходом по всем точкам), она точно является вершиной выпуклой оболочки. Следующей точкой  $p_2$ берём такую точку, которая имеет наименьший положительный полярный угол относительно точки $p_1$, как начала координат. После этого для каждой точки $p_i$, где $(2<i<=|P|)$, против часовой стрелки ищется такая точка $p_{i+1}$, путём нахождения за $O(n)$ среди оставшихся точек, в которой будет образовываться наибольший угол между прямыми $p_{i-1} p_i$ и $p_{i}p_{i+1}$. Она и будет следующей вершиной выпуклой оболочки. Сам угол при этом не ищется, а ищется только его косинус через скалярное произведение между лучами  $p_{i-1}p_{i}$ и $p_{i}p'_{i+1}$, где  $p_{i}$ — последний найденный минимум,  $p_{i-1}$ — предыдущий минимум(претедент), а $p'_{i+1}$ — претендент на следующий минимум. Новым минимумом будет та точка, в которой косинус будет принимать наименьшее значение (чем меньше косинус, тем больше его угол). Нахождение вершин выпуклой оболочки продолжается до тех пор, пока $p_{i+1}\neq p_{1}$. В тот момент, когда следующая точка в выпуклой оболочке совпала с первой, алгоритм останавливается — выпуклая оболочка построена.
\newpage

% Схема распараллеливания
\section*{Схема распараллеливания}
\addcontentsline{toc}{section}{Схема распараллеливания}
Для распараллеливания алгоритма построения выпуклой оболочки необходимо разделить, полученный после перевода бинарного изображения, вектор на части, в зависимости от числа процессов. Количество частей совпадает с числом процессов. Если размер изначального массива не делится нацело на число процессов, то останутся точки которые не войдут в блок. Их мы отдадим корневому процессу.
\par
Далее корневой процесс раздает части вектора с точками каждому процессу, учитывая свдиг, и затем, каждый процесс ищет в своем наборе точек самую крайнюю левую точку, передавая ее в корневой процесс для обработки.
\par
Корневой процесс производит сравнение, и находит глобальную левую нижнюю точку. Она будет являться первой точкой нашей оболочки. Затем он передает координаты этой точки всем процессам.
\par
После нахождения первой точки, в каждом процессе локально выполняется проход Джарвиса для имеющейся части вектора точек. Завершив проход, процессы отправляют найденную точку на корневой процесс, и в нем снова выбирается одна из полученных точек(наиболее подходящая). Таким образом, получаем вторую глобальную точку, которая принадлежит выпуклой оболочке. Снова передаем ее из корневого процесса остальным. Этот алгоритм выполняется до тех пор, пока наша новая точка не совпадет с первой. 
\newpage

% Описание программной реализации
\section*{Описание программной реализации}
\addcontentsline{toc}{section}{Описание программной реализации}
Для перевода изображения в набор точек, используется функция, переводящая компоненты бинарного изображения в вектор точек:
\begin{lstlisting}
std::vector<Point> interpriate_basic(int ** image, int height, int width);
\end{lstlisting}
\par Входными параметрами являются изображение, и его размеры. Функция возвращает вектор точек.
\par
Алгоритм последовательного построения выпуклой оболочки вызывается при помощи функции:
\begin{lstlisting}
std::vector<Point> buildConvexHull(const std::vector<Point> &mas);
\end{lstlisting}
\par Входным параметром функции является вектор точек исходного изображения, по которым нужно построить выпуклую оболочку. Выходными данными является вектор точек, входящих в выпуклую оболочку.
\par Алгоритм параллельного построения выпуклой оболочки вызывается при помощи функции:
\begin{lstlisting}
std::vector<Point> buildConvexHullParallel(const std::vector<Point> &mas);
\end{lstlisting}
\par Входным параметром функции является вектор точек исходного изображения, по которым нужно построить выпуклую оболочку. Выходными данными является вектор точек, входящих в выпуклую оболочку.
\par Также мы имеем реализации методов нахождения первой точки выпуклой оболочки.
\begin{itemize}
\par \item Данная функция ищет самую крайнюю левую точку в переданном векторе точек и возвращает её индекс:
\begin{lstlisting}
size_t searchMinPoint(const std::vector<Point> &mas);
\end{lstlisting}
\par \item Данная функция является параллельной реализацией поиска самой крайней левой точки:
\begin{lstlisting}
Point searchMinPointParallel(const std::vector<Point> &mas, int rank, int procCount);
\end{lstlisting}
\par Входным параметром является ссылка на неизменяемый вектор точек,  ранг процесса, и их количество. Выходные параметры самая крайняя левая точка.
\par \item Данная функция проверяет - является ли потенциальная точка, наилучшей для входа в оболочку.
\begin{lstlisting}
bool pointCheck(const Point& back, const Point& current, const Point& challenger);
\end{lstlisting}
\par Входным параметром является ссылки на неизменяемые точки. Функция возврашает \verb|true|, если новый претендент больше подходит.
\end{itemize}
\newpage

% Подтверждение корректности
\section*{Подтверждение корректности}
\addcontentsline{toc}{section}{Подтверждение корректности}
Для подтверждения корректности в программе представлен набор тестов, разработанных с помощью использования Google C++ Testing Framework.
\par Набор представляет из себя тесты, проверяющие корректность вычислений (проверка на совпадение результатов работы параллельной программы и последовательной), а также эффективность (вычисление и сравнение времени работы последовательной и параллельной реализации).
\newpage

% Результаты экспериментов
\section*{Результаты экспериментов}
\addcontentsline{toc}{section}{Результаты экспериментов}
Вычислительные эксперименты для оценки алгоритма построения выпуклой оболочки проходом Джарвиса проводились на оборудовании со следующей аппаратной конфигурацией:

\begin{itemize}
\item Процессор: Intel Core i5-3450H CPU @ 3.10GHz, ядер: 4;
\item Оперативная память: 8 Gb (DDR3);
\item ОС: Microsoft Windows 10 Максимальная 64-bit.
\end{itemize}

\par Для проведения экспериментов была построена выпуклая оболочка для изображения 5000 * 5000 точек. 
\par Результаты экспериментов представлены в Таблице 1.

\begin{table}[!h]
\caption{Результаты вычислительных экспериментов}
\centering
\begin{tabular}{llllll}
Количество процессов & Параллельная & Последовательная & Ускорение  \\
1        & 3.85061        & 3.70978      & 0.96                     \\
2        & 1,94932        & 3.70212      & 1.90                     \\
3        & 1.35345        & 3.76861      & 2,78                     \\
4        & 1.06695        & 3.71491      & 3.48                     \\
5        & 1.20345        & 3.85271      & 3,21                     \\
6        & 1.39345        & 3.75271      & 2,69                     \\
\end{tabular}
\end{table}

\par Результаты полученные в ходе вычислительных экспериментов действительно подтверждают эффективность параллельной реализации в сравнении с последовательной. Увеличение количества процессов не имеет смысла, так как увеличиваются накладные расходы. Они могут принести большую эффективность, на большем по размеру изображении. Наибольший выигрыш при использовании параллельного алгоритма, мы получаем при использовании 4 процессоров для данной аппаратной кофигурации.
\newpage

% Заключение
\section*{Заключение}
\addcontentsline{toc}{section}{Заключение}
В ходе выполнения лабораторной работы я реализовал последовательную и параллельную реализации алгоритма построения выпуклой оболочки для компонент бинарного изображения, используя прохода Джарвиса.
\newpage

% Список литературы
\begin{thebibliography}{1}
\addcontentsline{toc}{section}{Список литературы}
\bibitem{Gergel} Гергель В. П., Стронгин Р. Г. Основы параллельных вычислений для многопроцессорных вычислительных систем. – 2003. (дата обращения: 4.12.2020)
\end{thebibliography}
\newpage

% Приложение
\section*{Приложение}
\addcontentsline{toc}{section}{Приложение}
\begin{lstlisting}
// Copyright 2020 Solovev Aleksandr
#include <mpi.h>
#include <gtest-mpi-listener.hpp>
#include <gtest/gtest.h>
#include <vector>
#include <iostream>
#include "../../../modules/task_3/solovev_a_building_image/building_image.h"

TEST(MyAlgos, Test_Data_Set_250) {
    int rank;
    MPI_Comm_rank(MPI_COMM_WORLD, &rank);
    int height = 250;
    int width = 250;
    int ** image = new int*[height];
    for (int i = 0; i < height; i++) {
        image[i] = new int[width];
    }
    for (int i = 0; i < height; i++) {
        for (int j = 0; j < width; j++) {
            image[i][j] = 1;
        }
    }

    std::vector<Point> points_parallel = interpriate_basic(image, height, width);
    std::vector<Point> points_sequence = points_parallel;
    std::vector<int> points_parallel_x;
    std::vector<int> points_parallel_y;
    std::vector<int> points_sequence_x;
    std::vector<int> points_sequence_y;

    points_parallel = buildConvexHullParallel(points_parallel);


    if (rank == 0) {
        for (size_t i = 0; i < points_parallel.size(); i++) {
            points_parallel_x.emplace_back(points_parallel[i].x);
            points_parallel_y.emplace_back(points_parallel[i].y);
        }

        points_sequence = buildConvexHull(points_sequence);

        for (size_t i = 0; i < points_parallel.size(); i++) {
            points_sequence_x.emplace_back(points_parallel[i].x);
            points_sequence_y.emplace_back(points_parallel[i].y);
        }


        ASSERT_EQ(points_parallel_x, points_sequence_x);
        ASSERT_EQ(points_parallel_y, points_sequence_y);
    }
}

TEST(MyAlgos2, Test_Data_Set_500) {
    int rank;
    MPI_Comm_rank(MPI_COMM_WORLD, &rank);
    int height = 500;
    int width = 500;
    int ** image = new int*[height];
    for (int i = 0; i < height; i++) {
        image[i] = new int[width];
    }
    for (int i = 0; i < width; i++) {
        for (int j = 0; j < height; j++) {
            image[i][j] = 1;
        }
    }

    std::vector<Point> points_parallel = interpriate_basic(image, height, width);
    std::vector<Point> points_sequence = points_parallel;
    std::vector<int> points_parallel_x;
    std::vector<int> points_parallel_y;
    std::vector<int> points_sequence_x;
    std::vector<int> points_sequence_y;

    points_parallel = buildConvexHullParallel(points_parallel);


    if (rank == 0) {
        for (size_t i = 0; i < points_parallel.size(); i++) {
            points_parallel_x.emplace_back(points_parallel[i].x);
            points_parallel_y.emplace_back(points_parallel[i].y);
        }

        points_sequence = buildConvexHull(points_sequence);

        for (size_t i = 0; i < points_parallel.size(); i++) {
            points_sequence_x.emplace_back(points_parallel[i].x);
            points_sequence_y.emplace_back(points_parallel[i].y);
        }


        ASSERT_EQ(points_parallel_x, points_sequence_x);
        ASSERT_EQ(points_parallel_y, points_sequence_y);
    }
}

TEST(MyAlgos3, Test_Data_Set_1000) {
    int rank;
    MPI_Comm_rank(MPI_COMM_WORLD, &rank);
    int height = 1000;
    int width = 1000;
    int ** image = new int*[height];
    for (int i = 0; i < height; i++) {
        image[i] = new int[width];
    }

    for (int i = 0; i < height; i++) {
        for (int j = 0; j < width; j++) {
            image[i][j] = 1;
        }
    }

    std::vector<Point> points_parallel = interpriate_basic(image, height, width);
    std::vector<Point> points_sequence = points_parallel;
    std::vector<int> points_parallel_x;
    std::vector<int> points_parallel_y;
    std::vector<int> points_sequence_x;
    std::vector<int> points_sequence_y;

    points_parallel = buildConvexHullParallel(points_parallel);


    if (rank == 0) {
        for (size_t i = 0; i < points_parallel.size(); i++) {
            points_parallel_x.emplace_back(points_parallel[i].x);
            points_parallel_y.emplace_back(points_parallel[i].y);
        }

        points_sequence = buildConvexHull(points_sequence);

        for (size_t i = 0; i < points_parallel.size(); i++) {
            points_sequence_x.emplace_back(points_parallel[i].x);
            points_sequence_y.emplace_back(points_parallel[i].y);
        }


        ASSERT_EQ(points_parallel_x, points_sequence_x);
        ASSERT_EQ(points_parallel_y, points_sequence_y);
    }
}

TEST(MyAlgos4, Test_Data_Set_2000) {
    int rank;
    MPI_Comm_rank(MPI_COMM_WORLD, &rank);
    int height = 2000;
    int width = 2000;
    int ** image = new int*[height];
    for (int i = 0; i < height; i++) {
        image[i] = new int[width];
    }

    for (int i = 0; i < height; i++) {
        for (int j = 0; j < width; j++) {
            image[i][j] = 1;
        }
    }

    std::vector<Point> points_parallel = interpriate_basic(image, height, width);
    std::vector<Point> points_sequence = points_parallel;
    std::vector<int> points_parallel_x;
    std::vector<int> points_parallel_y;
    std::vector<int> points_sequence_x;
    std::vector<int> points_sequence_y;

    points_parallel = buildConvexHullParallel(points_parallel);


    if (rank == 0) {
        for (size_t i = 0; i < points_parallel.size(); i++) {
            points_parallel_x.emplace_back(points_parallel[i].x);
            points_parallel_y.emplace_back(points_parallel[i].y);
        }

        points_sequence = buildConvexHull(points_sequence);

        for (size_t i = 0; i < points_parallel.size(); i++) {
            points_sequence_x.emplace_back(points_parallel[i].x);
            points_sequence_y.emplace_back(points_parallel[i].y);
        }


        ASSERT_EQ(points_parallel_x, points_sequence_x);
        ASSERT_EQ(points_parallel_y, points_sequence_y);
    }
}

TEST(MyAlgos5, Test_Data_Set_2500) {
    int rank;
    MPI_Comm_rank(MPI_COMM_WORLD, &rank);
    int height = 2500;
    int width = 2500;
    int ** image = new int*[height];
    for (int i = 0; i < height; i++) {
        image[i] = new int[width];
    }

    for (int i = 0; i < height; i++) {
        for (int j = 0; j < width; j++) {
            image[i][j] = 1;
        }
    }

    std::vector<Point> points_parallel = interpriate_basic(image, height, width);
    std::vector<Point> points_sequence = points_parallel;
    std::vector<int> points_parallel_x;
    std::vector<int> points_parallel_y;
    std::vector<int> points_sequence_x;
    std::vector<int> points_sequence_y;

    double startMY = MPI_Wtime();
    points_parallel = buildConvexHullParallel(points_parallel);
    double endMY = MPI_Wtime();


    if (rank == 0) {
        for (size_t i = 0; i < points_parallel.size(); i++) {
            points_parallel_x.emplace_back(points_parallel[i].x);
            points_parallel_y.emplace_back(points_parallel[i].y);
        }

        double startT = MPI_Wtime();
        points_sequence = buildConvexHull(points_sequence);
        double endT = MPI_Wtime();

        for (size_t i = 0; i < points_parallel.size(); i++) {
            points_sequence_x.emplace_back(points_parallel[i].x);
            points_sequence_y.emplace_back(points_parallel[i].y);
        }
        std::cout << "Parallel_Time: " << endMY - startMY << std::endl;
        std::cout << "Sequence_Time: " << endT - startT << std::endl;

        ASSERT_EQ(points_parallel_x, points_sequence_x);
        ASSERT_EQ(points_parallel_y, points_sequence_y);
    }
}

int main(int argc, char** argv) {
    ::testing::InitGoogleTest(&argc, argv);
    MPI_Init(&argc, &argv);

    ::testing::AddGlobalTestEnvironment(new GTestMPIListener::MPIEnvironment);
    ::testing::TestEventListeners& listeners =
        ::testing::UnitTest::GetInstance()->listeners();

    listeners.Release(listeners.default_result_printer());
    listeners.Release(listeners.default_xml_generator());

    listeners.Append(new GTestMPIListener::MPIMinimalistPrinter);
    return RUN_ALL_TESTS();
}
\end{lstlisting}
\begin{lstlisting}
// Copyright 2020 Solovev Aleksandr
#ifndef MODULES_TASK_3_SOLOVEV_A_BUILDING_IMAGE_BUILDING_IMAGE_H_
#define MODULES_TASK_3_SOLOVEV_A_BUILDING_IMAGE_BUILDING_IMAGE_H_

#include <mpi.h>
#include <vector>
#include <iostream>


struct Point {
    int x, y;
    Point() = default;
    Point(int X, int Y) {
        x = X;
        y = Y;
    }
    const Point operator=(const Point& point) {
        x = point.x;
        y = point.y;
        return *this;
    }

    bool operator!=(const Point& point) {
        if ((x != point.x) || (y != point.y)) {
            return true;
        } else {
            return false;
        }
    }

    bool operator==(const Point& point) {
        if ((x == point.x)&&(y == point.y)) {
            return true;
        } else {
            return false;
        }
    }
};

std::vector<Point> interpriate_basic(int ** image, int height, int width);
size_t searchFirstPoint(const std::vector<Point> &mas);
Point searchFirstPointParallel(const std::vector<Point> &mas, int rank, int procCount);
std::vector<Point> buildConvexHullParallel(const std::vector<Point> &mas);
std::vector<Point> buildConvexHull(const std::vector<Point> &mas);
bool pointCheck(const Point& back, const Point& current, const Point& challenger);
#endif  // MODULES_TASK_3_SOLOVEV_A_BUILDING_IMAGE_BUILDING_IMAGE_H_
\end{lstlisting}
\begin{lstlisting}
// Copyright 2020 Solovev Aleksandr
#include <vector>
#include "../../../modules/task_3/solovev_a_building_image/building_image.h"

std::vector<Point> interpriate_basic(int ** image, int height, int width) {
    int count = 0;
     for (int i = 0; i < height; i++) {
        for (int j = 0; j < width; j++) {
            if (image[i][j] == 1)
            count++;
        }
    }
    std::vector<Point> result(count);
    count = 0;
    for (int i = 0; i < height; i++) {
        for (int j = 0; j < width; j++) {
            if (image[i][j] == 1) {
            result[count].x = i;
            result[count].y = j;
            count++;
            }
        }
    }
    return result;
}

bool pointCheck(const Point& previous, const Point& current, const Point& potential) {
    double x;
    int potential_x = potential.x;
    int potential_y = potential.y;
    int prev_x = previous.x;
    int prev_y = previous.y;
    int curr_x = current.x;
    int curr_y = current.y;
    if (potential_x == curr_x) {
        if (potential_x == prev_x) {
            if (potential_y > curr_y) {
                if (potential_y > prev_y) {
                    return true;
                } else {
                    return false;
                }
            }
            if (potential_y < curr_y) {
                if (potential_y < prev_y) {
                    return true;
                } else {
                    return false;
                }
            }
        }
        x = potential_x;
    } else {
        if (potential_y == curr_y) {
            x = prev_x;
            if (potential_x > curr_x) {
                if (potential_y == prev_y) {
                    if (prev_x > potential_x) {
                        return false;
                    } else {
                        if ((prev_x == potential_x) && (potential_y > prev_y)) {
                            return false;
                        }
                        return true;
                    }
                } else {
                    return potential_y <= prev_y;
                }
            } else {
                if (potential_x < curr_x) {
                    if (potential_y == prev_y) {
                        if (prev_x > potential_x) {
                            return true;
                        } else {
                            if ((prev_x == potential_x) && (prev_y > potential_y)) {
                                return true;
                            }
                            return false;
                        }
                    } else {
                        if (potential_y <= prev_y) {
                            return false;
                        } else {
                            return true;
                        }
                    }
                }
            }
        } else {
            double tan1 = (static_cast<double>(potential_y - curr_y)) /
            (static_cast<double>(potential_x - curr_x));
            double tan2 = tan1 + 1.0;
            if ((prev_x - curr_x) != 0) {
                tan2 = (prev_y - curr_y) / (prev_x - curr_x);
            }
            if (tan1 == tan2) {
                if (potential_y > curr_y) {
                    return potential_y > prev_y;
                } else {
                    return potential_y < prev_y;
                }
            } else {
                x = static_cast<double>(prev_y - potential_y) /
                tan1 + static_cast<double>(potential_x);
            }
        }
    }
    if (potential_y > curr_y) {
        if (potential_x > curr_x) {
            return !(x <= prev_x);
        } else {
            return x > prev_x;
        }
    } else {
        if ((potential_y - curr_y) != 0) {
            return x <= prev_x;
        }
    }
    return false;
}

size_t searchFirstPoint(const std::vector<Point>& mas) {
    size_t start = 0;
    for (int i = 0; i < static_cast<int>(mas.size()); i++) {
        if (mas[start].x > mas[i].x) {
            start = i;
        } else {
            if ((mas[start].x == mas[i].x) && (mas[start].y > mas[i].y)) {
                start = i;
            }
        }
    }
    return start;
}

Point searchFirstPointParallel(const std::vector<Point>& mas, int rank, int procCount) {
    Point local_right_point = mas[searchFirstPoint(mas)];
    std::vector<Point> local_points(procCount);
    MPI_Gather(&local_right_point, 2, MPI_INT, &local_points[0], 2, MPI_INT, 0, MPI_COMM_WORLD);
    if (rank == 0) {
        local_right_point = local_points[searchFirstPoint(local_points)];
    }
    MPI_Bcast(&local_right_point, 2, MPI_INT, 0, MPI_COMM_WORLD);
    return local_right_point;
}

std::vector<Point> buildConvexHullParallel(const std::vector<Point>& mas) {
    int rank, procCount;
    MPI_Comm_rank(MPI_COMM_WORLD, &rank);
    MPI_Comm_size(MPI_COMM_WORLD, &procCount);
    int elements_count = static_cast<int>(mas.size());
    if (elements_count == 1) {
        return mas;
    }
    if (elements_count == 2) {
        return mas;
    }
    int elements_part_count = elements_count / procCount;
    int remainder = elements_count % procCount;
    if ((elements_part_count == 1) || (static_cast<size_t>(procCount) > mas.size())) {
        if (rank == 0) {
            return buildConvexHull(mas);
        } else {
            return mas;
        }
    }
    int* recvcount = new int[procCount];
    int* delay = new int[procCount];
    recvcount[0] = 2 * (elements_part_count + remainder);
    delay[0] = 0;
    for (int i = 1; i < procCount; i++) {
        recvcount[i] = elements_part_count * 2;
    }
    for (int i = 1; i < procCount; i++) {
        delay[i] = (elements_part_count * i + remainder) * 2;
    }
    std::vector<Point> part_vector(0);
    if (rank == 0) {
        part_vector.resize(elements_part_count + remainder);
    } else {
        part_vector.resize(elements_part_count);
    }
    MPI_Scatterv(mas.data(), recvcount, delay, MPI_INT, part_vector.data(),
        recvcount[rank], MPI_INT, 0, MPI_COMM_WORLD);
    Point left_point = searchFirstPointParallel(part_vector, rank, procCount);
    std::vector<Point> convex_hull;
    convex_hull.emplace_back(left_point);
    std::vector<Point> local_points(procCount);
    Point next;
    do {
        next = part_vector[0];
        if (next == convex_hull.back()) {
            next = part_vector[1];
        }
        for (int i = 0; i < static_cast<int>(part_vector.size()); i++) {
            if ((part_vector[i] == convex_hull.back()) || (convex_hull.back() == next)) {
                continue;
            }
            if (pointCheck(convex_hull.back(), next, part_vector[i]) || (part_vector[i] == next)) {
                next = part_vector[i];
            }
        }
        MPI_Gather(&next, 2, MPI_INT, &local_points[0], 2, MPI_INT, 0, MPI_COMM_WORLD);
        if (rank == 0) {
            for (int i = 0; i < static_cast<int>(local_points.size()); i++) {
                if (pointCheck(convex_hull.back(), next, local_points[i])) {
                    next = local_points[i];
                }
            }
        }
        MPI_Bcast(&next, 2, MPI_INT, 0, MPI_COMM_WORLD);
        convex_hull.emplace_back(next);
        if (convex_hull.size() == mas.size()) {
            break;
        }
    } while (convex_hull.back() != left_point);
    return convex_hull;
}

std::vector<Point> buildConvexHull(const std::vector<Point>& mas) {
    if ((mas.size() == 1) || (mas.size() == 2)) {
        return mas;
    }
    std::vector<Point> part_vector(mas);
    Point left_point = part_vector[searchFirstPoint(part_vector)];
    std::vector<Point> convex_hull;
    convex_hull.emplace_back(left_point);
    Point next;
    do {
        next = part_vector[0];
        if (next == convex_hull.back()) {
            next = part_vector[1];
        }
        for (int i = 0; i < static_cast<int>(part_vector.size()); i++) {
            if ((part_vector[i] == convex_hull.back()) || (convex_hull.back() == next)) {
                continue;
            }
            if ((part_vector[i] == next) || pointCheck(convex_hull.back(), next, part_vector[i])) {
                next = part_vector[i];
            }
        }
        convex_hull.push_back(next);
        if (convex_hull.size() == mas.size()) {
            break;
        }
    } while (convex_hull.back() != left_point);

    return convex_hull;
}

\end{lstlisting}
\end{document}
