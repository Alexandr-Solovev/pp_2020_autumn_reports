\documentclass{report}

\usepackage[T2A]{fontenc}
\usepackage[utf8]{luainputenc}
\usepackage[english, russian]{babel}
\usepackage[pdftex]{hyperref}
\usepackage[14pt]{extsizes}
\usepackage{listings}
\usepackage{color}
\usepackage{geometry}
\usepackage{enumitem}
\usepackage{multirow}
\usepackage{graphicx}
\usepackage{indentfirst}

\geometry{a4paper,top=2cm,bottom=3cm,left=2cm,right=1.5cm}
\setlength{\parskip}{0.5cm}
\setlist{nolistsep, itemsep=0.3cm,parsep=0pt}

\lstset{language=C++,
basicstyle=\footnotesize,
keywordstyle=\color{blue}\ttfamily,
stringstyle=\color{red}\ttfamily,
commentstyle=\color{green}\ttfamily,
morecomment=[l][\color{magenta}]{\#},
tabsize=4,
breaklines=true,
breakatwhitespace=true,
title=\lstname,
}

\makeatletter
\renewcommand\@biblabel[1]{#1.\hfil}
\makeatother

\begin{document}

\begin{titlepage}

\begin{center}
Министерство науки и высшего образования Российской Федерации
\end{center}

\begin{center}
Федеральное государственное автономное образовательное учреждение высшего образования \\
Национальный исследовательский Нижегородский государственный университет им. Н.И. Лобачевского
\end{center}

\begin{center}
Институт информационных технологий, математики и механики
\end{center}

\vspace{4em}

\begin{center}
\textbf{\LargeОтчет по лабораторной работе} \\
\end{center}
\begin{center}
\textbf{\Large«Поразрядная сортировка для целых чисел с четно-нечетным слиянием Бэтчера»} \\
\end{center}

\vspace{4em}

\newbox{\lbox}
\savebox{\lbox}{\hbox{text}}
\newlength{\maxl}
\setlength{\maxl}{\wd\lbox}
\hfill\parbox{7cm}{
\hspace*{5cm}\hspace*{-5cm}\textbf{Выполнил:} \\ студент группы 381808-1 \\ Гаврилов Н. И.\\
\\
\hspace*{5cm}\hspace*{-5cm}\textbf{Проверил:}\\ доцент кафедры МОСТ, \\ кандидат технических наук \\ Сысоев А.В.\\
}
\vspace{\fill}

\begin{center} Нижний Новгород \\ 2020 \end{center}

\end{titlepage}

\setcounter{page}{2}

% Содержание
\tableofcontents

\newpage
% Введение
\begin{center}
\section*{Введение}
\end{center}
\addcontentsline{toc}{section}{Введение}

Вопрос о размещении данных в указанном (возрастающем или убывающем) порядке возникает довольно часто. От порядка, в котором хранятся элементы в памяти, во многом зависит скорость и простота алгоритмов, предназначенных для их обработки. Таким образом, сортировка является инструментом, полезным в самых различных ситуациях.
\par Существует много различных методов сортировки, различающихся по эффективности. В данной работе рассматривается метод поразрядной сортировки.
\par Для ускорения работы следует разбивать данные на группы меньшего размера и сортировать эти группы парраллельно. В таком случае возникает проблема слияния отсортированных групп данных. С этим нам поможет четно-нечетное слияние Бетчера.
\par Целью данной работы является реализация поразрядной сортировки для целых чисел с четно-нечетным слиянием Бэтчера. 

\newpage
% Постановка задачи
\begin{center}
\section*{Постановка задачи}
\end{center}
\addcontentsline{toc}{section}{Постановка задачи}

В данной работе необходимо:
\begin{enumerate}
\item Реализовать поразрядную сортировку для целых чисел.
\item Реализовать параллельную поразрядную сортировку для целых чисел с четно-нечетным слиянием Бэтчера. 
\item Сравнить время работы без распараллеливания и с распараллеливанием алгоритма.
\end{enumerate}

\newpage
% Описание алгоритма последовательной сортировки
\begin{center}
\section*{Описание алгоритма последовательной сортировки}
\end{center}
\addcontentsline{toc}{section}{Описание алгоритма последовательной сортировки}

Алгоритм поразрядной сортировки: 
\begin{enumerate}
\item Выбираем rank(система счисления).
\item Для каждого разряда(начиная с младшего) выполняем следующие:
\begin{enumerate}
\item Создаем n массивов(n=rank-1).
\item Распределяем элементы по этим массивам в зависимости от цифры расположенной на данном разряде.
\item Собираем созданные массивы по порядку.
\end{enumerate}
\end{enumerate}
В итоге получается отсортированый массив

\newpage
% Схема распараллеливания
\begin{center}
\section*{Схема распараллеливания}
\end{center}
\addcontentsline{toc}{section}{Схема распараллеливания}

\begin{enumerate}
\item Кол-во элементов массива должно быть кратно кол-ву процессов, если это не так, добавляем элементы, равные максимально возможному числу в конец массива, чтобы условие выполнилось.
\item Разбиваем массив на n частей(n=кол-во процессов) и распределяем по процессам.
\item Каждый процесс сортирует свою часть с помощью поразрядной сортировки.
\item Создаем сеть слияния Бэтчера.
\item Каждый элемент созданной сети - пара номеров проццесов которые должны выполнить слияние и распределить элементы, поэтому проходим по сети и выполняем вышесказанное.
\item Собираем массив из частей содержашихся в процессах.
\item Удаляем элементы добавленные для кратности массива. Они находятся в конце.
\end{enumerate}

\newpage
% Описание программной реализации
\begin{center}
\section*{Описание программной реализации}
\end{center}
\addcontentsline{toc}{section}{Описание программной реализации}

Реализация последовательного алгоритма поразрядной сортировки:
\begin{itemize}
\item Функция возвращающая цифру из числа:
\begin{lstlisting}
	int8_t GetDigit(int value, uint8_t digidNum, uint8_t rang)
\end{lstlisting}
\item Функция возвращающая кол-во разрядов числа:
\begin{lstlisting}
	uint8_t GetNumOfDigits(int value, uint8_t rang)
\end{lstlisting}
\item Функция объединяющая векторы:
\begin{lstlisting}
	void ConcatVectors(const std::vector<std::vector<int>>& from, std::vector<int>* dest)
\end{lstlisting}
\item Функция сортирующая по определенному разряду:
\begin{lstlisting}
	void SortByDigid(std::vector<int>* data, uint8_t digitNum, uint8_t rang)
\end{lstlisting}
\item Сама функия сортировки:
\begin{lstlisting}
	void SortByNumPlace(std::vector<int>* data, uint8_t rang)
\end{lstlisting}
\end{itemize}

Реализация построения сети слияния Бэтчера:
\begin{itemize}
\item Функция разделяющая вектор на два по четным и нечетным местам:
\begin{lstlisting}
	void SplitEvenOdd(const std::vector<int>& from, std::vector<int>* destEven, std::vector<int>* destOdd)
\end{lstlisting}
\item Функция слияния:
\begin{lstlisting}
	void BatcherMerge(std::vector<int> procsLeft, std::vector<int> procsRight, std::vector<std::pair<int, int>>* comps)
\end{lstlisting}
\item Функция разделения и слияния:
\begin{lstlisting}
	void BatcherSplitNMerge(std::vector<int> procs, std::vector<std::pair<int, int>>* comps)
\end{lstlisting}
\item Функция построения сети:
\begin{lstlisting}
	std::vector<std::pair<int, int>> Batcher(int procCount)
\end{lstlisting}
\end{itemize}

Функия параллельной поразрядной сортировки слиянием Бэтчера:
\begin{lstlisting}
	void Sort(std::vector<int>* data)
\end{lstlisting}
\newpage

% Подтверждение корректности
\begin{center}
\section*{Подтверждение корректности}
\end{center}
\addcontentsline{toc}{section}{Подтверждение корректности}

Для подтверждения корректности в программе представлен набор тестов, разработанных с помощью использования Google C++ Testing Framework.
\par Набор представляет из себя тесты, которые проверяют корректность работы алгоритма(проверки отсортированности массива) при разных условиях: ранг сортировки(2, 10), неподходящее кол-во элементов в массиве(меньше кол-ва процессов, не кратное кол-ву процессов). Также присутствует подсчет времени сортировки.
\par Успешное прохождение всех тестов доказывает корректность работы программного комплекса.
\newpage

% Описание экспериментов
\begin{center}
\section*{Описание экспериментов}
\end{center}
\addcontentsline{toc}{section}{Описание экспериментов}

Эксперементы проводились на оборудовании со следующей аппаратной конфигурацией:
\begin{itemize}
\item Процессор: AMD Ryzen 5 3600X 6-Core Processor, 3,79 ГГц;
\item Оперативная память: 16 ГБ (DDR4), 3200 МГц;
\item ОС: Microsoft Windows 10 Interprise.
\end{itemize}
\begin{table}[h]
    \begin{tabular}{ | p{4cm} | p{4cm} | p{4cm} | p{4cm} | }
    \hline
    Количество процессов & Параллельная \par реализация, сек & Разница с реализацией на 1 процессе\\ \hline
    1    & 0.288  &0 \\ \hline
    2    & 0.148  &-0.14 \\ \hline
    3    & 0.226  &-0.062 \\ \hline
    4    & 0.088  &-0.2 \\ \hline
    5    & 0.069 &-0.219 \\ \hline 
    6    & 0.118 &-0.17 \\ \hline
    7    & 0.107  &-0.181 \\ \hline
    8    & 0.052  &-0.236 \\ \hline
    9    & 0.087  &-0.201\\ \hline
    10   & 0.047  &-0.241 \\ \hline
    11   & 0.076  &-0.212 \\ \hline
    \end{tabular}
\end{table}
\par Сортировка производилась над массивом размером $10^7$, заполненым рандомно.
\par По результатам эксперимента, можно сделать вывод о том, что в большинстве случаев параллельная реализация работает быстрее.

\newpage
% Заключение
\begin{center}
\section*{Заключение}
\end{center}
\addcontentsline{toc}{section}{Заключение}

В ходе данной работы было реализована последовательная поразрядная сортировка и распараллеливание данного алгоритма с помощью четно - нечетного слияния Бэтчера. Были разработаны тесты для подтверждения корректности работы алгоритмов.
\par Также была произведено сравнение производительности данных алгоритмов, результатом которого, является вывод, что параллельный вариант алгоритма работает быстрее.

\newpage
% Список литературы
\begin{thebibliography}{1}
\addcontentsline{toc}{section}{Литература}
\bibitem{Algolist} Поразрядная сортировка [Электронный ресурс] // URL: \url{http://algolist.ru/sort/radix_sort.php/}
\bibitem{Habr} Сеть обменной сортировки со слиянием Бэтчера [Электронный ресурс] // URL: \url{https://habr.com/ru/post/275889/}
\end{thebibliography}

\newpage
% Приложение
\section*{Приложение}
\addcontentsline{toc}{section}{Приложение}
\begin{lstlisting}
// main.cpp
// Copyright 2020 Gavrilov Nikita
#define _SILENCE_TR1_NAMESPACE_DEPRECATION_WARNING

#include <stdio.h>
#include <mpi.h>
#include <gtest/gtest.h>
#include <gtest-mpi-listener.hpp>
#include "BatcherMergingSort.h"
#include <vector>
#include <iostream>
#include <ctime>
#include <random>

bool IsSorted(const std::vector<int>& data) {
    if (data.size() == 0)
        return true;
    for (size_t i = 0; i < data.size()-1; i++) {
        if (data[i] > data[i + 1])
            return false;
    }
    return true;
}
std::vector<int> GetRandomVector(size_t size) {
    std::mt19937 rnd;
    rnd.seed(static_cast<unsigned int>(time(0)));

    std::vector<int> result(size);
    for (size_t i = 0; i < size; i++) {
        result[i] = rnd() % 20000 - 10000;
    }

    return result;
}

TEST(Parallel_Operations_MPI, No_Throw) {
    std::vector<int> data = GetRandomVector(10000);
    ASSERT_NO_THROW(Sort(&data));
}

TEST(Parallel_Operations_MPI, Sorting_Right) {
    std::vector<int> data = GetRandomVector(10000);
    Sort(&data);
    int rank;
    MPI_Comm_rank(MPI_COMM_WORLD, &rank);
    if (rank == 0) {
        ASSERT_TRUE(IsSorted(data));
    }
}

TEST(Parallel_Operations_MPI, Sorting_Time) {
    int rank, size;
    MPI_Comm_rank(MPI_COMM_WORLD, &rank);
    MPI_Comm_size(MPI_COMM_WORLD, &size);
	
    std::vector<int> data = GetRandomVector(1000000);
	
    double t1 = MPI_Wtime();
    Sort(&data);
    if (rank == 0) {
		std::cout<<"time(100000): "<<(MPI_Wtime() - t1)<<std::endl;
    }
}

TEST(Parallel_Operations_MPI, Sort_By_Num_Place_Rang10) {
    std::vector<int> data = GetRandomVector(10000);
    int rank;
    MPI_Comm_rank(MPI_COMM_WORLD, &rank);
    if (rank == 0) {
        ASSERT_NO_THROW(SortByNumPlace(&data, 10));
        ASSERT_TRUE(IsSorted(data));
    }
}

TEST(Parallel_Operations_MPI, Sort_By_Num_Place_Rang2) {
    std::vector<int> data = GetRandomVector(10000);
    int rank;
    MPI_Comm_rank(MPI_COMM_WORLD, &rank);
    if (rank == 0) {
        ASSERT_NO_THROW(SortByNumPlace(&data, 2));
        ASSERT_TRUE(IsSorted(data));
    }
}

TEST(Parallel_Operations_MPI, Sorting_Right_Even_If_Bad_Size) {
    int size;
    MPI_Comm_size(MPI_COMM_WORLD, &size);
    for (int i = 1; i < size; i++) {
        std::vector<int> data = GetRandomVector(size * 1000 + size - i);
        ASSERT_NO_THROW(Sort(&data));
        int rank;
        MPI_Comm_rank(MPI_COMM_WORLD, &rank);
        if (rank == 0) {
            ASSERT_TRUE(IsSorted(data));
        }
    }
}

TEST(Parallel_Operations_MPI, Sorting_Right_Even_If_Bad_Size2) {
    int size;
    MPI_Comm_size(MPI_COMM_WORLD, &size);
    for (int i = 1; i <= size; i++) {
        std::vector<int> data = GetRandomVector(size - 1);
        ASSERT_NO_THROW(Sort(&data));
        int rank;
        MPI_Comm_rank(MPI_COMM_WORLD, &rank);
        if (rank == 0) {
            ASSERT_TRUE(IsSorted(data));
        }
    }
}


int main(int argc, char** argv) {
    ::testing::InitGoogleTest(&argc, argv);
    MPI_Init(&argc, &argv);
    ::testing::AddGlobalTestEnvironment(new GTestMPIListener::MPIEnvironment);
    ::testing::TestEventListeners& listeners =
        ::testing::UnitTest::GetInstance()->listeners();

    listeners.Release(listeners.default_result_printer());
    listeners.Release(listeners.default_xml_generator());

    listeners.Append(new GTestMPIListener::MPIMinimalistPrinter);
    return RUN_ALL_TESTS();
}
\end{lstlisting}
\begin{lstlisting}
// BatcherMergingSort.h
// Copyright 2020 Gavrilov Nikita
#ifndef MODULES_TASK_3_GAVRILOV_N_BATCHER_MERGING_SORT_BATCHERMERGINGSORT_H_
#define MODULES_TASK_3_GAVRILOV_N_BATCHER_MERGING_SORT_BATCHERMERGINGSORT_H_

#include <stdint.h>
#include <utility>
#include <vector>

int8_t GetDigit(int value, uint8_t digidNum, uint8_t rang);
uint8_t GetNumOfDigits(int value, uint8_t rang);
void ConcatVectors(const std::vector<std::vector<int>>& from, std::vector<int>* dest);
void SortByDigid(std::vector<int>* data, uint8_t digitNum, uint8_t rang = 10);
void SortByNumPlace(std::vector<int>* data, uint8_t rang = 10);

void SplitEvenOdd(const std::vector<int>& from, std::vector<int>* destEven, std::vector<int>* destOdd);
void BatcherMerge(std::vector<int> procsLeft, std::vector<int> procsRight, std::vector<std::pair<int, int>>* comps);
void BatcherSplitNMerge(std::vector<int> procs, std::vector<std::pair<int, int>>* comps);
std::vector<std::pair<int, int>> Batcher(int procCount);

void Sort(std::vector<int>* data);

#endif  // MODULES_TASK_3_GAVRILOV_N_BATCHER_MERGING_SORT_BATCHERMERGINGSORT_H_
\end{lstlisting}
\begin{lstlisting}
// BatcherMergingSort.cpp
// Copyright 2020 Gavrilov Nikita
#include <mpi.h>
#include <utility>
#include <algorithm>
#include <vector>
#include <cmath>
#include "../../modules/task_3/gavrilov_n_batcher_merging_sort/BatcherMergingSort.h"

#define MY_INT_MAX 2147483647

int8_t GetDigit(int value, uint8_t digidNum, uint8_t rang) {
    if (rang <= 1)
        throw "rang must be more than 1";

    int powRang = static_cast<int>(pow(rang, digidNum));
    int powRangNext = static_cast<int>(pow(rang, digidNum + 1.0));

    return value % powRangNext / powRang;
}
uint8_t GetNumOfDigits(int value, uint8_t rang) {
    if (rang <= 1)
        throw "rang must be more than 1";

    int length = 1;
    while (value /= rang)
        length++;

    return length;
}
void ConcatVectors(const std::vector<std::vector<int>>& from, std::vector<int>* dest) {
    for (size_t i = 0; i < from.size(); i++) {
        for (size_t j = 0; j < from[i].size(); j++) {
            dest->push_back(from[i][j]);
        }
    }
}
void SortByDigid(std::vector<int>* data, uint8_t digitNum, uint8_t rang) {
    if (rang <= 1)
        throw "rang must be more than 1";
    std::vector<std::vector<int>> vectorsByDigits(rang * 2 - 1);

    for (size_t i = 0; i < data->size(); i++) {
        int8_t digit = GetDigit((*data)[i], digitNum, rang);
        vectorsByDigits[digit + rang - 1].push_back((*data)[i]);
    }

    data->clear();
    ConcatVectors(vectorsByDigits, data);
}
void SortByNumPlace(std::vector<int>* data, uint8_t rang) {
    if (rang <= 1)
        throw "rang must be more than 1";

    std::vector<std::vector<int>> vectorsByDigits(rang * 2 - 1);
    uint8_t maxNumOfDigits = 0;

    // it is the same with SortByDigid, but we need to count maxNumOfDigits,
    // we can do this because number has 1 digit as minimum
    // bad part of this - we write same code
    // it makes sort faster by data.size() operations
    for (size_t i = 0; i < data->size(); i++) {
        int8_t digit = GetDigit((*data)[i], 0, rang);
        vectorsByDigits[digit + rang - 1].push_back((*data)[i]);

        uint8_t numOfDigits = GetNumOfDigits((*data)[i], rang);
        maxNumOfDigits = std::max(maxNumOfDigits, numOfDigits);
    }

    data->clear();
    ConcatVectors(vectorsByDigits, data);

    for (uint8_t i = 1; i < maxNumOfDigits; i++) {
        SortByDigid(data, i, rang);
    }
}

void SplitEvenOdd(const std::vector<int>& from, std::vector<int>* destEven, std::vector<int>* destOdd) {
    for (size_t i = 0; i < from.size() / 2; i++) {
        destEven->push_back(from[i * 2]);
        destOdd->push_back(from[i * 2 + 1]);
    }
    if (from.size() % 2 == 1) {
        destEven->push_back(from.back());
    }
}
void BatcherMerge(std::vector<int> procsLeft, std::vector<int> procsRight, std::vector<std::pair<int, int>>* comps) {
    int procsCount = procsLeft.size() + procsRight.size();
    if (procsCount <= 1) {
        return;
    } else if (procsCount == 2) {
        comps->push_back(std::pair<int, int>(procsLeft[0], procsRight[0]));
        return;
    }

    std::vector<int> procsLeftOdd, procsLeftEven;
    std::vector<int> procsRightOdd, procsRightEven;
    std::vector<int> procsResult;

    SplitEvenOdd(procsLeft, &procsLeftEven, &procsLeftOdd);
    SplitEvenOdd(procsRight, &procsRightEven, &procsRightOdd);

    BatcherMerge(procsLeftOdd, procsRightOdd, comps);
    BatcherMerge(procsLeftEven, procsRightEven, comps);

    std::vector<std::vector<int>> vecs{ procsLeft, procsRight };
    ConcatVectors(vecs, &procsResult);

    for (size_t i = 1; i + 1 < procsResult.size(); i += 2) {
        comps->push_back(std::pair<int, int>(procsResult[i], procsResult[i + 1]));
    }
}
void BatcherSplitNMerge(std::vector<int> procs, std::vector<std::pair<int, int>>* comps) {
    if (procs.size() <= 1) {
        return;
    }

    std::vector<int> procs_up(procs.begin(), procs.begin() + procs.size() / 2);
    std::vector<int> procs_down(procs.begin() + procs.size() / 2, procs.end());

    BatcherSplitNMerge(procs_up, comps);
    BatcherSplitNMerge(procs_down, comps);
    BatcherMerge(procs_up, procs_down, comps);
}
std::vector<std::pair<int, int>> Batcher(int procCount) {
    std::vector<int> procs(procCount);
    for (int i = 0; i < procCount; i++) {
        procs[i] = i;
    }

    std::vector<std::pair<int, int>> comps;
    BatcherSplitNMerge(procs, &comps);

    return comps;
}

void Sort(std::vector<int>* data) {
    size_t realSize = data->size();

    int size, rank;
    MPI_Comm_size(MPI_COMM_WORLD, &size);
    MPI_Comm_rank(MPI_COMM_WORLD, &rank);

    int maxSize = std::min(size, static_cast<int>(data->size()));
    // calculating comparers
    std::vector<std::pair<int, int>> comps = Batcher(maxSize);

    // calculating how many of data will owns each proc
    int countPerProc;
    if (rank == 0) {
        while (data->size() % size != 0) {
            data->push_back(MY_INT_MAX);
        }
        countPerProc = data->size() / size;
    }
    MPI_Bcast(&countPerProc, 1, MPI_INT, 0, MPI_COMM_WORLD);

    // splitting data by procs
    std::vector<int> localData;
    if (rank == 0) {
        for (int i = 1; i < maxSize; i++) {
            MPI_Send(data->data() + (countPerProc * i), countPerProc, MPI_INT, i, 1, MPI_COMM_WORLD);
        }
        localData = std::vector<int>(data->begin(), data->begin() + countPerProc);
    } else if (rank < maxSize) {
        localData = std::vector<int>(countPerProc);
        MPI_Status st;
        MPI_Recv(localData.data(), countPerProc, MPI_INT, 0, 1, MPI_COMM_WORLD, &st);
    }

    // sort each proc's data
    SortByNumPlace(&localData);

    if (size > 1) {
        // making all comparers
        for (size_t i = 0; i < comps.size(); i++) {
            if (rank == comps[i].first) {
                MPI_Send(localData.data(), countPerProc, MPI_INT, comps[i].second, 2, MPI_COMM_WORLD);
                MPI_Status st;
                MPI_Recv(localData.data(), countPerProc, MPI_INT, comps[i].second, 3, MPI_COMM_WORLD, &st);
            } else if (rank == comps[i].second) {
                MPI_Status st;

                std::vector<int> recvData(countPerProc);
                MPI_Recv(recvData.data(), countPerProc, MPI_INT, comps[i].first, 2, MPI_COMM_WORLD, &st);

                std::vector<int> tmp(countPerProc * 2);

                for (int j = 0, k = 0, l = 0; l < countPerProc * 2; l++) {
                    if (j < countPerProc && (k >= countPerProc || localData[j] < recvData[k]))
                        tmp[l] = localData[j++];
                    else
                        tmp[l] = recvData[k++];
                }
                MPI_Send(tmp.data(), countPerProc, MPI_INT, comps[i].first, 3, MPI_COMM_WORLD);
                for (int j = 0; j < countPerProc; j++) {
                    localData[j] = tmp[j + countPerProc];
                }
            }
        }

        // recieving data from all procs to main one
        if (rank == 0) {
            for (int i = 1; i < maxSize; i++) {
                MPI_Status st;
                MPI_Recv(data->data() + (countPerProc * i), countPerProc, MPI_INT, i, 4, MPI_COMM_WORLD, &st);
            }
        } else if (rank < maxSize) {
            MPI_Send(localData.data(), countPerProc, MPI_INT, 0, 4, MPI_COMM_WORLD);
        }
    }

    if (rank == 0) {
        // copying data from main to itself
        std::copy(localData.begin(), localData.end(), data->begin());

        // erasing added vars;
        data->erase(data->begin() + realSize, data->end());
    }
}
\end{lstlisting}
\end{document}

