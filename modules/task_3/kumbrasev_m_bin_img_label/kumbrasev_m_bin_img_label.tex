\documentclass{report}

\usepackage[T2A]{fontenc}
\usepackage[utf8]{luainputenc}
\usepackage[english, russian]{babel}
\usepackage[pdftex]{hyperref}
\usepackage[14pt]{extsizes}
\usepackage{listings}
\usepackage{color}
\usepackage{geometry}
\usepackage{enumitem}
\usepackage{multirow}
\usepackage{graphicx}
\usepackage{indentfirst}
\renewcommand{\labelenumii}{\arabic{enumi}.\arabic{enumii}.}

\geometry{a4paper,top=2cm,bottom=3cm,left=2cm,right=1.5cm}
\setlength{\parskip}{0.5cm}
\setlist{nolistsep, itemsep=0.3cm,parsep=0pt}

\lstset{language=C++,
		basicstyle=\footnotesize,
		keywordstyle=\color{blue}\ttfamily,
		stringstyle=\color{red}\ttfamily,
		commentstyle=\color{green}\ttfamily,
		morecomment=[l][\color{magenta}]{\#}, 
		tabsize=4,
		breaklines=true,
  		breakatwhitespace=true,
  		title=\lstname,       
}

\makeatletter
\renewcommand\@biblabel[1]{#1.\hfil}
\makeatother

\begin{document}

\begin{titlepage}

\begin{center}
Министерство науки и высшего образования Российской Федерации
\end{center}

\begin{center}
Федеральное государственное автономное образовательное учреждение высшего образования \\
Национальный исследовательский Нижегородский государственный университет им. Н.И. Лобачевского
\end{center}

\begin{center}
Институт информационных технологий, математики и механики
\end{center}

\vspace{4em}

\begin{center}
\textbf{\LargeОтчет по лабораторной работе} \\
\end{center}
\begin{center}
\textbf{\Large «Маркировка компонент на бинарном изображении.»}
\vspace{4em}
\end{center}
\newbox{\lbox}
\savebox{\lbox}{\hbox{text}}
\newlength{\maxl}
\setlength{\maxl}{\wd\lbox}
\hfill\parbox{7cm}{
\hspace*{5cm}\hspace*{-5cm}\textbf{Выполнил:} \\ студент группы 381806-4 \\ Кумбрасьев М. Э.\\
\\
\hspace*{5cm}\hspace*{-5cm}\textbf{Проверил:}\\ доцент кафедры МОСТ, \\ кандидат технических наук \\ Сысоев А. В.\\
}
\vspace{\fill}

\begin{center} Нижний Новгород \\ 2020 \end{center}

\end{titlepage}

\setcounter{page}{2}

% Содержание
\tableofcontents
\newpage

% Введение
\section*{Введение}
\addcontentsline{toc}{section}{Введение}
Бинарное  изображение  (двухуровневое,  двоичное)  —  разновидность  цифровых растровых изображений, когда каждый пиксел может представлять только один из двух цветов.
\par Значения  каждого  пиксела  условно  кодируются  как  «0»  и  «1».  Значение  «0»  условно называют  задним  планом  или  фоном  (англ.  background),  а  «1»  —  передним  планом  (англ.foreground).
\par В данной программе бинарное изображение будет представлено в виде отдельного класса, поля  которого  содержат  ширину,  высоту  изображения  и  двумерный  целочисленный  массив, необходимый для хранения данных изображения.
\par Задача  маркировки  компонент  на  бинарном  изображении  предполагает  выделение связных объектов и последующую маркировку уникальным индексом (цветом). Таким образом, результатом работы программы, получившей на вход бинарное изображение, будет размеченное изображение (матрица, элементами которой являются «0» и индексы).
\newpage

% Постановка задачи
\section*{Постановка задачи}
\addcontentsline{toc}{section}{Постановка задачи}
В данной лабораторной работе необходимо выполнить следующие задачи.
\begin{enumerate} 
\item Разработать алгоритм  параллельной  маркировки бинарного изображения  некоторым 
числом процессов.
\item Реализовать класс, представляющий бинарное изображение.
\item Написать программу, реализующую данный алгоритм.
\item Написать тесты для проверки работы алгоритма.
\item Провести вычислительные эксперименты.
\item Проанализировать полученные результаты.
\end{enumerate}
\newpage

% Описание алгоритма
\section*{Описание алгоритма}
\addcontentsline{toc}{section}{Описание алгоритма}
Алгоритм маркировки:
\begin{enumerate}
\item Если индекс условного пикселя (элемента матрицы) равен «1», маркируем его, то есть присваиваем уникальный индекс.
\item Если  индекс  пикселя, расположенного  под найденным, равен «1» маркируем его  тем же индексом, двигаемся вниз, пока не найдём  пиксель с индексом «0».
\item При нахождении каждого пикселя ниже предыдущего, проходим по пикселям слева и справа от него, пока не найдём «0». Все найденные пиксели маркируем. 
\item Если все пиксели вокруг первоначального исследованы, увеличиваем счетчик фигур на 1.
\end{enumerate}
\newpage

% Схема распараллеливания
\section*{Схема распараллеливания}
\addcontentsline{toc}{section}{Схема распараллеливания}
Принцип распараллеливания в данной программе заключается в том, что бинарное изображение  разбивается  на  некоторое  число  областей,  зависящее  от  числа  процессов, затем  эти  области  передаются  процессам  для  маркировки,  после  чего,  размеченные области возвращаются корневому процессу. Он проходит по границам областей, размечая фигуры,  расположенные  на  границах,  одним  из  индексов.  Здесь  появляется  проблема, 
заключающаяся  в  том,  что  число  процессов  не  всегда  пропорционально  размеру изображения,  так  что  нужно  вычислить  остаток  и  увеличить  область,  обрабатываемую нулевым процессом, чтобы всё изображение было промаркировано. Рассмотрим алгоритм более подробно:
\begin{enumerate}
\item Бинарное изображение, представляющее собой матрицу, трансформируется в строку нулевым процессом.
\item Строка разбивается на подстроки некоторой длины, подстроки передаются нулевым процессом всем остальным процессам. 
\item Процесс, получив подстроку, преобразует её в локальное бинарное изображение.
\item Каждый процесс проводит маркировку своего изображения.
\item Затем преобразует изображение назад в подстроку.
\item Передает сформированную подстроку нулевому процессу. 
\item Нулевой  процесс  формирует  из  полученных  подстрок  строку  и  строит  новое изображение.
\item Затем проходит по всем границам областей, если находит связные фигуры на границе разных областей, маркирует их одним из индексов.
\end{enumerate}
\newpage

% Описание программной реализации
\section*{Описание программной реализации}
\addcontentsline{toc}{section}{Описание программной реализации}
Для решения поставленной задачи используется основные функции:
\begin{enumerate}
\item struct image – класс  «image»  с полями  height (высота),  width (ширина) и  data,  а также перегруженным оператором сдвига, представляющий бинарное изображение;
\item void labeling(image* img) – алгоритм  «labelling»,  предназначенный  для параллельной маркировки бинарного изображения некоторым числом процессов;
\end{enumerate}
\newpage

% Подтверждение корректности
\section*{Подтверждение корректности}
\addcontentsline{toc}{section}{Подтверждение корректности}
Для  проверки  корректности  результата  работы  алгоритма  сравним  значения заданного  количества  фигур  и  полученное  значение  счетчика.  Таким  образом,  можно сделать  вывод,  что  алгоритм  нашел  и  промаркировал  все  заданные  фигуры  на изображении. Для наглядности реализован небольшой алгоритм, выводящий полученное изображение на экран, так что наблюдатель может убедиться, что изображение размечено 
правильно. 
\par В программе реализованы тесты на основе Google tests для проверки корректности работы  алгоритма.  Для  каждого  теста  задано  некоторое  бинарное  изображение. 
\par Успешное прохождение всех тестов доказывает корректность работы всей программы.
\newpage

% Результаты экспериментов
\section*{Результаты экспериментов}
\addcontentsline{toc}{section}{Результаты экспериментов}
Эксперименты проводились на ПК со следующими характеристиками:
\begin{enumerate}
\item Процессор: AMD FX(tm)-6300 Six-Core Processor, 3.52 GHz, ядер 6;
\item Оперативная память: 16 ГБ (DDR4), 2.1 GHz;
\item ОС: Microsoft Windows 10 
\end{enumerate}

\par Задано  случайное  бинарное  изображение.  Проведем  замеры  времени  работы алгоритма на различном числе процессо в и получим среднее. 
\par Результаты экспериментов представлены в таблице

\begin{table}[!h]
\caption{Результаты вычислительных экспериментов}
\centering
\begin{tabular}{p{3cm} p{4cm} p{4cm} p{4cm}}
Число процессов/сторона изображения & 2 & 4 & 6 \\
1024     & 0,017335 с &  0,016961 с & 0,019742 с\\
2048     & 0,081043 с & 0,071152 с & 0,064605 с  \\
4096     & 0,356986 с & 0,296069 с & 0,264022 с \\
  

\end{tabular}
\end{table}

\par При  небольшом  размере  изображения  1024  *  1024  наблюдается  прирост производительности в 1,562 раза при выполнении на 2 процессах, в 1,596 при выполнении на 4. Выполнение на 6 процессах дает прирост в 1,371 раз. Дальнейшее увеличение числа процессов  не  дает  прироста  производительности,  напротив,  увеличивает  время выполнения алгоритма. Это объясняется большим возрастанием накладных расходов на разбиение  изображения  на  области  и  их  рассылку  всем  процессам.   Таким  образом, оптимальным числом процессов для этого алгоритма на малом изображении является 4 процесса.
\par При  среднем  размере  изображения  2048  *  2048  наблюдается  прирост производительности в 1,393 раза при выполнении на 2 процессах, в 1,586 при выполнении на 4. Выполнение на 6 процессах дает прирост в 1,748 раз. Дальнейшее увеличение числа процессов не дает прироста производительности. Таким образом, оптимальным числом процессов для этого алгоритма на среднем изображении является 6 процессов.
\par При  большом  размере  изображения  4096  *  4096  наблюдается  прирост производительности в 1,435 раза при выполнении на 2 процессах, в 1,731  при выполнении на 4. Выполнение на 6 процессах дает прирост в 1,942 раз. Дальнейшее увеличение числа процессов не дает прироста производительности. Таким образом, оптимальным числом процессов для этого алгоритма на малом изображении является 6 процессов.
\par Таким образом, в среднем выполнение на 2 процессах дает ускорение в 1,528 раз, на  4  в  1,637  раз,  а  на  6 в  1,687  раз.  Дальнейшее  увеличение  числа  процессов  не  имеет смысла, так как сильно возрастают накладные расходы, замедляющие работу алгоритма. Можно  сделать  вывод,  что  для  небольших  изображений  оптимальнее использовать  4 процесса, для остальных – 6.
\newpage

% Заключение
\section*{Заключение}
\addcontentsline{toc}{section}{Заключение}
В  результате  данной  лабораторной  работы  успешно  выполнены  следующие действия:
\begin{enumerate}

\item Написана программа, успешно реализующая алгоритм маркировки на практике. 
\item Разработаны  и  реализованы  Google-тесты  для  проверки  корректности работы алгоритма, результаты их работы приведены выше.
\item Проведены  вычислительные  эксперименты,  получено  среднее  время выполнения  алгоритма  на  разном  числе  процессов  и  размерах  исходного изображения, полученные результаты занесены в таблицу.
\item Сделаны следующие выводы на основе полученных результатов:
\begin{itemize}
\item для малых изображений (со сторонами до 1024) эффективнее задействовать 4 процесса при выполнении алгоритма;
\item для всех остальных оптимальным вариантом является 8 процессов. 
\end{itemize}
\end{enumerate}
Таким образом, все поставленные в данной лабораторной работе задачи успешно выполнены.   
\newpage

% Список литературы
\begin{thebibliography}{1}
\addcontentsline{toc}{section}{Список литературы}
\bibitem{Gergel}
Гергель  В.П.,  Стронгин  Р.Г.  Основы  параллельных  вычислений  для многопроцессорных вычислительных систем. Учебное пособие  –  Нижний Новгород: Изд-во ННГУ им. Н.И. Лобачевского, 2003. 184 с.
\bibitem{Wikipedia}
Википедия:  свободная  электронная  энциклопедия:  на  русском  языке  [Электронный ресурс]  -  Режим  доступа: свободный.
\end{thebibliography}
\newpage

% Приложение
\section*{Приложение}
\addcontentsline{toc}{section}{Приложение}
В данном разделе находится листинг всего кода, написанного в рамках лабораторной работы.
\begin{lstlisting}
// bin_img_labeling.h

// Copyright 2020 Kumbrasev Mark
#ifndef MODULES_TASK_3_KUMBRASEV_M_BIN_IMG_LABEL_BIN_IMG_LABELING_H_
#define MODULES_TASK_3_KUMBRASEV_M_BIN_IMG_LABEL_BIN_IMG_LABELING_H_

#include <mpi.h>
#include <iostream>

struct image {
  int height, width, count;
  int** data;

  image(int _height, int _width) : height(_height), width(_width), count(1) {
    data = new int*[height];
    for (int i = 0; i < height; i++) {
      data[i] = new int[width];
      for (int j = 0; j < width; j++) {
        data[i][j] = 0;
      }
    }
  }

  friend std::ostream& operator<< (std::ostream& os, const image& img) {
    for (int i = 0; i < img.height; i++) {
      for (int j = 0; j < img.width; j++) {
        os << img.data[i][j] << "\t";
      }
      os << std::endl;
    }
    return os;
  }
};

void labeling(image* img);

#endif  // MODULES_TASK_3_KUMBRASEV_M_BIN_IMG_LABEL_BIN_IMG_LABELING_H_

\end{lstlisting}
\begin{lstlisting}
// bin_img_labeling.cpp

// Copyright 2020 Kumbrasev Mark
#include <mpi.h>
#include <iostream>
#include "../../../modules/task_3/kumbrasev_m_bin_img_label/bin_img_labeling.h"

void labeling(image* img) {
  int rank, size;
  int c_str, new_c_str, label, c_label = 0, res_exp = 0, rest;
  bool flag = 0;
  int* img_arr, *loc_img_arr;
  image* loc_img;
  MPI_Comm_rank(MPI_COMM_WORLD, &rank);
  MPI_Comm_size(MPI_COMM_WORLD, &size);
  MPI_Status status;
  c_str = img->height / size;
  rest = img->height % size;

  if (rank == 0) {
    new_c_str = c_str + rest;
  } else {
    new_c_str = c_str;
  }

  label = rank * 100 + 2;
  img_arr = new int[img->height * img->width];
  loc_img_arr = new int[new_c_str * img->width];
  loc_img = new image(new_c_str, img->width);

  if (rank == 0) {
    for (int i = 0; i < img->height; i++) {
      for (int j = 0; j < img->width; j++) {
        img_arr[i * img->width + j] = img->data[i][j];
      }
    }

    for (int i = 0; i < (new_c_str) * img->width; i++) {
      loc_img_arr[i] = img_arr[i];
    }

    for (int i = 1; i < size; i++) {
      MPI_Send(&img_arr[i * img->width * c_str + rest * img->width], c_str * img->width, MPI_INT, i, 0, MPI_COMM_WORLD);
    }
  } else {
    MPI_Recv(&loc_img_arr[0], c_str * img->width, MPI_INT, 0, 0, MPI_COMM_WORLD, &status);
    }

  for (int i = 0; i < new_c_str; i++) {
    for (int j = 0; j < img->width; j++) {
      loc_img->data[i][j] = loc_img_arr[i * img->width + j];
    }
  }

  for (int i = 0; i < loc_img->height; i++) {
    for (int j = 0; j < img->width; j++) {
      if (loc_img->data[i][j] == 1) {
        flag = 1;
        loc_img->data[i][j] = label;
        if (i + 1 < new_c_str && loc_img->data[i + 1][j] == 1) {
          int k = i + 1;
          while (k < new_c_str && loc_img->data[k][j] == 1) {
            loc_img->data[k][j] = label;
            if (j - 1 >= 0 && loc_img->data[k][j - 1]) {
              int l = j - 1;
              while (l >= 0 && loc_img->data[k][l]) {
                loc_img->data[k][l] = label;
                l--;
              }
            }
            if (j + 1 < img->width && loc_img->data[k][j + 1] == 1) {
              int l = j + 1;
              while (l < img->width && loc_img->data[k][l]) {
                loc_img->data[k][l] = label;
                l++;
              }
            }
            k++;
          }
        }
      } else {
        if (flag) {
          flag = 0;
          label++;
          c_label++;
        }
      }
    }
  }

  for (int i = 0; i < new_c_str; i++) {
    for (int j = 0; j < img->width; j++) {
      loc_img_arr[i * img->width + j] = loc_img->data[i][j];
    }
  }

  if (rank == 0) {
    for (int i = 0; i < new_c_str * img->width; i++) {
      img_arr[i] = loc_img_arr[i];
    }
  }

  if (rank != 0) {
      MPI_Send(&loc_img_arr[0], new_c_str * img->width, MPI_INT, 0, 0, MPI_COMM_WORLD);
  } else {
    for (int i = 1; i < size; i++) {
      MPI_Recv(&img_arr[i * img->width * c_str + rest * img->width],
      new_c_str * img->width, MPI_INT, i, 0, MPI_COMM_WORLD, &status);
    }
  }

  MPI_Reduce(&c_label, &res_exp, 1, MPI_INT, MPI_SUM, 0, MPI_COMM_WORLD);

  if (rank == 0) {
    for (int i = 0; i < img->height; i++) {
      for (int j = 0; j < img->width; j++) {
        img->data[i][j] = img_arr[i * img->width + j];
      }
    }

    flag = 0;
    for (int i = new_c_str - 1; i < img->height - c_str; i += c_str) {
      for (int j = 0; j < img->width; j++) {
        if (img->data[i][j] != 0 && img->data[i + 1][j] != 0) {
          flag = 1;
          int k = i + 1;
          while (k < img->height && img->data[k][j] != 0) {
            img->data[k][j] = img->data[i][j];
            if (j > 0 && img->data[k][j - 1] != 0) {
              int l = j - 1;
              while (l >= 0 && img->data[k][l] != 0) {
                img->data[k][l] = img->data[i][j];
                l--;
              }
            }

            if (j < img->width - 1 && img->data[k][j + 1] != 0) {
              int l = j + 1;
              while (l < img->width && img->data[k][l] != 0) {
                img->data[k][l] = img->data[i][j];
                img->data[k][l] = img->data[i][j];
                l++;
              }
            }
            k++;
          }
        } else {
          if (flag) {
            flag = 0;
            res_exp--;
          }
        }
      }
    }
  }
  img->count = res_exp;
  delete loc_img;
}

\end{lstlisting}
\begin{lstlisting}
// main.cpp
// Copyright 2020 Kumbrasev Mark
#include <gtest-mpi-listener.hpp>
#include <gtest/gtest.h>
#include <time.h>
#include <iostream>
#include "./bin_img_labeling.h"

TEST(label, test1) {
  int rank, size, exp_count = 2;
  double start_time, end_time;
  MPI_Comm_rank(MPI_COMM_WORLD, &rank);
  MPI_Comm_size(MPI_COMM_WORLD, &size);

  image img(4, 4);

  if (rank == 0) {
    img.data[0][0] = 1;
    img.data[1][0] = 1;
    img.data[0][3] = 1;
    img.data[0][2] = 1;
    img.data[1][3] = 1;
    img.data[3][1] = 1;
    img.data[3][2] = 1;
    img.data[3][3] = 1;
  }

  start_time = MPI_Wtime();
  labeling(&img);
  end_time = MPI_Wtime();

  printf("\tTime  = %f\n", end_time - start_time);

  if (rank == 0) {
    ASSERT_EQ(exp_count, 2);
  }
}

TEST(label, test2) {
  int rank, size, exp_count = 3;
  double start_time, end_time;
  MPI_Comm_rank(MPI_COMM_WORLD, &rank);
  MPI_Comm_size(MPI_COMM_WORLD, &size);

  image img(6, 6);

  if (rank == 0) {
    img.data[0][0] = 1;
    img.data[1][0] = 1;
    img.data[0][1] = 1;
    img.data[1][1] = 1;
    img.data[0][4] = 1;
    img.data[0][5] = 1;
    img.data[1][4] = 1;
    img.data[4][3] = 1;
    img.data[4][2] = 1;
    img.data[5][5] = 1;
  }
  start_time = MPI_Wtime();
  labeling(&img);
  end_time = MPI_Wtime();

  printf("\tTime  = %f\n", end_time - start_time);

  if (rank == 0) {
    ASSERT_EQ(exp_count, 3);
  }
}

TEST(label, test3) {
  int rank, size, exp_count = 1;
  double start_time, end_time;
  MPI_Comm_rank(MPI_COMM_WORLD, &rank);
  MPI_Comm_size(MPI_COMM_WORLD, &size);
  image img(4, 4);

  if (rank == 0) {
    img.data[0][0] = 1;
    img.data[1][0] = 1;
    img.data[2][0] = 1;
    img.data[3][0] = 1;
    img.data[3][2] = 1;
    img.data[3][3] = 1;
    img.data[1][1] = 1;
    img.data[1][2] = 1;
  }

  start_time = MPI_Wtime();
  labeling(&img);
  end_time = MPI_Wtime();

  printf("\tTime  = %f\n", end_time - start_time);

  if (rank == 0) {
    ASSERT_EQ(exp_count, 1);
  }
}

TEST(label, test4) {
  int rank, size, exp_count = 8;
  double start_time, end_time;
  MPI_Comm_rank(MPI_COMM_WORLD, &rank);
  MPI_Comm_size(MPI_COMM_WORLD, &size);

  image img(6, 6);

  if (rank == 0) {
    img.data[0][0] = 1;
    img.data[1][1] = 1;
    img.data[2][2] = 1;
    img.data[3][3] = 1;
    img.data[4][4] = 1;
    img.data[5][5] = 1;
    img.data[4][1] = 1;
    img.data[0][5] = 1;
    img.data[1][4] = 1;
    img.data[2][3] = 1;
    img.data[3][2] = 1;
    img.data[5][0] = 1;
  }
  start_time = MPI_Wtime();
  labeling(&img);
  end_time = MPI_Wtime();

  printf("\tTime  = %f\n", end_time - start_time);

  if (rank == 0) {
    ASSERT_EQ(exp_count, 8);
  }
}

TEST(label, test5) {
  int rank, size, exp_count = 2;
  double start_time, end_time;
  MPI_Comm_rank(MPI_COMM_WORLD, &rank);
  MPI_Comm_size(MPI_COMM_WORLD, &size);

  image img(6, 6);

  if (rank == 0) {
    img.data[0][0] = 1;
    img.data[1][0] = 1;
    img.data[2][0] = 1;
    img.data[3][0] = 1;
    img.data[4][0] = 1;
    img.data[5][0] = 1;
    img.data[0][1] = 1;
    img.data[0][2] = 1;
    img.data[0][3] = 1;
    img.data[0][4] = 1;
    img.data[0][5] = 1;
    img.data[4][2] = 1;
    img.data[4][3] = 1;
    img.data[4][4] = 1;
    img.data[3][4] = 1;
    img.data[2][4] = 1;
  }
  start_time = MPI_Wtime();
  labeling(&img);
  end_time = MPI_Wtime();

  printf("\tTime  = %f\n", end_time - start_time);

  if (rank == 0) {
    ASSERT_EQ(exp_count, 2);
  }
}

int main(int argc, char** argv) {
  ::testing::InitGoogleTest(&argc, argv);
  MPI_Init(&argc, &argv);

  ::testing::AddGlobalTestEnvironment(new GTestMPIListener::MPIEnvironment);
  ::testing::TestEventListeners& listeners =
    ::testing::UnitTest::GetInstance()->listeners();

  listeners.Release(listeners.default_result_printer());
  listeners.Release(listeners.default_xml_generator());

  listeners.Append(new GTestMPIListener::MPIMinimalistPrinter);
  return RUN_ALL_TESTS();
}

\end{lstlisting}

\end{document}
