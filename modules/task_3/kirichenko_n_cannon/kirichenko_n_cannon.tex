\documentclass{report}

\usepackage[T2A]{fontenc}
\usepackage[utf8]{luainputenc}
\usepackage[english, russian]{babel}
\usepackage[pdftex]{hyperref}
\usepackage[14pt]{extsizes}
\usepackage{listings}
\usepackage{color}
\usepackage{geometry}
\usepackage{enumitem}
\usepackage{multirow}
\usepackage{graphicx}
\usepackage{indentfirst}

\geometry{a4paper,top=2cm,bottom=3cm,left=2cm,right=1.5cm}
\setlength{\parskip}{0.5cm}
\setlist{nolistsep, itemsep=0.3cm,parsep=0pt}

\lstset{language=C++,
		basicstyle=\footnotesize,
		keywordstyle=\color{blue}\ttfamily,
		stringstyle=\color{red}\ttfamily,
		commentstyle=\color{green}\ttfamily,
		morecomment=[l][\color{magenta}]{\#}, 
		tabsize=4,
		breaklines=true,
  		breakatwhitespace=true,
  		title=\lstname,       
}

\makeatletter
\renewcommand\@biblabel[1]{#1.\hfil}
\makeatother

\begin{document}

\begin{titlepage}

\begin{center}
Министерство науки и высшего образования Российской Федерации
\end{center}

\begin{center}
Федеральное государственное автономное образовательное учреждение высшего образования \\
Национальный исследовательский Нижегородский государственный университет им. Н.И. Лобачевского
\end{center}

\begin{center}
Институт информационных технологий, математики и механики
\end{center}

\vspace{4em}

\begin{center}
\textbf{\LargeОтчет по лабораторной работе} \\
\end{center}
\begin{center}
\textbf{\Large «Умножение плотных матриц. Элементы типа double. Блочная схема, алгоритм Кэннона.»}
\vspace{4em}
\end{center}
\newbox{\lbox}
\savebox{\lbox}{\hbox{text}}
\newlength{\maxl}
\setlength{\maxl}{\wd\lbox}
\hfill\parbox{7cm}{
\hspace*{5cm}\hspace*{-5cm}\textbf{Выполнил:} \\ студент группы 381806-3 \\ Кириченко Н. А.\\
\\
\hspace*{5cm}\hspace*{-5cm}\textbf{Проверил:}\\ доцент кафедры МОСТ, \\ кандидат технических наук \\ Сысоев А. В.\\
}
\vspace{\fill}

\begin{center} Нижний Новгород \\ 2020 \end{center}

\end{titlepage}

\setcounter{page}{2}

% Содержание
\tableofcontents
\newpage

% Введение
\section*{Введение}
\addcontentsline{toc}{section}{Введение}
В современном мире многие вычисления уже не производятся вручную – для этого существует специальная вычислительная техника, ведь проводимые эксперименты становятся сложнее: объём выборки и трудоёмкость алгоритмов уже не позволяют производить расчёты на бумаге – такой объём работы может оказаться просто не под силу даже группе людей, к тому же в таком случае нельзя гарантировать правильность подсчётов наверняка, так как имеет место быть человеческих фактор. Для больших объёмов вычислений уже давно используют специальную технику и имеющиеся алгоритмы, однако даже при использовании производящих подсчёты
устройств может пройти достаточно много времени до получения результата. А если это не научный эксперимент, результаты которого в дальнейшем станут почвой для размышлений больших умов, а программа, работающая в реальном времени, то вопрос скорости получения ответа встаёт более остро. Здесь нам на помощь приходит понятие оптимизации.
\par Оптимизация — процесс максимизации выгодных характеристик, соотношений (например, оптимизация производственных процессов и производства), и минимизации расходов.
\par В нашем случае речь идёт о минимизации времени вычислений путём распараллеливания программы на некоторое количество процессов, так как время в контексте поставленной задачи – самый ценный ресурс.
\par Реализация блочного алгоритма Кэннона для умножения плотных матриц основывается на
распараллеливании программы, что позволяет сократить время работы алгоритма, в отличие от
стандартного алгоритма умножения плотных матриц.
\par В данной лабораторной работе будет рассмотрено умножение плотных матриц по алгоритму Кэннона.
\newpage

% Постановка задачи
\section*{Постановка задачи}
\addcontentsline{toc}{section}{Постановка задачи}
Одной из основных операций над матрицами является их перемножение. Эта операция используется практически везде: от компьютерной графики до расчета поверхностей самолетов.
\par Цель данной работы - реализация параллельного и последовательного алгоритмов, вычисляющих произведение квадратных матриц nxn. В результате работы программы, мы должны получить еще одну матрицу порядка nxn, являющуюся произведение двух исходных, а так же время работы каждого алгоритма.
\par Требуется реализовать последовательный и параллельный (Кэннона) алгоритмы. Реализацию параллельного алгоритма провести средствами MPI. Кроме того, необходимо провести сравнения времени работы последовательного и параллельного алгоритмов.
\par Для реализации параллельной версии необходимо использовать средства MPI. Для проверки корректности работы алгоритмов требуется использовать Google C++ Testing Framework.
\newpage

% Описание алгоритма и метод решения
\section*{Описание алгоритма и метод решения}
\addcontentsline{toc}{section}{Описание алгоритма и метод решения}
Суть алгоритма Кэннона заключается в разбиении матриц А и В на блоки одинакового размера. В качестве базовой подзадачи выберем вычисления, связанные с определением одного из блоков результирующей матрицы C. Для вычисления элементов этого блока подзадача должна иметь доступ к элементам горизонтальной полосы матрицы А и элементам вертикальной полосы матрицы В.
\par Начальное расположение блоков в алгоритме Кэннона подбирается таким образом, чтобы блоки в подзадачах могли бы быть перемножены без каких-либо дополнительных передач данных. При этом подобное распределение блоков может быть организовано таким образом, что перемещение блоков между подзадачами в ходе вычислений может осуществляться с использованием более простых коммуникационных операций.
\par Этап инициализации алгоритма Кэннона  включает выполнение следующих
операций передач данных:
\begin{itemize}
\item в каждую подзадачу (i,j) передаются блоки Aij, Bij;
\item для каждой строки i решетки подзадач блоки матрицы A сдвигаются на (i-1) позиций влево;
\item для каждого столбца j решетки подзадач блоки матрицы B сдвигаются на (j-1) позиций
вверх.
\end{itemize}
\par В результате такого начального распределения в каждой базовой подзадаче будут
располагаться блоки, которые могут быть перемножены без дополнительных операций передачи
данных. Кроме того, получение всех последующих блоков для подзадач может быть обеспечено
при помощи простых коммуникационных действий — после выполнения операции блочного
умножения каждый блок матрицы A должен быть передан предшествующей подзадаче влево по
строкам решетки подзадач, а каждый блок матрицы В – предшествующей подзадаче вверх по
столбцам решетки. Как можно показать, последовательность таких циклических сдвигов и
умножение получаемых блоков исходных матриц A и B приведет к получению в базовых
подзадачах соответствующих блоков результирующей матрицы C.
\newpage

% Схема распараллеливания
\section*{Схема распараллеливания}
\addcontentsline{toc}{section}{Схема распараллеливания}
В первую очередь создается некоторое число p исполняющих процессов. Это число должно быть полным квадратом. Процессы организуются в виртуальную декартову топологию. Исходные матрицы должны иметь размерность, кратную $\sqrt{p}$.  Матрицы разбиваются на равное количество квадратных блоков. Блоки исходных матриц распределяются по исполняющим процессам. Затем для каждой строки i декартовой решетки блоки матрицы A сдвигаются на (i-1) позиций влево, для каждого столбца j декартовой решетки блоки матрицы B сдвигаются на (j-1) позиций вверх. 
\par Далее проводится $\sqrt{p}$ итераций, во время которых сначала происходит перемножение
блоков по методу трех вложенных циклов, и произведение складывается с текущим значением результирующего блока. Затем выполняется циклический сдвиг блоков матрицы А вдоль строк решетки и циклический сдвиг блоков матрицы В вверх по столбцам виртуальной решетки. После выполнения всех итераций результирующая матрица собирается из полученных на каждом процессе результирующих блоков.
\newpage

% Описание программной реализации
\section*{Описание программной реализации}
\addcontentsline{toc}{section}{Описание программной реализации}
Для решения поставленной задачи используется три основные функции:
\begin{itemize}
\item std::vector<double> GetRandomMatrix(const sizet size)– генерирование случайной матрицы,размером size;
\item std::vector<double> MatrixMultiplication(const std::vector<double> a, const std::vector<double> b, const sizet size)– последовательное умножение матриц;;
\item std::vector<double> Cannon(const std::vector<double> a, const std::vector<double> b, const sizet size);
\end{itemize}


\newpage

% Подтверждение корректности
\section*{Подтверждение корректности}
\addcontentsline{toc}{section}{Подтверждение корректности}
Для подтверждения корректности в программе представлен набор тестов, разработанных с помощью использования Google C++ Testing Framework.
\par Набор представляет из себя тесты, которые проверяют правильность выполнения транспонирования, корректность вычислений(то есть сравнение результата последовательного метода и результата параллельного метода), а также эффективность выполнения последовательного и параллельного метода.
\par Успешное прохождение всех тестов доказывает корректность работы всей программы.
\newpage

% Результаты экспериментов
\section*{Результаты экспериментов}
\addcontentsline{toc}{section}{Результаты экспериментов}
Вычислительные эксперименты для оценки эффективности алгоритма Кэннона производились на оборудовании со следующей аппаратной конфигурацией:

\begin{itemize}
\item Процессор: Intel(R) Core(TM) i7-2600K CPU @ 3.40 GHz, 3400 МГц, ядер 4;
\item Оперативная память: 8192 МБ (DDR4), 1200 MHz;
\item ОС: Microsoft Windows 7 professional 
\end{itemize}

\par Проведём эксперимент алгоритма Кэннона на матрицах с разными размерами на 4 процессорах. 
\par Результаты экспериментов представлены в таблице

\begin{table}[!h]
\caption{Результаты вычислительных экспериментов}
\centering
\begin{tabular}{p{3cm} p{4cm} p{4cm} p{4cm}}
Матрица  & Последовательно & Параллельно \\
500      & 2,682453          & 0,609559  \\
1000     & 24,109888         & 7,889110  \\
1500     & 84,156007         & 24,504893 \\
2000     & 236,019478        & 57,258855   

\end{tabular}
\end{table}

\par Полученные данные демонстрируют разность во времени работы при последовательном и параллельном вычислениях. По результатам можно сделать вывод, что параллельное выполнение программы выигрывает во времени у последовательного во всех случаях уравнений.
\newpage

% Заключение
\section*{Заключение}
\addcontentsline{toc}{section}{Заключение}
В результате выполнения лабораторной работы была разработана библиотека, реализующая
параллельный метод матричного умножения – алгоритм Кэннона, используя технологию MPI.
\par Для подтверждения корректности работы программы разработан и доведен до успешного
выполнения набор тестов с использованием библиотеки модульного тестирования Google C++
Testing Framework.
\par По данным экспериментов удалось сравнить время работы параллельного и последовательного
алгоритмов умножения матриц. Выявлено, что параллельный алгоритм Кэннона показывает высокую
эффективность на достаточно большом объеме данных.
\newpage

% Список литературы
\begin{thebibliography}{1}
\addcontentsline{toc}{section}{Список литературы}
\bibitem{Gergel}
Гергель В. П. Теория и практика параллельных вычислений. – 2007. 
\bibitem{Gergel}
Гергель В. П., Стронгин Р. Г. Основы параллельных вычислений для многопроцессорных вычислительных систем. – 2003.
\bibitem{Korneev}
Корнеев В. Д. Параллельное программирование в MPI //Новосибирск: Изд-во СО РАН. – 2000.
\end{thebibliography}
\newpage

% Приложение
\section*{Приложение}
\addcontentsline{toc}{section}{Приложение}
В данном разделе находится листинг всего кода, написанного в рамках лабораторной работы.
\begin{lstlisting}
// cannon.h

/ Copyright 2020 Kirichenko Nikita
#ifndef MODULES_TASK_3_KIRICHENKO_N_CANNON_CANNON_H_
#define MODULES_TASK_3_KIRICHENKO_N_CANNON_CANNON_H_

#include <vector>

std::vector<double> GetRandomMatrix(const size_t size);

std::vector<double> MatrixMultiplication(const std::vector<double>& a, const std::vector<double>& b, const size_t size);
std::vector<double> Cannon(const std::vector<double>& a, const std::vector<double>& b, const size_t size);

#endif  // MODULES_TASK_3_KIRICHENKO_N_CANNON_CANNON_H_#pragma once

\end{lstlisting}
\begin{lstlisting}
// cannon.cpp

// Copyright 2020 Kirichenko Nikita
#include <mpi.h>
#include <vector>
#include <random>
#include <ctime>
#include <cmath>
#include <exception>
#include "../../../modules/task_3/kirichenko_n_cannon/Cannon.h"
#define EPS 1e-5

static unsigned int offset = 0;

std::vector<double> GetRandomMatrix(const size_t size) {
    std::mt19937 gen;
    gen.seed(static_cast<unsigned int>(time(0)) + offset);
    offset += 10;

    const size_t count = size * size;
    std::vector<double> vec(count);
    for (size_t i = 0; i < count; ++i)
        vec[i] = gen() % 20;
    return vec;
}

std::vector<double> MatrixMultiplication(const std::vector<double>& a,
    const std::vector<double>& b, const size_t size) {
    if (a.size() != b.size())
        throw "[ERROR] Matrices A and B have got different size!";

    std::vector<double> c(size * size, 0);
    for (size_t i = 0; i < size; ++i)
        for (size_t j = 0; j < size; ++j)
            for (size_t k = 0; k < size; ++k)
                c[i * size + j] += a[i * size + k] * b[k * size + j];

    return c;
}

std::vector<double> Cannon(const std::vector<double>& a, const std::vector<double>& b, const size_t size) {
    int procRank, procCount;
    MPI_Status status;
    MPI_Comm_size(MPI_COMM_WORLD, &procCount);
    MPI_Comm_rank(MPI_COMM_WORLD, &procRank);

    if (procCount == 1)
        return MatrixMultiplication(a, b, size);

    int sizeA = 0;
    int sizeB = 0;
    int globalSize;
    if (procRank == 0) {
        sizeA = static_cast<int>(a.size());
        sizeB = static_cast<int>(b.size());
        globalSize = static_cast<int>(size);
    }
    MPI_Bcast(&sizeA, 1, MPI_INT, 0, MPI_COMM_WORLD);
    MPI_Bcast(&sizeB, 1, MPI_INT, 0, MPI_COMM_WORLD);
    MPI_Bcast(&globalSize, 1, MPI_INT, 0, MPI_COMM_WORLD);

    if (sizeA != sizeB) {
        throw "[ERROR] Matrices A and B have got different size!";
    }

    double sqrtProcCount = std::sqrt(procCount);
    if ((sqrtProcCount - std::floor(sqrtProcCount)) >= EPS) {
        throw "[ERROR] Number of processes must be a perfect square!";
    }

    if (globalSize % static_cast<int>(sqrtProcCount) != 0) {
        throw "[ERROR] Incorrect matrices size for Cannon algorithm!";
    }

    std::vector<double> c(globalSize * globalSize);

    int dimProcess = static_cast<int>(sqrtProcCount);
    int dimBlock = globalSize / dimProcess;

    int dims[] = { dimProcess, dimProcess };
    int periods[] = { 1, 1 };
    int reorder = 1;
    MPI_Comm cartComm;
    MPI_Cart_create(MPI_COMM_WORLD, 2, dims, periods, reorder, &cartComm);

    int globalSizes[2] = { static_cast<int>(globalSize), static_cast<int>(globalSize) };
    int localSize[2] = { dimBlock, dimBlock };
    int starts[2] = { 0, 0 };
    MPI_Datatype type, subarrtype;
    MPI_Type_create_subarray(2, globalSizes, localSize, starts, MPI_ORDER_C, MPI_DOUBLE, &type);
    MPI_Type_create_resized(type, 0, dimBlock * sizeof(a[0]), &subarrtype);
    MPI_Type_commit(&subarrtype);

    const int blockElemCount = dimBlock * dimBlock;
    std::vector<double> localA(blockElemCount);
    std::vector<double> localB(blockElemCount);
    std::vector<double> localC(blockElemCount, 0);

    int* sendCounts = new int[procCount];
    int* displs = new int[procCount];
    if (procRank == 0) {
        std::fill_n(sendCounts, procCount, 1);

        int disp = 0;
        for (int i = 0; i < dimProcess; i++) {
            for (int j = 0; j < dimProcess; j++) {
                displs[i * dimProcess + j] = disp;
                disp++;
            }
            disp += (dimBlock - 1) * dimProcess;
        }
    }

    MPI_Scatterv(a.data(), sendCounts, displs, subarrtype, &localA[0], blockElemCount, MPI_DOUBLE, 0, MPI_COMM_WORLD);
    MPI_Scatterv(b.data(), sendCounts, displs, subarrtype, &localB[0], blockElemCount, MPI_DOUBLE, 0, MPI_COMM_WORLD);

    int coord[2];
    int left, right, up, down;
    MPI_Cart_coords(cartComm, procRank, 2, coord);
    MPI_Cart_shift(cartComm, 1, coord[0], &left, &right);
    MPI_Sendrecv_replace(&localA[0], blockElemCount, MPI_DOUBLE, left, 1, right, 1, cartComm, &status);
    MPI_Cart_shift(cartComm, 0, coord[1], &up, &down);
    MPI_Sendrecv_replace(&localB[0], blockElemCount, MPI_DOUBLE, up, 1, down, 1, cartComm, &status);

    std::vector<double> tmpC(blockElemCount);
    for (int k = 0; k < dimProcess; k++) {
        tmpC = MatrixMultiplication(localA, localB, dimBlock);

        for (int i = 0; i < dimBlock; i++) {
            for (int j = 0; j < dimBlock; j++) {
                localC[i * dimBlock + j] += tmpC[i * dimBlock + j];
            }
        }

        MPI_Cart_shift(cartComm, 1, 1, &left, &right);
        MPI_Cart_shift(cartComm, 0, 1, &up, &down);
        MPI_Sendrecv_replace(&localA[0], blockElemCount, MPI_DOUBLE, left, 1, right, 1, cartComm, &status);
        MPI_Sendrecv_replace(&localB[0], blockElemCount, MPI_DOUBLE, up, 1, down, 1, cartComm, &status);
    }

    MPI_Gatherv(&localC[0], blockElemCount, MPI_DOUBLE, &c[0], sendCounts, displs, subarrtype, 0, MPI_COMM_WORLD);

    return c;
}

\end{lstlisting}
\begin{lstlisting}
// main.cpp
// Copyright 2020 Kirichenko Nikita

#include <gtest-mpi-listener.hpp>
#include <gtest/gtest.h>
#include <vector>
#include <cmath>
#include "./Cannon.h"

#define EPS 1e-5

TEST(Cannon, DISABLED_Time) {
    int procRank, procCount;
    MPI_Comm_rank(MPI_COMM_WORLD, &procRank);
    MPI_Comm_size(MPI_COMM_WORLD, &procCount);

    const size_t blockSize = 200;

    int size = procCount * blockSize;
    double dimProcess = std::sqrt(procCount);
    int intDimProcess = static_cast<int>(dimProcess);
    if (std::fabs(dimProcess - intDimProcess) > EPS)
        return;

    std::vector<double> a, b, c;

    double starttime, endtime;
    if (procRank == 0) {
        a = GetRandomMatrix(size);
        b = GetRandomMatrix(size);

        starttime = MPI_Wtime();
        std::vector<double> ref = MatrixMultiplication(a, b, size);
        endtime = MPI_Wtime();
        std::cout << "Sequential time: " << endtime - starttime << std::endl;

        a = GetRandomMatrix(size);
        b = GetRandomMatrix(size);
    }

    MPI_Barrier(MPI_COMM_WORLD);

    starttime = MPI_Wtime();
    c = Cannon(a, b, size);
    endtime = MPI_Wtime();
    if (procRank == 0)
        std::cout << "Parallel time: " << endtime - starttime << std::endl;
}

TEST(Cannon, IncorrectSize) {
    std::vector<double> a(9, 1);
    std::vector<double> b(16, 1);
    ASSERT_ANY_THROW(Cannon(a, b, 3));
}

TEST(Cannon, SequantalMultuplication) {
    int procRank, procCount;
    MPI_Comm_rank(MPI_COMM_WORLD, &procRank);
    MPI_Comm_size(MPI_COMM_WORLD, &procCount);

    if (procRank == 0) {
        const size_t size = 2;
        const size_t countElem = size * size;
        std::vector<double> a = { 0, 1, 2, 3 };
        std::vector<double> b = { 3, 2, 1, 0 };
        std::vector<double> ref = { 1, 0, 9, 4 };
        std::vector<double> c(countElem);
        c = MatrixMultiplication(a, b, size);

        for (size_t i = 0; i < countElem; ++i)
            ASSERT_NEAR(ref[i], c[i], EPS);
    }
}

TEST(Cannon, SmallSize) {
    int procRank, procCount;
    MPI_Comm_rank(MPI_COMM_WORLD, &procRank);
    MPI_Comm_size(MPI_COMM_WORLD, &procCount);

    const size_t blockSize = 4;

    int size = procCount * blockSize;
    double dimProcess = std::sqrt(procCount);
    int intDimProcess = static_cast<int>(dimProcess);
    if (std::fabs(dimProcess - intDimProcess) > EPS)
        return;

    std::vector<double> a, b, c;

    if (procRank == 0) {
        a = GetRandomMatrix(size);
        b = GetRandomMatrix(size);
    }

    c = Cannon(a, b, size);

    if (procRank == 0) {
        std::vector<double> ref = MatrixMultiplication(a, b, size);

        const size_t countElem = size * size;
        for (size_t i = 0; i < countElem; ++i)
            ASSERT_NEAR(ref[i], c[i], EPS);
    }
}

TEST(Cannon, MediumSize) {
    int procRank, procCount;
    MPI_Comm_rank(MPI_COMM_WORLD, &procRank);
    MPI_Comm_size(MPI_COMM_WORLD, &procCount);

    const size_t blockSize = 11;

    int size = procCount * blockSize;
    double dimProcess = std::sqrt(procCount);
    int intDimProcess = static_cast<int>(dimProcess);
    if (std::fabs(dimProcess - intDimProcess) > EPS)
        return;

    std::vector<double> a, b, c;

    if (procRank == 0) {
        a = GetRandomMatrix(size);
        b = GetRandomMatrix(size);
    }

    c = Cannon(a, b, size);

    if (procRank == 0) {
        std::vector<double> ref = MatrixMultiplication(a, b, size);

        const size_t countElem = size * size;
        for (size_t i = 0; i < countElem; ++i)
            ASSERT_NEAR(ref[i], c[i], EPS);
    }
}

TEST(Cannon, LargeSize) {
    int procRank, procCount;
    MPI_Comm_rank(MPI_COMM_WORLD, &procRank);
    MPI_Comm_size(MPI_COMM_WORLD, &procCount);

    const size_t blockSize = 23;

    int size = procCount * blockSize;
    double dimProcess = std::sqrt(procCount);
    int intDimProcess = static_cast<int>(dimProcess);
    if (std::fabs(dimProcess - intDimProcess) > EPS)
        return;

    std::vector<double> a, b, c;

    if (procRank == 0) {
        a = GetRandomMatrix(size);
        b = GetRandomMatrix(size);
    }

    c = Cannon(a, b, size);

    if (procRank == 0) {
        std::vector<double> ref = MatrixMultiplication(a, b, size);

        const size_t countElem = size * size;
        for (size_t i = 0; i < countElem; ++i)
            ASSERT_NEAR(ref[i], c[i], EPS);
    }
}

int main(int argc, char** argv) {
    ::testing::InitGoogleTest(&argc, argv);
    MPI_Init(&argc, &argv);

    ::testing::AddGlobalTestEnvironment(new GTestMPIListener::MPIEnvironment);
    ::testing::TestEventListeners& listeners =
        ::testing::UnitTest::GetInstance()->listeners();

    listeners.Release(listeners.default_result_printer());
    listeners.Release(listeners.default_xml_generator());

    listeners.Append(new GTestMPIListener::MPIMinimalistPrinter);
    return RUN_ALL_TESTS();
}

\end{lstlisting}

\end{document}