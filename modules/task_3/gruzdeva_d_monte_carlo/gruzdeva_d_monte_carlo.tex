\documentclass{report}

\usepackage[T2A]{fontenc}
\usepackage[utf8]{luainputenc}
\usepackage[english, russian]{babel}
\usepackage[pdftex]{hyperref}
\usepackage[14pt]{extsizes}
\usepackage{listings}
\usepackage{color}
\usepackage{geometry}
\usepackage{enumitem}
\usepackage{multirow}
\usepackage{graphicx}
\usepackage{indentfirst}
\usepackage{amsmath}

\geometry{a4paper,top=2cm,bottom=2cm,left=2.5cm,right=1.5cm}
\setlength{\parskip}{0.5cm}
\setlist{nolistsep, itemsep=0.3cm,parsep=0pt}

\usepackage{xcolor}
\definecolor{weborange}{RGB}{255,165,0}
\definecolor{codepurple}{RGB}{220,20,60}
\definecolor{codecyan}{RGB}{135,0,198}
\definecolor{codeblue}{RGB}{69,97,189}
\definecolor{backcolour}{RGB}{230,230,220}

\usepackage{listings}
\lstset{language=C++,
	basicstyle=\ttfamily\footnotesize,
	backgroundcolor=\color{backcolour},
    	emph={int, double, time_t, char, new},
    	emphstyle={\color{blue}},
    	commentstyle=\color{green!60!black},
 	keywordstyle={\color{codepurple}},
 	keywords=[2]{std},
    	keywordstyle=[2]{\color{weborange}},
    	otherkeywords = {:},
	morekeywords = [2]{:},
   	stringstyle=\color{codepurple},
	morecomment=[l][\color{codecyan}]{\#}, 
	tabsize=4,
	breaklines=true,                               
  	breakatwhitespace=true,
  	title=\lstname,  
}

\makeatletter
\renewcommand\@biblabel[1]{#1.\hfil}
\makeatother

\begin{document}

\begin{titlepage}

\begin{center}
Министерство науки и высшего образования Российской Федерации
\end{center}

\begin{center}
Федеральное государственное автономное образовательное учреждение высшего образования \\
Национальный исследовательский Нижегородский государственный университет им. Н.И. Лобачевского
\end{center}

\begin{center}
Институт информационных технологий, математики и механики
\end{center}

\vspace{4em}

\begin{center}
\textbf{\LargeОтчет по лабораторной работе} \\
\end{center}
\begin{center}
\textbf{\Large«Вычисление многомерных интегралов методом Монте-Карло»} \\
\end{center}

\vspace{4em}

\newbox{\lbox}
\savebox{\lbox}{\hbox{text}}
\newlength{\maxl}
\setlength{\maxl}{\wd\lbox}
\hfill\parbox{7cm}{
\hspace*{5cm}\hspace*{-5cm}\textbf{Выполнил:} \\ студент группы 381806-2 \\ Груздева Д. М.\\
\\
\hspace*{5cm}\hspace*{-5cm}\textbf{Проверил:}\\ доцент кафедры МОСТ, \\ кандидат технических наук \\ Сысоев А. В.\\
}
\vspace{\fill}

\begin{center} Нижний Новгород \\ 2020 \end{center}

\end{titlepage}

\setcounter{page}{2}

% Содержание
\tableofcontents
\newpage

% Введение
\section*{Введение}
\addcontentsline{toc}{section}{Введение}
\par Методами Монте-Карло называют численные методы решения математических задач при помощи моделирования случайных величин. Однако, решать методами Монте-Карло можно любые математические задачи, а не только задачи вероятностного происхождения, связанные со случайными величинами. Важнейшим приемом построения методов Монте-Карло является сведение задачи к расчету математических ожиданий. Так как математические ожидания чаще всего представляют собой обычные интегралы, то центральное положение в теории метода Монте-Карло занимают методы вычисления интегралов.
\newpage

% Постановка задачи
\section*{Постановка задачи}
\addcontentsline{toc}{section}{Постановка задачи}
\par В данном лабораторной работе требуется рассмотреть вычисление тройных интегралов методом Монте-Карло, реализовать последовательную и параллельную версии этого метода, проверить корректность разработанной реализации, провести эксперименты для оценки времени. По полученным результатам сделать выводы.
\par Параллельная версия будет реализована при помощи технологии MPI. Проверка корректности будет осуществлена при помощи библиотеки для модульного тестирования Google C++ Testing Framework.
\newpage

% Описание алгоритма
\section*{Описание алгоритма}
\addcontentsline{toc}{section}{Описание алгоритма}
\par Задан тройной интеграл вида:
$$\iiint\limits_D f(x,y,z) \,dx\,dy\,dz, $$
где подинтегральная функция задана в трехмерном промежутке (параллелепипеде) $D=[a_1,b_1]\times\*[a_2,b_2]\times[a_3,b_3]$.
\par Введем трехмерную случайную величину $X$, имеющую в замкнутной области равномерное распределение вероятностей с дифференциальной функцией плотности
$$p(x)=\frac{1}{\mu(D)}.$$
\par Функция плотности равномерного распределения есть величина постоянная, поэтому введем ее под знак интеграла следующим образом:
$$I=\iiint\limits_D f(x,y,z)\frac{\mu(D)}{\mu(D)} \,dx\,dy\,dz.$$
\par Вынесем числитель дроби за знак интеграла, т. е.
$$I=\mu(D)\iiint\limits_D f(x,y,z)\frac{1}{\mu(D)} \,dx\,dy\,dz.$$
\par Получается, что тройной интеграл - математическое ожидание от функции $f(x)$ случайной величины в предположении, что случайная величина $X$ распределена равномерно с рассмотренной выше плотностью. Значит, мы можем записать
$$I=\mu(D)M[f(\xi)],$$
где $\xi=(\xi_1,\xi_2,\xi_3)$.
\par В свою очередь, математическое ожидание может быть оценено с помощью арифметического среднего. Тогда приближенное значение тройного интеграла будет определяться приближенной формулой
$$I\approx\frac{\mu(D)}{N}\sum\limits_{k=1}^N f(\xi^{(k)}),$$
где $\xi^{(k)} \in D$ - значение случайной величины $X$ в $k$-м испытании.
\par Чтобы смоделировать выборку трехмерной случайной величины $Х$, равномерно распределенной в трехмерном промежутке $D$, используются псевдослучайные числа. Для этого в каждом испытании с номером $k$ выбирают 3 псевдослучайных числа $r_i^{(k)}$, $i=\overline{1,3}$, и по ним определяют координаты случайных величин $\xi_i^{(k)}=a_i+(b_i-a_i)\*r_i^{(k)}$, $i=\overline{1,3}$, псевдослучайной точки $\xi^{(k)}$.
\par Таким образом, техника применения метода Монте-Карло будет заключаться в генерировании в области $D$ псевдослучайных чисел и применении приближенной формулы, описанной выше.
\newpage

% Схема распараллеливания
\section*{Схема распараллеливания}
\addcontentsline{toc}{section}{Схема распараллеливания}
\par Для распараллеливания подсчета тройного интеграла методом Монте-Карло, разделим параллелепипед $D$ на равные части: отрезки $[a_1,b_1]$ и $[a_2,b_2]$ оставляем прежними, а $[a_3,b_3]$ делим на части одинаковой длины, количество этих частей равно запускаемым процессам. Получаем несколько параллелепипедов. Распределим изначальное количество испытаний между ними (последнему параллелепипеду может достаться меньше, в случае если невозможно разделить испытания на процессы без остатка). После этого, параллельно запускаем последовательный алгоритм на всех процессах.
\par В итоге, каждый процесс считает свою часть интеграла и отправляет ее корневому процессу, после чего тот суммирует все присланные ему части и возвращает уже итоговый инеграл.
\newpage

% Описание программной реализации
\section*{Описание программной реализации}
\addcontentsline{toc}{section}{Описание программной реализации}
Для начала реализуем вызов интегрируемых функций через шаблонный класс \lstinline$std:function$
\begin{lstlisting}
double callFunction(std::function<double(double, double, double)> function, double x, double y, double z);
\end{lstlisting}
\par Теперь, мы можем передавать функцию в функцию. Опишем функцию для последовательной версии алгоритма.
\begin{lstlisting}
double getSequentialIntegral(double x1, double x2, double y1, double y2, double z1, double z2, int n, std::function<double(double, double, double)> function, time_t seed);
\end{lstlisting}
Первые шесть параметров - границы интегрирования, далее \lstinline$n$ - количество испытаний, \lstinline$std::function<double(double, double, double)> function$ - интегрируемая функция, \lstinline$time_t seed$ - семя генератора псевдослучайных чисел.
Внутри функции создаем сам генератор:
\begin{lstlisting}
std::mt19937 gen(seed);
std::uniform_real_distribution<double> uniform(0, 1);
\end{lstlisting}
Создаем цикл с количеством итераций, равным количеству испытаний, в нем для каждой координаты имеем:
\begin{lstlisting}
double x = x1 + (x2 - x1) * uniform(gen);
\end{lstlisting}
После чего вызываем функцию \lstinline$std::function<double(double, double, double)> function$ и приплюсовываем ее значение к конечному результату.
\par Перейдем к параллельной функции
\begin{lstlisting}
double getParallelIntegral(double x1, double x2, double y1, double y2, double z1, double z2, int n, std::function<double(double, double, double)> function, time_t seed) 
\end{lstlisting}
Она получает такие же параметры, что и ее последовательная версия. Найдем высоту параллелепипеда: \lstinline$double height = (z2 - z1) / n;$ и количество испытаний (в том числе и для последнего процесса): \lstinline$int local_n = n / size, remaining = n - (size - 1) * local_n;$.
Находим границы интегрирования:  \lstinline$double local_z1 = z1 + rank * local_n * height, local_z2 = local_z1 + local_n * height;$. Передаем все эти переменные в последовательную функцию. Получаем посчитанный параллельно интеграл.
\newpage

% Подтверждение корректности
\section*{Подтверждение корректности}
\addcontentsline{toc}{section}{Подтверждение корректности}
Для подтверждения корректности в программе представлен набор тестов, разработанных с помощью использования Google C++ Testing Framework.
\par Набор представляет из себя тесты, которые проверяют:
\begin{itemize}
\item Корректность входных данных (количество испытаний не может быть отрицательным);
\item Корректность вычислений (проверка результатов работы последовательной версии, идет сравнение с результатами, полученными аналитическим интегрированием);
\item Эффективность по времени работы последовательной и параллельной функции.
\end{itemize}
\par Прохождение всего набора тестов подтверждает правильность реализации программы.
\newpage

% Результаты экспериментов
\section*{Результаты экспериментов}
\addcontentsline{toc}{section}{Результаты экспериментов}
Вычислительные эксперименты для оценки временной эффективности параллельной реализации вычисления тройных интегралов методом Монте-Карло проводились на оборудовании со следующей конфигурацией:
\begin{itemize}
\item Процессор: Intel Core i3-4130, 3400 MHz, физических ядер: 2, потоков: 4;
\item Оперативная память: 8192 МБ (DDR3), 800 MHz;
\item ОС: Microsoft Windows 10 Home, версия 2004 сборка 19041.572.
\end{itemize}

\par Количество испытаний для оценки быстродействия программы составило \verb|1 000 000|. 
\par Результаты экспериментов представлены в Таблице 1.

\begin{table}[!h]
\caption{Результаты вычислительных экспериментов}
\centering
\begin{tabular}{| r | r | r | r |}
\hline
Количество процессов & Последовательно & Параллельно & Ускорение  \\[5pt]
\hline
1        & 0.335709        & 0.339896     & 0.99       \\
2        & 0.343937        & 0.1721     & 2.00       \\
3        & 0.377498        & 0.112703     & 3.35       \\
4        & 0.338811        & 0.0877148     & 3.86       \\
5        & 0.340334        & 0.0686788     & 4.95       \\
6        & 0.338043        & 0.056471     & 5.99       \\
7        & 0.328287        & 0.0496489     & 6.61	    \\
8        & 0.333299        & 0.0427445     & 7.80	  \\
\hline
\end{tabular}
\end{table}

\par Из данных, представленных в Таблице 1 видно, что параллельная версия быстрее последовательной, а ее ускорение в общем случае пропорционально количеству запускаемых процессов. Такой хороший результат объясняется тем, что мы использовали относительно большое количество испытаний, из-за чего последовательная версия выполнялась гораздо дольше.
\newpage

% Заключение
\section*{Заключение}
\addcontentsline{toc}{section}{Заключение}
В результате лабораторной работы были разработаны последовательная и параллельная реализация подсчета тройных интегралов методом Монте-Карло.
\par Главной задачей данной лабораторной работы была реализация параллельной версии, которая должна была быть эффективнее последовательной. Эта задача была успешно достигнута, о чем говорят результаты экспериментов, проведенных в ходе работы.
\par Кроме того, были разработаны и доведены до успешного выполнения тесты, созданные для данного программного проекта с использованием Google C++ Testing Framework и необходимые для подтверждения корректности работы программы.
\newpage

% Список литературы
\begin{thebibliography}{1}
\addcontentsline{toc}{section}{Список литературы}
\bibitem{Voytishek} Войтишек А.В. Основы метода Монте-Карло: Учеб. пособие / Новосиб. гос. ун-т. Новосибирск, 2010, 108 с. 
\bibitem{Zorinl} Зорин А.В, Федоткин М.А. Методы Монте-Карло для параллельных вычислений. — М.: Изд-во Московского университета, 2013.
\bibitem{Gergel} Гергель В.П., Стронгин Р.Г. Высокопроизводительные вычисления для многоядерных многопроцессорных систем. — Н.Новгород: Изд-во ННГУ, 2010.
\end{thebibliography}
\newpage

% Приложение
\section*{Приложение}
\addcontentsline{toc}{section}{Приложение}

monte\_carlo.h
\begin{lstlisting}

// Copyright 2020 Gruzdeva Diana
#ifndef MODULES_TASK_3_GRUZDEVA_D_MONTE_CARLO_MONTE_CARLO_H_
#define MODULES_TASK_3_GRUZDEVA_D_MONTE_CARLO_MONTE_CARLO_H_

#include <mpi.h>
#include <random>
#include <ctime>
#include <functional>

double calculateError(int p, int n);

double getSequentialIntegral(double x1, double x2,
          double y1, double y2, double z1, double z2, int n,
          std::function<double(double, double, double)> function, time_t seed);
double getParallelIntegral(double x1, double x2,
        double y1, double y2, double z1, double z2, int n,
        std::function<double(double, double, double)> function, time_t seed);

double callFunction(std::function<double(double)> function, double value);
double linearFunction(double x, double y, double z);
double polinomFunction(double x, double y, double z);
double compositeFunction(double x, double y, double z);

#endif  // MODULES_TASK_3_GRUZDEVA_D_MONTE_CARLO_MONTE_CARLO_H_
\end{lstlisting}
monte\_carlo.cpp
\begin{lstlisting}
// Copyright 2020 Gruzdeva Diana

#include "../../../modules/task_3/gruzdeva_d_monte_carlo/monte_carlo.h"
#include <random>
#include <ctime>

double calculateError(int p, int n) {
    double error = p * sqrt(0.125 / n);
    return error;
}

double getSequentialIntegral(double x1, double x2,
          double y1, double y2, double z1, double z2, int n,
          std::function<double(double, double, double)> function, time_t seed) {
    std::mt19937 gen(seed);
    std::uniform_real_distribution<double> uniform(0, 1);
    double result = 0.0;
    double sum = 0.0;
    for (int i = 0; i < n; i++) {
        double x = x1 + (x2 - x1) * uniform(gen);
        double y = y1 + (y2 - y1) * uniform(gen);
        double z = z1 + (z2 - z1) * uniform(gen);
        sum += function(x, y, z);
    }
    result = sum * (x2 - x1) * (y2 - y1) * (z2 - z1) / n;
    return result;
}

double getParallelIntegral(double x1, double x2,
          double y1, double y2, double z1, double z2, int n,
          std::function<double(double, double, double)> function, time_t seed) {
    int size, rank;
    double height = (z2 - z1) / n;
    MPI_Comm_size(MPI_COMM_WORLD, &size);
    MPI_Comm_rank(MPI_COMM_WORLD, &rank);
    double result = 0;
    int local_n = n / size,
        remaining = n - (size - 1) * local_n;
    double local_result = 0.0,
           local_z1 = z1 + rank * local_n * height,
           local_z2 = local_z1 + local_n * height;
    if (rank != size - 1) {
        local_result = getSequentialIntegral(x1, x2, y1, y2, local_z1, local_z2, local_n, function, seed);
    } else {
        local_result = getSequentialIntegral(x1, x2, y1, y2, local_z1, z2, remaining, function, seed);
    }
    MPI_Reduce(&local_result, &result, 1, MPI_DOUBLE, MPI_SUM, 0, MPI_COMM_WORLD);
    return result;
}

double callFunction(std::function<double(double, double, double)> function, double x, double y, double z) {
    return function(x, y, z);
}

double linearFunction(double x, double y, double z) {
    return (x + y + z);
}

double polinomFunction(double x, double y, double z) {
    return x * x + y - z * z * z + 6;
}

double compositeFunction(double x, double y, double z) {
    return sin(x * y * z);
}
\end{lstlisting}
main.cpp
\begin{lstlisting}
// Copyright 2020 Gruzdeva Diana
#include <gtest-mpi-listener.hpp>
#include <gtest/gtest.h>
#include <iostream>
#include "./monte_carlo.h"


TEST(Parallel_Operations_MPI, LINEAR_10_3) {
    int rank;
    int n = 1000;
    double error = calculateError(10, n);
    MPI_Comm_rank(MPI_COMM_WORLD, &rank);
    double start = MPI_Wtime();
    double parallel_result = getParallelIntegral(0.0, 1.0, 0.0, 1.0, 0.0, 1.0, n, linearFunction, time(0));
    double end = MPI_Wtime();
    if (rank == 0) {
        std::cout << end - start << std::endl;
        start = MPI_Wtime();
        double sequential_result = getSequentialIntegral(0.0, 1.0, 0.0, 1.0, 0.0, 1.0, n, linearFunction, time(0));
        double end = MPI_Wtime();
        std::cout << end - start << std::endl;
        std::cout << sequential_result << std::endl;
        ASSERT_LT(std::fabs(parallel_result - sequential_result), error);
    }
}

TEST(Parallel_Operations_MPI, LINEAR_10_6) {
    int rank;
    int n = 1000000;
    double error = calculateError(10, n);
    MPI_Comm_rank(MPI_COMM_WORLD, &rank);
    double start = MPI_Wtime();
    double parallel_result = getParallelIntegral(0.0, 1.0, 0.0, 1.0, 0.0, 1.0, n, linearFunction, time(0));
    double end = MPI_Wtime();
    if (rank == 0) {
        std::cout << end - start << std::endl;
        start = MPI_Wtime();
        double sequential_result = getSequentialIntegral(0.0, 1.0, 0.0, 1.0, 0.0, 1.0, n, linearFunction, time(0));
        double end = MPI_Wtime();
        std::cout << end - start << std::endl;
        std::cout << sequential_result << std::endl;
        std::cout << parallel_result << std::endl;
        ASSERT_LT(std::fabs(parallel_result - sequential_result), error);
    }
}

TEST(Parallel_Operations_MPI, POLINOM_10_3) {
    int rank;
    int n = 1000;
    double error = calculateError(300, n);
    MPI_Comm_rank(MPI_COMM_WORLD, &rank);
    double start = MPI_Wtime();
    double parallel_result = getParallelIntegral(0.0, 1.0, 0.0, 1.0, 0.0, 1.0, n, polinomFunction, time(0));
    double end = MPI_Wtime();
    if (rank == 0) {
        std::cout << end - start << std::endl;
        start = MPI_Wtime();
        double sequential_result = getSequentialIntegral(0.0, 1.0, 0.0, 1.0, 0.0, 1.0, n, polinomFunction, time(0));
        double end = MPI_Wtime();
        std::cout << end - start << std::endl;
        std::cout << sequential_result << std::endl;
        std::cout << parallel_result << std::endl;
        ASSERT_LT(std::fabs(parallel_result - sequential_result), error);
    }
}

TEST(Parallel_Operations_MPI, POLINOM_10_6) {
    int rank;
    int n = 1000000;
    double error = calculateError(300, n);
    MPI_Comm_rank(MPI_COMM_WORLD, &rank);
    double start = MPI_Wtime();
    double parallel_result = getParallelIntegral(0.0, 1.0, 0.0, 1.0, 0.0, 1.0, n, polinomFunction, time(0));
    double end = MPI_Wtime();
    if (rank == 0) {
        std::cout << end - start << std::endl;
        start = MPI_Wtime();
        double sequential_result = getSequentialIntegral(0.0, 1.0, 0.0, 1.0, 0.0, 1.0, n, polinomFunction, time(0));
        double end = MPI_Wtime();
        std::cout << end - start << std::endl;
        std::cout << sequential_result << std::endl;
        std::cout << parallel_result << std::endl;
        ASSERT_LT(std::fabs(parallel_result - sequential_result), error);
    }
}


TEST(Parallel_Operations_MPI, COMPOSITE_10_3) {
    int rank;
    int n = 1000;
    double error = calculateError(300, n);
    MPI_Comm_rank(MPI_COMM_WORLD, &rank);
    double start = MPI_Wtime();
    double parallel_result = getParallelIntegral(0.0, 1.0, 0.0, 1.0, 0.0, 1.0, n, compositeFunction, time(0));
    double end = MPI_Wtime();
    if (rank == 0) {
        std::cout << end - start << std::endl;
        start = MPI_Wtime();
        double sequential_result = getSequentialIntegral(0.0, 1.0, 0.0, 1.0, 0.0, 1.0, n, compositeFunction, time(0));
        double end = MPI_Wtime();
        std::cout << end - start << std::endl;
        std::cout << sequential_result << std::endl;
        std::cout << parallel_result << std::endl;
        ASSERT_LT(std::fabs(parallel_result - sequential_result), error);
    }
}

TEST(Parallel_Operations_MPI, COMPOSITE_10_6) {
    int rank;
    int n = 1000000;
    double error = calculateError(300, n);
    MPI_Comm_rank(MPI_COMM_WORLD, &rank);
    double start = MPI_Wtime();
    double parallel_result = getParallelIntegral(0.0, 1.0, 0.0, 1.0, 0.0, 1.0, n, compositeFunction, time(0));
    double end = MPI_Wtime();
    if (rank == 0) {
        std::cout << end - start << std::endl;
        start = MPI_Wtime();
        double sequential_result = getSequentialIntegral(0.0, 1.0, 0.0, 1.0, 0.0, 1.0, n, compositeFunction, time(0));
        double end = MPI_Wtime();
        std::cout << end - start << std::endl;
        std::cout << sequential_result << std::endl;
        std::cout << parallel_result << std::endl;
        ASSERT_LT(std::fabs(parallel_result - sequential_result), error);
    }
}

int main(int argc, char** argv) {
    ::testing::InitGoogleTest(&argc, argv);
    MPI_Init(&argc, &argv);

    ::testing::AddGlobalTestEnvironment(new GTestMPIListener::MPIEnvironment);
    ::testing::TestEventListeners& listeners =
        ::testing::UnitTest::GetInstance()->listeners();

    listeners.Release(listeners.default_result_printer());
    listeners.Release(listeners.default_xml_generator());

    listeners.Append(new GTestMPIListener::MPIMinimalistPrinter);
    return RUN_ALL_TESTS();
}
\end{lstlisting}

\end{document}
