\documentclass{report}

\usepackage[T2A]{fontenc}
\usepackage[utf8]{luainputenc}
\usepackage[english, russian]{babel}
\usepackage[pdftex]{hyperref}
\usepackage[14pt]{extsizes}
\usepackage{listings}
\usepackage{color}
\usepackage{geometry}
\usepackage{enumitem}
\usepackage{multirow}
\usepackage{graphicx}
\usepackage{indentfirst}
\usepackage{amsmath}


\geometry{a4paper,top=2cm,bottom=3cm,left=2cm,right=1.5cm}
\setlength{\parskip}{0.5cm}
\setlist{nolistsep, itemsep=0.3cm,parsep=0pt}

\lstset{language=C++,
		basicstyle=\footnotesize,
		keywordstyle=\color{blue}\ttfamily,
		stringstyle=\color{red}\ttfamily,
		commentstyle=\color{green}\ttfamily,
		morecomment=[l][\color{magenta}]{\#}, 
		tabsize=4,
		breaklines=true,
  		breakatwhitespace=true,
  		title=\lstname,       
}

\makeatletter
\renewcommand\@biblabel[1]{#1.\hfil}
\makeatother

\begin{document}

\begin{titlepage}

\begin{center}
Министерство науки и высшего образования Российской Федерации
\end{center}

\begin{center}
Федеральное государственное автономное образовательное учреждение высшего образования \\
Национальный исследовательский Нижегородский государственный университет им. Н.И. Лобачевского
\end{center}

\begin{center}
Институт информационных технологий, математики и механики
\end{center}

\vspace{4em}

\begin{center}
\textbf{\LargeОтчет по лабораторной работе} \\
\end{center}
\begin{center}
\textbf{\Large«Алгоритм глобального поиска (Стронгина) для одномерных задач оптимизации»} \\
\end{center}

\vspace{4em}

\newbox{\lbox}
\savebox{\lbox}{\hbox{text}}
\newlength{\maxl}
\setlength{\maxl}{\wd\lbox}
\hfill\parbox{7cm}{
\hspace*{5cm}\hspace*{-5cm}\textbf{Выполнил:} \\ студент группы 381806-1 \\ Тактаев А. А.\\
\\
\hspace*{5cm}\hspace*{-5cm}\textbf{Проверил:}\\ доцент кафедры МОСТ, \\ кандидат технических наук \\ Сысоев А. В.\\
}
\vspace{\fill}

\begin{center} Нижний Новгород \\ 2020 \end{center}

\end{titlepage}

\setcounter{page}{2}

% Содержание
\tableofcontents
\newpage

% Введение
\section*{Введение}
\addcontentsline{toc}{section}{Введение}
Задачи многоэкстремальной оптимизации обладают узкой сферой возможностей аналитического исследования и высокой трудоемкостью численного решения, поскольку в последнем случае для них характерен экспоненциальный рост вычислительных затрат с ростом размерности.
\par Одним из методов решения задач одномерной оптимизации является алгоритм глобального поиска, носящий имя доктора физико-математических наук Романа Григорьевича Стронгина. Этот метод работает для функций, удовлетворяющих условию Липшица и основан на вероятностной модели функций, заданных на конечном множестве точек.

\newpage

% Постановка задачи
\section*{Постановка задачи}
\addcontentsline{toc}{section}{Постановка задачи}
В рамках лабораторной работы необходимо реализовать последовательную и параллельную реализации алгоритма Стронгина для одномерных задач оптимизации, проверить корректность работы алгоритмов, сравнить эффективность последовательной и параллельной реализации. Распараллеливание должно происходить по характеристическим функциям. В качестве целевой используем липшицеву функцию одного аргумента: 
\par $y = f(x), x\in[a, b], \mid f(x_1) - f(x_2) \mid \le L \mid x_1 - x_2 \mid , L = const $
\par Для реализации параллельной версии необходимо использовать библиотеку межпроцессорного взаимодействия MPI. Для проверки корректности работы алгоритмов используется Google Testing Framework.
\newpage

% Метод решения
\section*{Описание алгоритмов}
\addcontentsline{toc}{section}{Метод решения}
Алгоритм глобального поиска (АГП) позволяет находить абсолютный минимум функции
на отрезке и основан на вероятностном подходе.
\par На основе набора известных значений функции в точках отрезка ищется интервал между
соседними точками, на котором абсолютный минимум наиболее вероятен. На этом интервале
берется точка, соответствующая математическому ожиданию положения минимума, вычисляется значение функции в ней. Точка добавляется в список известных значений, и происходит
переход к следующей итерации. Алгоритм останавливается, когда расстояние между точками
отрезка последовательных итераций становится меньше заданного критерия.
Единственным требованием, которому должна удовлетворять целевая функция $g(x)$,  —
это выполнение обобщенного условия Липшица на всем интервале поиска: 
\par $\mid f(x_1) - f(x_2) \mid \le L \mid x_1 - x_2 \mid , L = const $
\par Опишем шаги алгоритма. Рассмотрим функцию $g(x)$ на отрезке $[a,b]$
вещественной оси. Обозначим $z_i = g(x_i)$ значения целевой функции в точках $x_i$ отрезка $[a,b]$.
\par В начале алгоритма положим $x_0 = a, x_1 = b$ и вычислим значения функции $g(a)$ и $g(b)$ в этих
точках. Процедура $k + 1$ итерации состоит в следующем. Пусть $x_i$ и $z_i$ для $i = 0, 1, ..., k$ нам уже
известны из предыдущих $k$ итераций.
\begin{enumerate}
\item Перенумеровать точки $x_i$, $i = 0, 1, ..., k$ в порядке возрастания значений $a = x_0 < ... < x_k = b$.
\item Оценить максимальное абсолютное значение относительной первой разности: 
\par $M = \max\limits_{ 1 \le i \le k} \cfrac{\mid z_i - z_{i-1} \mid}{x_i - x_{i-1}}$.
\item Положить: 
\par $m = \begin{cases}
1, &\text{если $M = 0$;}\\
rM, &\text{если $M > 0$.}
\end{cases}$
\par где $r > 1$ есть заданный коэффициент, параметр алгоритма, который определяется из предположений о коэффициенте K в условии Липшица.
\item Для каждого интервала $(x_{i-1}, x_i), 0 \le i \le k$ , вычислить величину:
\par $R(i) = m(x_i - x_{i-1}) + \cfrac{(z_i - z_{i - 1})^2}{m(x_i - x_{i-1})} - 2(z_i + z_{i - 1})$
\par которая называется характеристикой интервала и определяет вероятность нахождения глобального минимума на этом интервале. Чем она больше, тем больше вероятность.
\item Определить интервал $(x_{i-1}, x_i)$, которому соответствует максимальная характеристика $R(t) = \max\limits_{ 1 \le i \le k} R(i)$
\par Если максимальная характеристика соответствует нескольким интервалам,
то в качестве t выбирается минимальное число.
\item Положить:
\par $x_{k + 1} = \cfrac{x_{t - 1} + x_t}{2} - \cfrac{z_t - z_{t-1}}{2m}$
\end{enumerate}
\par Алгоритм останавливается, когда $x_t - x_{t - 1} < \varepsilon$, где $\varepsilon$ - заданное число, определяющее точность приближения к минимуму.

\newpage
\righthyphenmin=2

% Схема распараллеливания
\section*{Схема распараллеливания}
\addcontentsline{toc}{section}{Схема распараллеливания}
В задании указано, что распараллеливание должно происходить по характеристикам. Таким образом, с помощью функций библиотеки межпроцессного взаимодействия MPI, было решено сделать следующее:
\begin{enumerate}
\item Для существующих векторов $z$ и $x$ в каждом из доступных процессов подсчитать оценку константы Липшица $m$ для каждых двух соседних элементов. Если процессов больше, чем количество элементов, то в последнем процессе найти максимум из оценок константы Липшица для оставшихся отрезков.
\item С помощью функции MPI\_Allreduce операцией MPI\_MAX во все процессы коммуникатора положить максимум из всех локально вычисленных значений $m$. Это и будет глобальной оценкой константы Липшица для векторов $z$ и $x$ на данной итерации цикла. 
\item Аналогичные действия проделать с функциями характеристик, но с учетом следующего фактора: помимо максимального из всех значений функций характеристик по алгоритму нужен также и номер отрезка, для которого была подсчитана максимальная характеристика. Эту проблему можно решить следующим образом:
\begin{itemize}
\item Создать структуру, содержащую в себе 2 поля: первое --- типа double, второе --- типа int. Первое значение --- это локально на каком-либо процессе вычисленное значение функции характеристик, а второе - номер отрезка, на котором производилось вычисление.
\item Вычисление максимума производить по типу MPI\_DOUBLE\_INT с помощью операции MPI\_MAXLOC. Эти параметры определены в библиотеке и обозначают следующее: производятся вычисления по структуре, содержащей только два соответствующих поля (типа double и типа int), а максимум брать по элементу типа double.
\end{itemize}
\end{enumerate}

\newpage

% Описание программной реализации
\section*{Описание программной реализации}
\addcontentsline{toc}{section}{Описание программной реализации}
Структура программы представляет собой класс, выглядящий следующим образом:
\begin{lstlisting}
class Strongin {
 public:
    std::vector<double> x, z;
    int n;
    double a, b;
    double r;
    int prec;
    std::function<double(double)> f;

    Strongin(double _a, double _b, double _r, int _prec, std::function<double(double)> _f);
    void addNode(double xx);
    double calculateM();
    std::vector<double> calculateR(double m);

    double seqStronginSearch();
    double parStronginSearch();
};
\end{lstlisting}
\par $x$ - вектор аргументов, $z$ - вектор значений функции в данных точках, $n$ - текущее количество элементов в векторах $x$ и $z$, $a$ и $b$ - соответственно левый и правый конец отрезка, на котором рассматривается функция, $r$ - заранее заданный параметр алгоритма, $prec$ - точность алгоритма (степень, которую нужно возвести 10 для получения точности), $f$ - сама функция, для которой выполняется алгоритм.
\par Конструктор инициализации начальных параметров алгоритма:
\begin{lstlisting}
Strongin(double _a, double _b, double _r, int prec, std::function<double(double)> _f);
\end{lstlisting}
\par Вход: функция, левый и правый конец отрезка, на котором она рассматривается, параметр алгоритма и точность вычислений.
\par Функция добавления нового узла в векторы $x$ и $z$:
\begin{lstlisting}
void addNode(double xx);
\end{lstlisting}
\par Вход: значение $x$, расположенное между $a$ и $b$.
\par Функция подсчета оценки константы Липшица для текущих векторов (для последовательного алгоритма):
\begin{lstlisting}
double calculateM();
\end{lstlisting}
\par Функция рассчета характеристик для каждого отрезка (для последовательного алгоритма):
\begin{lstlisting}
std::vector<double> calculateR(double m);
\end{lstlisting}
\par Вход: оценка константы Липшица, посчитанная предыдущей функцией. Выход: вектор, содержащий значение характеристики для каждого отрезка.
\par Функция, содержащая реализацию последовательного алгоритма Стронгина:
\begin{lstlisting}
double seqStronginSearch();
\end{lstlisting}
\par Функция, содержащая реализацию параллельного алгоритма Стронгина:
\begin{lstlisting}
double parStronginSearch();
\end{lstlisting}
\par Выходом обеих этих функций является глобальный минимум функции $f$ на отрезке $[a, b]$.
\newpage

% Подтверждение корректности
\section*{Подтверждение корректности}
\addcontentsline{toc}{section}{Подтверждение корректности}
Для подтверждения корректности в программе представлен набор тестов, разработанных с Google Testing Framework.
\par Набор представляет из себя тесты, которые проверяют корректность вычислений (сравниваются матрицы, полученные из последовательного и параллельного алгоритма), а также эффективность (вычисление занимаемого последовательной и параллельной сортировок времени и сравнение полученных данных).
\par Успешное прохождение всех тестов подтверждает корректность работы написанной программы.
\newpage

% Результаты экспериментов
\section*{Результаты экспериментов}
\addcontentsline{toc}{section}{Результаты экспериментов}
Вычислительные эксперименты для оценки эффективности параллельного алгоритма Стронгина проводились на оборудовании со следующей аппаратной конфигурацией:

\begin{itemize}
\item Процессор: Intel Core i5-9400f, 2.90 GHz, 6 ядер;
\item Оперативная память: 16 ГБ;
\item ОС: Майкрософт Windows 10 Home 64-bit Build 19041.685.
\end{itemize}

\par В рамках эксперимента было вычислено время работы последовательного и параллельного алгоритма глобального поиска Стронгина для задач одномерной оптимизации.
\par Результаты экспериментов представлены ниже.

\begin{table}[!h]
\caption{Результаты экспериментов для функции $y = \sqrt{x}*sin(x), x \in [0, 100], \varepsilon = 10^{-12}$}
\centering
\begin{tabular}{lllll}
Процессы & Последовательно & Параллельно & Ускорение  \\
4        & 0.0006763         & 0.0005965     & 1.13       \\
6        & 0.0006630         & 0.0005401     & 1.23       \\
10       & 0.0006624         & 0.0005613     & 1.18       
\end{tabular}
\end{table}

\begin{table}[!h]
\caption{Результаты экспериментов для функции $y = x^3, x \in [-100, 100], \varepsilon = 10^{-12}$}
\centering
\begin{tabular}{lllll}
Процессы & Последовательно & Параллельно & Ускорение  \\
4        & 0.0002098         & 0.0001988     & 1.05       \\
6        & 0.0002124         & 0.0001903     & 1.12       \\
10       & 0.0002077         & 0.0002003     & 1.04       
\end{tabular}
\end{table}

\par По данным экспериментов видно, что параллельный алгоритм Стронгина работает быстрее, но ускорение незначительно ввиду различных накладных расходов на передачу данных между процессами.
\newpage

% Заключение
\section*{Заключение}
\addcontentsline{toc}{section}{Заключение}
В ходе выполнения лабораторной работы был разработан последовательный и параллельный алгоритм глобального поиска Стронгина для задач одномерной оптимизации.
\par По данным экспериментов удалось сравнить время работы параллельной и последовательной реализаций. Выявлено, что параллельный алгоритм работает несколько быстрее, что показывает преимущество параллельных вычислений.
\par Для подтверждения корректности работы программы написаны тесты с использованием Google Testing Framework.
\newpage

% Список литературы
\begin{thebibliography}{1}
\addcontentsline{toc}{section}{Список литературы}
\bibitem{nngu} Центр суперкомпьютерных технологий, официальный сайт. Нижегородский
государственный университет им. Н.И. Лобачевского.  // URL \url {http://www.hpcc.unn.ru/?dir=1034} (дата обращения: 06.12.2020)
\bibitem{spbgu} Реализация и применение параллельного алгоритма глобального поиска минимума к задаче оптимизации параметров молекулярно-динамического потенциала ReaxFF/ К. С. Шефовa, М. М. Степанова. (дата обращения: 06.12.2020)
\bibitem{wiki} Википедия: Метод Стронгина // URL \url {https://wikipedia.org/wiki/} (дата обращения: 07.12.2020)
\end{thebibliography}
\newpage

% Приложение
\section*{Приложение}
\addcontentsline{toc}{section}{Приложение}
В этом разделе находится листинг всего кода, написанного в рамках лабораторной работы.
\begin{lstlisting}
// strongin_method.h

// Copyright 2020 Taktaev Artem
#ifndef MODULES_TASK_3_TAKTAEV_A_STRONGIN_METHOD_STRONGIN_METHOD_H_
#define MODULES_TASK_3_TAKTAEV_A_STRONGIN_METHOD_STRONGIN_METHOD_H_
#include <vector>
#include <functional>

class Strongin {
 public:
    std::vector<double> x, z;
    int n;
    double a, b;
    double r;
    int prec;
    std::function<double(double)> f;

    Strongin(double _a, double _b, double _r, int prec, std::function<double(double)> _f);
    void addNode(double xx);
    double calculateM();
    std::vector<double> calculateR(double m);

    double seqStronginSearch();
    double parStronginSearch();
};

#endif  // MODULES_TASK_3_TAKTAEV_A_STRONGIN_METHOD_STRONGIN_METHOD_H_

\end{lstlisting}
\begin{lstlisting}
// strongin_method.cpp

// Copyright 2020 Taktaev Artem
#include <mpi.h>
#include <cmath>
#include <vector>
#include <functional>
#include <algorithm>
#include "../../../modules/task_3/taktaev_a_strongin_method/strongin_method.h"

Strongin::Strongin(double _a, double _b, double _r, int _prec, std::function<double(double)> _f) {
    if (_a > _b) throw "Must be a < b";
    if (_r <= 1) throw "Must be r > 1";
    if (_prec > -2) throw "More precisely please";
    a = _a;
    b = _b;
    r = _r;
    prec = _prec;
    f = _f;

    n = 2;
    x.resize(n);
    x[0] = a;
    x[1] = b;
    z.resize(n);
    z[0] = f(a);
    z[1] = f(b);
}

void Strongin::addNode(double xx) {
    if ((xx <= a) || (xx >= b)) throw "New node out of range";
    n++;
    x.resize(n);
    z.resize(n);
    int i = 0;
    while (xx > x[i]) i++;
    for (int j = i + 1; j < n; j++) {
        x[j] = x[j - 1];
        z[j] = z[j - 1];
    }
    x[i] = xx;
    z[i] = f(xx);
}

double Strongin::calculateM() {
    double max = (std::abs(z[1] - z[0])) / (x[1] - x[0]);
    for (int i = 2; i < n; i++) {
        if ((std::abs(z[i] - z[i - 1])) / (x[i] - x[i - 1]) > max) {
            max = (std::abs(z[i] - z[i - 1])) / (x[i] - x[i - 1]);
        }
    }
    if (max == 0) return 1.0;
    return r * max;
}

std::vector<double> Strongin::calculateR(double m) {
    std::vector<double> r_vec;
    for (int i = 1; i < n; i++) {
        double r = m * (x[i] - x[i - 1]) + (z[i] - z[i - 1]) * (z[i] - z[i - 1]) / (m * (x[i] - x[i - 1]))
                   - 2 * (z[i] + z[i - 1]);
        r_vec.push_back(r);
    }
    return r_vec;
}

double Strongin::seqStronginSearch() {
    double delta = b - a;
    double eps = pow(10, prec);
    int t = 1;
    double xx;
    while (delta > eps) {
        double m = calculateM();
        std::vector<double> r_vec = calculateR(m);
        double r_max = r_vec[0];
        t = 1;
        for (int i = 2; i < n; i++) {
            if (r_vec[i - 1] > r_max) {
                r_max = r_vec[i - 1];
                t = i;
            }
        }
        xx = (x[t - 1] + x[t]) / 2 - (z[t] - z[t - 1]) / (2 * m);
        addNode(xx);
        delta = x[t] - x[t - 1];
    }
    return xx;
}

struct Arteboss {
    double R;
    int T;

    Arteboss() {
        R = 0;
        T = 1;
    }
};

double Strongin::parStronginSearch() {
    int proc_num, proc_rank;
    MPI_Comm_rank(MPI_COMM_WORLD, &proc_rank);
    MPI_Comm_size(MPI_COMM_WORLD, &proc_num);
    int t = 0;
    double delta = b - a;
    double eps = pow(10, prec);
    double m = 0;
    double _m = 0;
    double xx;

    while (delta > eps) {
        for (int i = 1; i < n; i++) {
            if (proc_rank == (i - 1) % proc_num) {
                if ((std::abs(z[i] - z[i - 1])) / (x[i] - x[i - 1]) > _m) {
                     _m = (std::abs(z[i] - z[i - 1])) / (x[i] - x[i - 1]);
                }
            }
        }
        MPI_Allreduce(&_m, &m, 1, MPI_DOUBLE, MPI_MAX, MPI_COMM_WORLD);
        m = (m == 0 ? 1 : r * m);
        Arteboss r_max;
        Arteboss r1;
        for (int i = 1; i < n; i++) {
            if (proc_rank == (i - 1) % proc_num) {
                if (m * (x[i] - x[i - 1]) + (z[i] - z[i - 1]) * (z[i] - z[i - 1]) / (m * (x[i] - x[i - 1]))
                         - 2 * (z[i] + z[i - 1]) > r1.R) {
                    r1.R = m * (x[i] - x[i - 1]) + (z[i] - z[i - 1]) * (z[i] - z[i - 1]) / (m * (x[i] - x[i - 1]))
                         - 2 * (z[i] + z[i - 1]);
                    r1.T = i;
                }
            }
        }
        MPI_Allreduce(&r1, &r_max, 1, MPI_DOUBLE_INT, MPI_MAXLOC, MPI_COMM_WORLD);
        t = r_max.T;
        xx = (x[t - 1] + x[t]) / 2 - (z[t] - z[t - 1]) / (2 * m);
        addNode(xx);
        delta = x[t] - x[t - 1];
    }
    return xx;
}

\end{lstlisting}
\begin{lstlisting}
// main.cpp
// Copyright 2020 Taktaev Artem
#include <mpi.h>
#include <gtest-mpi-listener.hpp>
#include <gtest/gtest.h>
#include <iostream>
#include <cmath>
#include "../../../modules/task_3/taktaev_a_strongin_method/strongin_method.h"

TEST(Strongin_Method_MPI, Test_Wrong_Parameters_In_Constructor) {
    ASSERT_ANY_THROW(Strongin X(2, -2, 2, -4, [](double x)->double { return std::sin(x); }));
}

TEST(Strongin_Method_MPI, Test_Correct_Parameters_In_Constructor) {
    ASSERT_NO_THROW(Strongin X(-1, 1, 2, -4, [](double x)->double { return std::sin(x); }));
}

TEST(Strongin_Method_MPI, Test_Wrong_X_In_add_Node) {
    Strongin X(-1, 1, 2, -4, [](double x)->double { return std::sin(x); });
    ASSERT_ANY_THROW(X.addNode(5));
}

TEST(Strongin_Method_MPI, Test_Correct_X_In_add_Node) {
    Strongin X(-1, 1, 2, -4, [](double x)->double { return std::sin(x); });
    ASSERT_NO_THROW(X.addNode(0.5));
}

TEST(Strongin_Method_MPI, Test_Correct_Work_Of_seq_Strongin_Method) {
    Strongin Y(-1, 1, 2, -10, [](double x)->double { return std::sin(x); });
    ASSERT_NEAR(Y.seqStronginSearch(), -1, 1.0e-10);
}

TEST(Strongin_Method_MPI, Test_Correct_Work_Of_par_Strongin_Method) {
    Strongin Z(-1, 1, 2, -10, [](double x)->double { return std::sin(x); });
    ASSERT_NEAR(Z.parStronginSearch(), -1, 1.0e-10);
}

TEST(Strongin_Method_MPI, Test_Eff) {
    Strongin A(-1, 1, 2, -10, [](double x)->double { return std::sin(x); });
    Strongin B(-1, 1, 2, -10, [](double x)->double { return std::sin(x); });
    int proc_rank;
    MPI_Comm_rank(MPI_COMM_WORLD, &proc_rank);
    double start, end;
    if (proc_rank == 0) {
        start = MPI_Wtime();
        A.seqStronginSearch();
        end = MPI_Wtime();
        std::cout << "Time sequential = " << end - start << std::endl;
    }
    start = MPI_Wtime();
    B.parStronginSearch();
    end = MPI_Wtime();
    if (proc_rank == 0) {
        std::cout << "Time parallel = " << end - start << std::endl;
    }
}

int main(int argc, char* argv[]) {
    ::testing::InitGoogleTest(&argc, argv);
    MPI_Init(&argc, &argv);

    ::testing::AddGlobalTestEnvironment(new GTestMPIListener::MPIEnvironment);
    ::testing::TestEventListeners &listeners = ::testing::UnitTest::GetInstance()->listeners();

    listeners.Release(listeners.default_result_printer());
    listeners.Release(listeners.default_xml_generator());

    listeners.Append(new GTestMPIListener::MPIMinimalistPrinter);
    return RUN_ALL_TESTS();
}

\end{lstlisting}

\end{document}
